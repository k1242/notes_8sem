\subsection*{Т8. Электронный газ}


И снова смотрим на уравнение Больцмана, ищём решение в виде $f = f_0 + \delta f$, смотрим на $\tau$-приближение, равновесным будет распределение Ферми:
\begin{equation*}
	f_0 = \frac{1}{e^{\frac{\varepsilon-\mu}{T}} + 1},
	\hspace{10 mm} 
	\mu = \mu (t, \vc{r}),
	\hspace{5 mm} 
	T = T(t, \vc{r}).
\end{equation*}
Будем решать уравнение рассматривая стационарный случай
\begin{equation*}
	\vc{v} \cdot \frac{\partial f_0}{\partial \vc{r}} - e \vc{E} \cdot \vc{v} \frac{\partial f_0}{\partial \varepsilon}  = - \frac{\delta f}{\tau}.
\end{equation*}
Можем переписать 
\begin{equation*}
	\frac{\partial f_0}{\partial \vc{r}} = \frac{\partial f_0}{\partial T}  \vc{\nabla} T + \frac{\partial f_0}{\partial \mu} \vc{\nabla} \mu = - \frac{\varepsilon-\mu}{T} \frac{\partial f}{\partial \varepsilon} \vc{\nabla} T - \frac{\partial f_0}{\partial \varepsilon} \nabla \mu.
\end{equation*}
Тогда, после подстановки, левая часть уравнения может быть найдена в виде
\begin{equation*}
	\delta f = \tau \left(
		\frac{\varepsilon-\mu}{T} (\vc{v} \cdot \vc{\nabla} T) + \vc{v} \cdot (\vc{\nabla} \mu + e \vc{E})
	\right) \frac{\partial f_0}{\partial \varepsilon} .
\end{equation*}


\textbf{Металл}. Достаточно рассмотреть $- \frac{\partial f_0}{\partial \varepsilon} \approx \delta(\varepsilon - \varepsilon_F)$. Для тока $\vc{j}$ находим
\begin{equation*}
	\vc{j} = - e \int \vc{v} (f_0 + \delta f) \frac{2 \d^3 p}{(2\pi \hbar)^3} = \frac{e}{3} (\vc{\nabla} \mu + e \vc{E}) \int \tau v^2 \left(
		- \frac{\partial f_0}{\partial \varepsilon} 
	\right) \frac{2 \d^3 p}{(2\pi \hbar)^3} + \frac{e}{3} \frac{\vc{\nabla} T}{T} \int \tau v^2 (\varepsilon-\mu) \left(-\frac{\partial f_0}{\partial \varepsilon} \right) \frac{2 \d^3 p}{(2\pi \hbar)^3}.
\end{equation*}
Для плотности потока энергии $\vc{q}$ 
\begin{equation*}
	\vc{q} = \int \vc{v} (\varepsilon- e \varphi) (f_0 + \delta f) \frac{2 \d^3 p}{(2\pi \hbar)^3} = - \frac{\vc{j}}{e} (\mu - e \varphi) - \ldots.
\end{equation*}
% вставить из фото!
Введём диссипативную часть $\vc{q}'$
\begin{equation*}
	\vc{q}' = \vc{q} + \frac{\vc{j}}{e} (\mu- e \varphi).
\end{equation*}
Также определим усреднение в виде
\begin{equation*}
	\langle F(\varepsilon)\rangle = \frac{m}{3n} \int \frac{2 \d^3 p}{(2 \pi \hbar)^3} v^2  \left(- \frac{\partial f_0}{\partial \varepsilon} \right)  F(\varepsilon) = \frac{2}{3n} \int_{0}^{\infty} \varepsilon \left(- \frac{\partial f_0}{\partial \varepsilon} \right) F(\varepsilon) g(\varepsilon) \d \varepsilon,
	\hspace{10 mm} 
	n = \int_{0}^{\infty} \varepsilon \left(- \frac{\partial f_0}{\partial \varepsilon} \right) g(\varepsilon) \d \varepsilon.
\end{equation*}
Тогда уравнение перепишется в виде
\begin{equation*}
	\vc{E} + \frac{\vc{\nabla} \mu}{e} = \frac{m \vc{j}}{n e^2 \langle \tau\rangle} - \frac{\vc{\nabla} T}{ e T} \frac{\langle (\varepsilon -\mu) \tau\rangle}{\langle \tau\rangle} = \frac{\vc{j}}{\sigma} + \alpha \vc{\nabla} T.
\end{equation*}
Тогда для потока энергии
\begin{equation*}
	\vc{q}' = - \frac{\langle (\varepsilon-\mu) \tau\rangle}{e \langle  \tau\rangle} \vc{j} + \frac{\vc{\nabla} T}{m T} \frac{n \langle (\varepsilon-\mu) \tau\rangle^2}{\langle \tau\rangle} - 
	\frac{\vc{\nabla} T}{m T} n \langle (\varepsilon-\mu)^2 \tau\rangle 
	= 
	\alpha T \vc{j} - \varkappa \vc{\nabla} T
	.
\end{equation*}
Где коэффициенты соответственно равны
\begin{equation}
	\alpha = - \frac{\langle (\varepsilon-\mu) \tau \rangle}{e T \langle \tau\rangle},
	\hspace{10 mm} 
	\varkappa = \frac{n \langle \tau\rangle}{m T}\left[
		\frac{\langle (\varepsilon-\mu)^2 \tau\rangle}{\langle \tau\rangle} - \frac{\langle (\varepsilon-\mu) \tau\rangle^2}{\langle \tau\rangle^2}
	\right],
	\hspace{10 mm} 
	\sigma = \frac{n e^2 \langle \tau\rangle}{m}.
\end{equation}
где $\varkappa$ -- коэффициент теплопроводности, $\alpha$ -- термоэлектрический коэффициентр, $\sigma$ -- проводимость. 


% Соотношения Онзагера можем получить, записав с фото
% \begin{equation*}
% 	\vc{j} = 
% \end{equation*}




\textbf{Полупроводник}. Здесь можем написать, что $\frac{\partial f_0}{\partial \varepsilon} = - \frac{\partial f_0}{\partial T}$, так как $f_0 \approx e^{(\mu-\varepsilon)/T}$. Тогда усредение можем переписать в виде
\begin{equation*}
	\langle F(\varepsilon)\rangle  = \frac{m}{3 n T} \frac{2 \d^3 p}{(2 \pi \hbar)^3} f_0 v^2 F(\varepsilon).
\end{equation*}
Считая, что $\tau(\varepsilon) \propto v^k \propto \varepsilon^{k/2}$ и что $f_0 \propto e^{-\frac{m v^2}{2T}}$, находим
\begin{equation*}
	\langle v^k\rangle \propto \left(\frac{2 T}{m}\right)^{k/2} \Gamma\left(\frac{3+k}{2}\right).
\end{equation*}
Так, например, для $\alpha$ получится
\begin{equation*}
	\alpha = \frac{1}{e} \left(
		\frac{\mu}{T} - \frac{\langle \tau v^2 \varepsilon\rangle}{\langle  \tau v^2\rangle}
	\right) = \frac{1}{e} \left(
		\frac{\mu}{T} - \frac{\Gamma\left(\frac{1+k}{2}\right)}{\Gamma\left(\frac{3+k}{2}\right)}
	\right) = \frac{1}{e} \left(
		\frac{\mu}{T} - \frac{5+k}{2}
	\right).
\end{equation*}
% êîýôôèöèåíòû Ïåëüòüå

% Для полупроводника получается коэффициент плетье гораздо больше.
% ДЗ: найти коэффициент плетье в металле, должно получиться \mu/E_F 


% число Лоренца -- винеман -франц закон вывести
% 