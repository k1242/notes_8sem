\begin{itemize}
    \item Посмотреть, как связаны $\eta$ и $\sub{N}{\scalebox{0.75}{MOT}}(\sub{t}{load})$.
\end{itemize}


\subsection*{Динамика количества атомов в МОЛ}

Количество атомов в ловушке $N$ после загрузки описывается уравнением \cite{vlad}
\begin{equation*}
	\frac{d N}{d t} = - \gamma N - \beta \int_V n(\vc{r}, t)^2 \d^3 \vc{r},
\end{equation*}
где $\gamma$ -- коэффициент линейных потерь, обусловленных столкновениями с буферным газом, $\beta$ -- скорость неэластичных бинарных столкновений, $n(\vc{r}, t)$ -- концентрация атомов, $V$ -- объем атомного облака. 



