Полный гамильтониан системы
\begin{equation*}
	\hat{H} = \hat{H}_s + \hat{H}_r + \hat{H}_{sr}.
\end{equation*}
Знаем уравнение на полную матрицу плотности
\begin{equation*}
	\frac{\partial \hat{\rho}}{\partial t}  = \frac{i}{\hbar}\left[\hat{\rho}, \hat{H}\right],
\end{equation*}
далее будем искать уравнение на $\hat{\rho}_s = \tr_r \hat{\rho}$. Будем работать в впредставление взаимодействия
\begin{equation*}
	\hat{\tilde{\rho}} = \exp\left(\frac{it}{\hbar}\left(\hat{H}_s + \hat{H}_r\right)\right) \hat{\rho} \exp\left(
		\frac{-it}{\hbar}(\hat{H}_s + \hat{H}_r)
	\right).
\end{equation*}
Можем переписать тогда в виде
\begin{equation*}
	\frac{\partial \hat{\tilde{\rho}}}{\partial t} = \frac{i}{\hbar}\left[
		\hat{\tilde{\rho}}(t_0),\, \hat{\tilde{H}}_{sr}(t)
	\right] + \left(\frac{i}{\hbar}\right)^2 \int_{t_0}^{t} \left[
		\left[
			\hat{\tilde{\rho}}(t'),\, \hat{\tilde{H}}_{sr}(t')
		\right],\, 
		\hat{\tilde{H}}_{sr}(t)
	\right] \d t'.
\end{equation*}
Будем считать, что взаимодействие адиабатически включалось, тогда можно забыть про первое слагаемое. Также считаем, что 
\begin{equation*}
	\hat{\tilde{\rho}}(t) \approx \hat{\tilde{\rho}}_s(t) \otimes \hat{\tilde{\rho}}_r(t),
	\hspace{10 mm} 
	\hat{\tilde{\rho}}(t) = \hat{\tilde{\rho}}_s(t) \otimes \hat{\tilde{\rho}}_r(-\infty),
\end{equation*}
то есть резервуар в термодинамическом равновесии и слабо взаимодействует с системой. В итоге
\begin{equation*}
	\frac{\partial \hat{\tilde{\rho}}_s(t)}{\partial t} \approx \left(\frac{i}{\hbar}\right)^2 \int_{-\infty}^{t} \d t' \tr \left(
		\left[
			\left[
				\hat{\tilde{\rho}}_s(t') \otimes \hat{\tilde{\rho}}_r(-\infty),\, \hat{\tilde{H}}_{sr}(t')
			\right],\, 
		\hat{\tilde{H}}_{sr}(t)
	\right] 
	\right).
\end{equation*}
Явно учитывая вид $\hat{H}$, можем найти коммутаторы. Введя
\begin{equation*}
	\hat{F}(t) = \sum_\alpha \gamma_\alpha \hat{b}_\alpha e^{i(\omega_s-\omega_\alpha)t},
	\hat{\tilde{H}}_{sr}(t) = \hbar\left(\hat{F}\con(t) \hat{a} + \hat{F}(t) \hat{a}\con\right),
\end{equation*}
посчитав корреляторы вида$\langle \hat{F}\con(t) \hat{F}(t')\rangle_r$, интегралы от них, и вернувшись к представлению Шрёдингера, получаем выражение
\begin{align*}
	\frac{\partial \hat{\rho}_s(t)}{\partial t} &= \frac{i}{\hbar}\left[
		\hat{\rho}_s, \hat{H}_s + \hbar \Delta_1 \hat{a} \hat{a}\con - \hbar (\Delta_1 + \Delta_2) \hat{a}\con \hat{a}
	\right] 
	\\&+ 
	\gamma_s\left(
		n(\omega_s) + 1
	\right)\left(
		2 \hat{a} \hat{\rho}_s(t) \hat{a}\con - \{\hat{\rho}_s(t),\, \hat{a}\con \hat{a}\}
	\right) + \gamma_s n(\omega_s) \left(
		2 \hat{a}\con \hat{\rho}_s (t) \hat{a} - \{\hat{\rho}_s(t),\, \hat{a} \hat{a}\con\}
	\right).
\end{align*}
где введены обозначения
\begin{equation*}
	\gamma_s = \pi D(\omega_s) |\gamma(\omega_s)|^2,
	\hspace{5 mm} 
	D(\omega) = \sum_\alpha \delta(\omega-\omega_\alpha),
\end{equation*}
а также
\begin{equation*}
	\Delta_1 = P \int_{0}^{\infty} D(\omega) |\gamma(\omega)|^2 n(\omega) \frac{1}{\omega-\omega_s} \d \omega,
	\hspace{10 mm} 
	\Delta_2 = P \int_{0}^{\infty} D(\omega) |\gamma(\omega)|^2 \frac{1}{\omega-\omega_s} \d \omega.
\end{equation*}
Таким образом добавление резервуара привело к перенормировки гамильтониана
\begin{equation*}
	\hat{H}_S = \hbar \Delta_1 + \hbar (\omega_s - \Delta_2) \hat{a}\con \hat{a},
\end{equation*}
и возникновения диссипативного члена:
\begin{equation*}
	\frac{\partial \hat{\rho}_s(t)}{\partial t} = \frac{1}{i\hbar}\left[\hat{H}_S, \hat{\rho}_s\right] + \gamma_s\left(
		n(\omega_s) + 1
	\right)\left(
		2 \hat{a} \hat{\rho}_s(t) \hat{a}\con - \{\hat{\rho}_s(t),\, \hat{a}\con \hat{a}\}
	\right) + \gamma_s n(\omega_s) \left(
		2 \hat{a}\con \hat{\rho}_s (t) \hat{a} - \{\hat{\rho}_s(t),\, \hat{a} \hat{a}\con\}
	\right).
\end{equation*}
Общий вид уравнения Линдблада:
\begin{equation*}
	\frac{\partial \hat{\rho}(t)}{\partial t} = \frac{1}{i\hbar} \left[\hat{H},\, \hat{\rho}\right] + \frac{1}{2} \sum_k \left(
		\left[\hat{L}_k \hat{\rho},\, \hat{L}\con_k\right] + 
		\left[\hat{L}_k,\, \hat{\rho} \hat{L}_k\con\right]
	\right),
\end{equation*}
сводится к полученному для $\hat{L}_1 \approx \hat{a}\con$ и $\hat{L}_2 \approx \hat{a}$.