% -- аннотация (отражаются цели и задачи работы и полученные результаты, в
% дальнейшем размещается в открытом доступе, не более 1500 знаков);


В работе исследованы методы охлаждения атомов и их загрузки в магнито-оптическую ловушку (МОЛ) с помощью зеемановского замедлителя (ЗЗ) и оптической патоки (ОП). Была построена модель охлаждения атомов с помощью ЗЗ и их дальнейшей загрузки в систему ОП+МОЛ. На основе построенной модели оптимизированы параметры магнитного поля и лазерного излучения ЗЗ, ОП, МОЛ, что привело к повышению эффективности охлаждения и позволило уменьшить температуру печи и увеличить время непрерывной работы установки в 5 раз. Также была рассмотрена альтернатива ЗЗ - двухмерная магнито-оптическая ловушка (2D-МОЛ), которая представляет более компактную альтернативу ЗЗ при сохранении потока загрузки МОЛ. 