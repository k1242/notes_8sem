В работе было рассмотрено замедление атомарного пучка с помощью зеемановского замедлителя (ЗЗ) и последующая загрузка в магнито-оптическую ловушку (МОЛ) с дополнительным охлаждением оптической патокой (ОП).  Были рассмотрены основные принципы работы ЗЗ и МОЛ, построена модель охлаждения атомов с помощью ЗЗ и последующая загрузка в систему ОП+МОЛ. В ходе работы на основе построенной модели были оптимизированы параметры магнитного поля и лазерного излучения ЗЗ, ОП и МОЛ, что привело к повышению эффективности охлаждения атомов. Была произведена калибровка температуры печи двумя способами, что позволило точнее управлять параметрами системы. В ходе работы удалось уменьшить температуру печи с $730\dC$ до $680\dC$ и, соответственно, увеличить время непрерывной работы установки в 5 раз с 2 месяцев непрерывной работы до 10 месяцев непрерывной работы, при этом сохранив поток загрузки МОЛ $\sub{\Phi}{load} \sim 10^{8}\,\text{c}^{-1}$. На текущий момент установка уже непрерывно работает 4 месяца.

Также в работе рассмотрена альтернатива ЗЗ, также позволяющая увеличить время жизни установки: 2D-МОЛ. Построена модель замедления атомов с помощью 2D-МОЛ, произведены основные оценки работы 2D-МОЛ в нашей установке. С помощью 2D-МОЛ возможно перейти к более стабильной и компактной установке, сохранив поток загрузки МОЛ, что делает 2D-МОЛ перспективной альтернативой ЗЗ.




