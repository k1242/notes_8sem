\subsection{Общая идея}

Рассмотрим два куска металла между которыми существует 1D идеальный провод. Химпотенциалы соответственно равны
\begin{equation*}
	\mu_L = \mu + \frac{1}{2} eV, 
	\hspace{5 mm} 
	\mu_R = \mu - \frac{1}{2} eV,
\end{equation*}
ток можем найти как $I = I_R-I_L$:
\begin{equation*}
	I = \sum_{k > 0} e v_k \left(
		f_R(\varepsilon_k) - f_L (\varepsilon_k)
	\right),
\end{equation*}
где $f_{R, L} (\varepsilon_k) = f(\varepsilon_k - \mu_{R, L})$ -- числа заполнения. 
Подставляя в выражение для тока, находим
\begin{equation*}
	I = -\sum_{k > 0} e v_k \frac{\partial f}{\partial \varepsilon_k} \left(
		\delta \mu_L - \delta \mu_R
	\right) = - e^2 V \int_{k>0} \frac{d k}{2\pi} \frac{\partial \varepsilon_k}{\partial \hbar k}  \frac{\partial f}{\partial \varepsilon_k} = - \frac{e^2}{2\pi} V \int_{0}^{\infty} \d \varepsilon \frac{\partial f}{\partial \varepsilon} =  \frac{e^2}{2\pi \hbar} V.
\end{equation*}
где сделали подстановку $v_k = {\partial \varepsilon_k}/{\partial \hbar k}$, числа заполнения равны $f(\varepsilon=0)=1$ и $f(\varepsilon=\infty)= 0$ соответственно. Таким образом находим квант проводимости
\begin{equation*}
	G = \frac{I}{V} = \frac{e^2}{2 \pi \hbar}.
\end{equation*}
Если скажем, что электроны отражаются с коэффициентом $|t_i|^2$ и всего всего есть $N$ одномерных каналов, получим \textit{формулу Ландауэра}
\begin{equation*}
	G = \frac{e^2}{2\pi \hbar} \sum_{i=1}^N |t_i|^2.
\end{equation*}





\subsection{Подход Ландауэра}


Рассмотрим точечный контакт двух проводников. Пусть хим. потенциал резервуаров $\mu_1$ и $\mu_2$, функция распределения при температуре $\Theta$
\begin{equation*}
	f_{\alpha}(E) = \left(e^{(E-\mu_{\alpha})/\Theta}+1\right)^{-1},
	\hspace{5 mm} 
	\alpha = 1,2.
\end{equation*}
Будем считать систему двухмерной, ось $x$ вдоль течения тока, тогда уравнение Шрёдингера на стационарные волновые функции имеет вид
\begin{equation*}
	\hat{H} \psi = E \psi,
	\hspace{10 mm} 
	\hat{H} = - \frac{\hbar^2}{2m}(\partial_x^2 + \partial_y^2) + U(x, y).
\end{equation*}
Считаем, что расстояние между стенками меняется как $W(x)$, тогда для $W = \const $ можем явно найти $\psi(x, y) = \varphi(x) \varphi(y)$ и 
\begin{equation*}
	\varphi_n(y) = \sqrt{\frac{2}{W}} \sin\left(
		\pi n\left(\frac{y}{W} + \frac{1}{2}\right)
	\right),
\end{equation*}
для которых верно, что
\begin{equation*}
	-\left(\frac{\hbar^2}{2m} \partial_y^2 + E\right) \varphi(y) = - \tilde{E} \varphi(y),
	\hspace{10 mm} 
	\tilde{E} = E - U_n,
	\hspace{5 mm} 
	U_n = \frac{(\pi n \hbar)^2}{2 m W^2}.
\end{equation*}
Для адиабатического приближения можем подставить $W=W(x)$. Таким образом эффцективный потенциал имеет вид потенциального барьера высоты $E_n = (\pi n \hbar)^2/(2m W_0^2)$, где $W_0 = \min W(x)$. 


Введём вероятности отражения и прохождения $R_{nm}$ и $T_{nm}$ из канала $m$ в канал $n$, тогда
\begin{equation*}
	I = 2 \sum_{n,m} \int_{0}^{\infty} \frac{\d E}{2 \pi \hbar v_n} e v_n \left(
		f_1(E) (\delta_{nm}-R_{nm}) - f_2(E)T_{nm}
	\right).
\end{equation*}
Для линейного кондактанса $G = dI /dV$ при $V \to 0$, пределе нулевой температуры $\mu_1 = E_F$, $\mu_2 = E_F-eV$ и $f(E) = \theta(\mu-E)$ найдём
\begin{equation*}
	G = 2 \sum_{n.m} \int_{0}^{\infty} \frac{e^2 \d E}{2 \pi \hbar} \delta\left(
		E_F - eV - E
	\right) T_{nm} (E) = \frac{e^2}{\pi \hbar} \sum_{n,m} T_{nm}(E_F).
\end{equation*}
Появился $G_q = e^2 / (\pi \hbar)$ -- квантовый кондактанс. 



% \subsection{Метод вторичного квантования}

\subsection{Поток тепла}

Запишем в линейном приближение ток и поток тепла $I_Q$
\begin{equation*}
	I = \frac{2e}{h} \int_{0}^{\infty} (f_1(E)-f_2(E))T(E)\d E= G V + L \,\delta \Theta,
	\hspace{5 mm} 
	I_Q = \frac{2}{h} \int_{0}^{\infty} (f_1(E)-f_2(E))(E-\mu)T(E)\d E = L' V + K\, \delta \Theta,
	\hspace{10 mm} 
	\delta \Theta = \Theta_1 - \Theta_2,
	\hspace{5 mm} 
	\mu_1-\mu_2= eV,
\end{equation*}
где ввели кинетические коэффициенты $G,\, L,\, L'$ и $K$. 

Раскладываясь по температуре, находим
\begin{equation*}
	I_Q / \delta \Theta = K \approx \frac{G}{e^2} \int_{0}^{\infty}  \frac{\partial f(E)}{\partial \Theta} (E-\mu) \d E = \frac{\pi^2}{3} \frac{G \Theta}{e^2}.
\end{equation*}
По сути это закон Видемана-Франца в применении к точечному контакту. 

Аналогично находим $L' = L \Theta$, что является проявлением принципа симметрии кинетических коэффициентов Онсагера. И, наконец, для $L$:
\begin{equation*}
	L = G \frac{\pi^2}{3} \frac{\Theta}{e} \frac{\partial \ln T}{\partial E}.
\end{equation*}




% \subsection{Время релаксации}

% Из золотого правила Ферми
% \begin{equation*}
% 	I_{\vc{k}}[f] = \frac{2\pi}{\hbar}\sum_{\vc{k} \in \Omega} |\bk{\vc{k}'}[\mathcal U]{\vc{k}}^|^2 (f_{\vc{k}'} - f_{\vc{k}}) \delta(\varepsilon_{\vc{k}} - \varepsilon_{\vc{k}'}),
% \end{equation*}
% где ввели сумму потенциалов индивидуальных примесей
% \begin{equation*}
% 	\mathcal U(\vc{r}) = \sum_{j=1}^{\sub{N}{imp}} U(\vc{r}-\vc{R}_j).
% \end{equation*}
% Считая примеси нескоррелированными, можем получить
% \begin{equation*}
% 	\overline{|\bk{\vc{k}'}[\mathcal U]{\vc{k}}^|^2} = \frac{\sub{N}{imp}}{V^2}|\hat{U}(\vc{k}'-)
% \end{equation*}













% \begin{equation*}
% 	\hat{H} = -J \sum_n \left(\kb{n}{n+1}+\kb{n+1}{n}\right) + \sum_n \varepsilon_n \kb{n}{n},
% 	\hspace{5 mm} 
% 	\delta_n 
% \end{equation*}

% \begin{equation*}
% 	\varepsilon_n = \tfrac{1}{2}W \cos(\sigma n)
% \end{equation*}


% \begin{equation*}
% 	\varepsilon_n \in [0, W] \hspace{5 mm} 
% 	W=0.1 \hspace{5 mm} W=0.5
% \end{equation*}

% \begin{tabular}{cl}
%  $J$ & туннелирование \\
%  $U$ & не дружат в одном узле \\
%  $V$ & внешний потенциал \\
%  $\delta$ & шум
% \end{tabular}

% \begin{tabular}{ll}
%  $J$ & масщтаб энергии \\
%  $\tau=\hbar/J$ & масштаб времени 
% \end{tabular}