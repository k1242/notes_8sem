\subsection{Сведение к осциллятору}

Работаем примерно с уравнением 
\begin{equation*}
	\frac{\partial n}{\partial t} = D \Delta n + \div(b n \nabla U),
\end{equation*}
точнее с уравнением вида
\begin{equation*}
	\frac{\partial P}{\partial t} = \frac{D}{T} k P + \frac{D}{T} k x \frac{\partial P}{\partial x}  + D \frac{\partial^2 P}{\partial x^2} ,
\end{equation*}
где подставили потенциал $U = kx^2/2$.

Введём $g=2 \gamma T$ и $\tau = b t$, тогда можем сделать подстановку
\begin{equation*}
	P(x, \tau) = e^{-\gamma k x^2/2g} \psi(x, \tau),
\end{equation*}
получаем уравнение вида
\begin{equation*}
	\frac{1}{k} \frac{\partial \psi}{\partial \tau} = \left(
		\frac{1}{2} - \frac{x^2}{4A}
	\right) \psi + A \frac{\partial^2 \psi}{\partial x^2},
	\hspace{10 mm} 
	A = \frac{g}{2k \gamma} = \frac{T}{k}. 
\end{equation*}
Таким образом пришли к гамильтониану гармонического осциллятора, с собственными функциями в виде полиномов эрмита
\begin{equation*}
	\varphi_n (x) = \frac{1}{\sqrt{2^n n! \sqrt{2\pi A}}} H_n\left(\frac{x}{\sqrt{2A}}\right) e^{-x^2/4A}.
\end{equation*}
Итого, искомая вероятность
\begin{equation*}
	P(x, \tau) = e^{-\gamma k x^2/2g} \psi(x, \tau) = e^{-x^2/4A} \psi(x, \tau) = \sum_{n=0}^{\infty} a_n e^{- n k \tau} \varphi_0(x) \varphi_n(x).
\end{equation*}


\subsection{Забываются начальные условия}

Забавный факт:
\begin{equation*}
	\sum_{n=0}^{\infty} \frac{H_n(x) H_n(y)}{n!} \left(\frac{U}{2}\right)^n = \frac{1}{\sqrt{1-U^2}} \exp\left(
		\frac{2 u xy}{1+u} - \frac{u^2 (x-y)^2}{1-u^2}
	\right),
\end{equation*}
и воспользуемся соотношением ортогональности, что найти эволюцию от $P(x, 0) = \delta(x-x_0)$:
\begin{equation*}
	a_n = \frac{1}{\sqrt{2^n n!}} H_n\left(\frac{x_0}{\sqrt{2A}}\right) = \frac{\varphi_n(x_0)}{\varphi_0(x_0)}.
\end{equation*}
Итого, эволюция запишется в виде
\begin{align*}
	P(x, \tau) &= \sum_{n=0}^{\infty} e^{-n k \tau} \frac{\varphi_0(x)}{\varphi_0(x_0)} \varphi_n(x) \varphi_n(x_0)
	= \frac{1}{\sqrt{2\pi A}} e^{-x^2/2A} \sum_{n=0}^{\infty} \frac{1}{2^n n!} e^{-n k \tau} H_n\left(\frac{x_0}{\sqrt{2A}}\right) H_n \left(\frac{x}{\sqrt{2A}}\right) \\ 
	&= \frac{1}{\sqrt{2\pi A(1-e^{-2k \tau})}} \exp\left(
		- \frac{(x-x_0 e^{- k \tau})}{2A(1-e^{-2 k \tau})}
	\right),
\end{align*}
где подставили ту сумму с $u = e^{- k \tau}$. Таким образом начальные условия забываются!


% Теперь работаем с непрерывным пространство, поэтому ПОВМ будем искать в виде
% \begin{align*}
% 	\Pi_0 &= \int p(0, x) \kb{0}{x} \d x, \\
% 	\Pi_1 &= \int p(1, x) \kb{1}{x} \d x.
% \end{align*}
% Для удобства введём функцию $I_\Delta (x)$:
% \begin{equation*}
% 	I_\Delta (x) = \left\{\begin{aligned}
% 	    &0, &|x|>\Delta, \\
% 	    &1, &|x|<\Delta.
% 	\end{aligned}\right.
% \end{equation*}
% Тогда искомые вероятности запишутся в виде
% \begin{align*}
% 	p(1, x) &= p \cdot I_\Delta(x), \\
% 	p(0, x) &= (1-p)I_\Delta(x) + (1-I_\Delta(x)).
% \end{align*}
% \begin{align*}
% 	\hat{\Omega}_{0} &= \sqrt{p} \kb{0}{0} + \sqrt{1-p} \kb{1}{1} \\
% 	\hat{\Omega}_{1} &= \sqrt{1-p} \kb{0}{0} + \sqrt{p} \kb{1}{1} \\
% \end{align*}

% \begin{equation*}
% 	P_0 = P_1 = \frac{1}{2}\left(\frac{1}{2} - \sqrt{(1-p)(p)}\right)
% \end{equation*}

Состояние с заданной $x$-компонентной спина -- собственное для $\hat{\sigma}_x$: $\psi \sim (\pm 1, 1)$, а значит 
\begin{equation*}
	\psi(t) \sim \begin{pmatrix}
		\pm e^{\mp i \Omega t/2} \\ e^{\mp i \Omega t/2}
	\end{pmatrix},
	|\psi(t)|^2 = \const,
\end{equation*}
то есть ситема будет равновероятно наблюдаться в состояние $\ket{0}$ или $\ket{1}$. 

Для постоянных измерений будем работать с системой, вида
\begin{equation*}
	\dot{\rho}_x = - 2 \gamma \rho_x,
	\hspace{5 mm} 
	\dot{\rho}_y = -\Omega \rho_z - 2 \gamma \rho_y,
	\hspace{5 mm} 
	\dot{\rho}_z = \Omega \rho_y,
\end{equation*}
которая очевидно для $\rho_x = 1$ будет иметь решение, вида
\begin{equation*}
	\rho_x (t) = e^{-2\gamma t}.
\end{equation*}