Запишем функцию Лагранжа
\begin{equation*}
	L(\dot{q}, q, \{\dot{x}_\alpha, x_\alpha\}) = \frac{M \dot{q}^2}{2} - U_0 (q) + \sum_\alpha \left(
		\frac{m \dot{x}_\alpha^2}{2} - \frac{m \omega_\alpha^2 x_\alpha^2}{2}
	\right) - q \sum_\alpha C_\alpha x_\alpha.
\end{equation*}
Теперь уравнения дваижения -- уравнения Лагранжа
\begin{equation*}
	M \ddot{q} + U'_0 (q) = - \sum_\alpha C_\alpha x_\alpha,
	\hspace{10 mm} 
	m \ddot{x}_\alpha + m \omega^2_\alpha x_\alpha = - q C_\alpha.
\end{equation*}
Решение для фононов можем записать в виде
\begin{equation*}
	x_\alpha(t) = - C_\alpha \int_{-\infty}^{t} \frac{\sin (\omega_\alpha (t-s))}{m \omega_\alpha} q(s) \d s + 
	x_\alpha(0) \cos \omega_\alpha t + \frac{\dot{x}_\alpha(0)}{\omega_\alpha} \sin (\omega_\alpha t).
\end{equation*}
Введем также функцию запаздывающего отклика
\begin{equation*}
	K_\alpha(t) = - \theta(t) \frac{C_\alpha \sin (\omega_\alpha t)}{m \omega_\alpha},
	\hspace{10 mm} 
	K_\alpha(\omega) = - \frac{C_\alpha}{m(\omega_\alpha^2 - (\omega+i \delta)^2)},
\end{equation*}
с полюсами $\omega = \pm \omega_\alpha - i \delta$, $\delta \to +0$. 

Можем ввести силу со стороны термостата на частицу
\begin{equation*}
	F(t) = \int_{-\infty}^{+\infty} K(t-s) q(s) \d s + f(t),
	\hspace{10 mm} 
	f(t) = - \sum_\alpha C_\alpha \left(
		x_\alpha(0) \cos \omega_\alpha t + \frac{\dot{x}_\alpha(0)}{\omega_\alpha} \sin \omega_\alpha t
	\right),
\end{equation*}
где $K(t)= - \sum_\alpha C_\alpha K_\alpha(t) = \theta(t) \sum_\alpha C_\alpha^2 \frac{\sin (\omega_\alpha t)}{m \omega_\alpha}$. Для Фурье-образа:
\begin{equation*}
	K(\omega) = \frac{2}{\pi} \int_{0}^{\infty}  \frac{\Omega^2}{\Omega^2- (\omega+i\delta)^2} \frac{J(\Omega)}{\Omega} \d \Omega,
	\hspace{10 mm} 
	J(\Omega) = \frac{\pi}{2} \sum_\alpha \frac{C_\alpha^2}{m \omega_\alpha} \delta(\Omega-\omega_\alpha).
\end{equation*}
Итого, для движения частицы
\begin{equation*}
	M \ddot{q} + U'_0 (q) = \int_{-\infty}^{+\infty} K(t-s) q(s) \d s + f(t).
\end{equation*}
Можем представить $K(\omega) = K_0 + i \eta \omega$ или
\begin{equation*}
	K(t) = K_0 \delta(t) - \eta \delta'(t), 
	\hspace{10 mm} 
	J(\Omega) = \eta \Omega, 
	\hspace{5 mm} 
	K_0 = \sum_\alpha \frac{C_\alpha^2}{m \omega_\alpha^2},
\end{equation*}
что приведёт для уравнений движения
\begin{equation}
\boxed{
	M \ddot{q} + U'_0 (q) = K_0 q(t) - \eta \dot{q}(t) + f(t)
}
\end{equation}
что формально соответствует трению $\eta$, перенормировке потенциала $U(q) = U_0 (q) - K_0 q^2/2$ и добавлению случайной силы $f(t)$. 

\textbf{Коррелятор}. Найдём корреляционную функцию случайной силы
\begin{equation*}
	\langle f(t) f(t')\rangle = \sum_{\alpha, \beta} C_\alpha C_\beta \langle \left(
	x_\alpha(0) \cos \omega_\alpha t + \frac{\dot{x}_\alpha(0)}{\omega_\alpha} \sin \omega_\alpha t
	\right)\left(
	x_\beta(0) \cos \omega_\beta t + \frac{\dot{x}_\beta(0)}{\omega_\beta} \sin \omega_\beta t
	\right)
	\rangle.
\end{equation*}
Так как фононы независимы друг от друга, то можем переписать в виде
\begin{equation*}
	\langle f(t) f(t')\rangle = \sum_{\alpha, \beta} C_\alpha^2 \left(
		\langle x_\alpha(0)^2\rangle \cos(\omega_\alpha t) \cos(\omega_\alpha t') + \frac{\langle \dot{x}_\alpha(0)^2\rangle}{\omega_\alpha^2} \sin(\omega_\alpha t) \sin(\omega_\alpha t')
	\right).
\end{equation*}
Здесь учли, что среднее от полной производной по времени равно нулю. Для скоростей
\begin{equation*}
	\langle \dot{x}_\alpha^2(0)\rangle = \omega_\alpha^2 \langle x_\alpha^2(0)\rangle = \frac{2}{m} \frac{1}{2} \left(\bar{n}_\alpha + \frac{1}{2}\right) \hbar \omega_\alpha = \frac{\hbar \omega_\alpha}{2m} \cth \frac{\hbar \omega_\alpha}{2T}.
\end{equation*}
И для временного коррелятора
\begin{align*}
	\langle f(t) f(t')\rangle 
	&= \sum_\alpha C_\alpha^2 \frac{\hbar}{2 m \omega_\alpha} \left(
		\cth\left(\frac{\hbar \omega_\alpha}{2 T}\right) \cos(\omega_\alpha(t-t')) - i \sin(\omega_\alpha(t-t'))
	\right) 
	\\ &= 
	\eta \int_{-\infty}^{+\infty} \hbar \Omega \cth\left(\frac{\hbar \Omega}{2T}\right) e^{-i\Omega(t-t')} \frac{\d \Omega}{2\pi}
	\\ &= 
	\int_{-\infty}^{+\infty} \langle f(t) f(t')\rangle _\Omega e^{-i\Omega(t-t')} \frac{\d \Omega}{2\pi}.
\end{align*}
Для Фурье-компоненты
\begin{equation*}
	\langle f(t) f(t')\rangle_\Omega = \hbar \hbar \Omega \cth\left(\frac{\hbar \Omega}{2T}\right) = \left\{\begin{aligned}
	    &2 \eta T, &\hbar=0\\
	    &\eta \hbar |\Omega|, &T=0.
	\end{aligned}\right.
\end{equation*}
Во временном представлении
\begin{equation*}
	\langle f(t) f(t')\rangle = \left\{\begin{aligned}
	    2 \eta T \delta(t-t'), &\hbar=0,
	    -\eta \hbar  (t-t')^{-2} /\pi, &T=0.
	\end{aligned}\right.
\end{equation*}
Это согласуется с ФДТ в виде
\begin{equation*}
	\langle f(t) f(t')\rangle_\Omega = - \Im\left(\frac{1}{\alpha(\omega)}\right) \hbar \cth\left(\frac{\hbar \omega}{2T}\right) = \frac{\hbar \alpha''(\omega)}{|\alpha(\omega)|^2} \cth \frac{\hbar \omega}{2T},
\end{equation*}
где $\alpha(\omega) f(\omega) = q(\omega)$, и тогда
\begin{equation*}
	\alpha(\omega) = \left(-M \omega^2 + M \omega_0^2 - i \eta \omega\right)^{-1},
\end{equation*}
где $U'(q) = \frac{m \omega_0^2}{2} q^2$, и значит
\begin{equation*}
	\Im(\alpha(\omega))^{-1} = - \eta \omega,
	\hspace{10 mm} 
	\langle f(t) f(t')\rangle_\Omega = \hbar \Omega \hbar \cth\left(\frac{\hbar \Omega}{2T}\right).
\end{equation*}