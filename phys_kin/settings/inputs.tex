11 12 13 14 15 16 17 18 19

% document's head

\begin{center}
    \LARGE \textsc{Желательные знания по курсу <<Физическая кинетика>>}
\end{center}

\hrule

\phantom{42}

\begin{flushright}
    \begin{tabular}{rr}
    % written by:
        % \textbf{Источник}: 
        % & \href{__ссылка__}{__название__} \\
        % & \\
        % \textbf{Лектор}: 
        % & _ФИО_ \\
        % & \\
        \textbf{Автор заметок}: 
        & Хоружий Кирилл \\
        & \\
    % date:
        \textbf{От}: &
        \textit{\today}\\
    \end{tabular}
\end{flushright}

\thispagestyle{empty}
\tableofcontents
\newpage




\section*{Проводимость Друде}


\textbf{Общефизическое рассмотрение}. 
Рассмотрим движение электронов под действием электрического поля
\begin{equation*}
	m (\ddot{x} + \gamma \dot{x}) = e E,
\end{equation*}
в установившемся режиме $\ddot{x} = 0$, $\gamma = 1/\tau$, где $\tau$ -- время столкновений. Так находим
\begin{equation*}
	v = \frac{e \tau}{m}E,
	\hspace{5 mm} 
	j = e n v = \frac{n e^2}{m} \tau E,
	\hspace{0.5cm} \Rightarrow \hspace{0.5cm}
	\sigD = \frac{n e^2}{m} \tau.
\end{equation*}

\textbf{$\tau$-приближение}. Воспользуемся $\tau$-приближением для $\sub{I}{st}$
\begin{equation*}
	\left(
		\frac{\partial }{\partial t} + v_i \frac{\partial }{\partial r_i} + e E_i \frac{\partial }{\partial p_i} 
	\right) f(r, p, t) = - \frac{f(r, p, t) - \feq (p)}{\tau}.
\end{equation*}
Рассматривая однородную стационарную задачу приходим к уравнению, вида
\begin{equation*}
	e E_i \frac{\partial }{\partial p_i} f(p) = - \frac{\delta f(p)}{\tau},
\end{equation*}
где $\delta f = f - \sub{f}{eq}$, а хотим найти $j = e \int \frac{\d p}{(2 \pi \hbar)^d} \delta f(p) v(p)$.

Рассматривая задачу в предположение о линейном отклике, находим
\begin{equation*}
	f(p) = \feq (p) + \chi_i (p) E_i + \ldots,
	\hspace{0.25cm} \Rightarrow \hspace{0.25cm}
	\chi_i (p) = - e \tau \frac{\partial }{\partial p_i} \feq (p),
\end{equation*}
и подставляя это в выражение для $j$, находим
\begin{equation*}
	j_i = - e^2 \tau E_s \int \frac{\d p}{(2 \pi \hbar)^d} \frac{p_i}{m} \frac{\partial }{\partial p_s} \feq (p) = \frac{e^2 \tau E_i}{m} \int \frac{\d p}{(2 \pi \hbar)^d} \feq (p),
\end{equation*}
где мы проинтегрировали по частям. Таким образом приходим к выражению для проводимости Друде
\begin{equation*}
	 j_i = \frac{\sub{n}{eq} e^2 \tau}{m} E_i = \sigD E_i,
	 \hspace{10 mm} 
	 \sigD = \frac{\sub{n}{eq} e^2}{m} \tau.
\end{equation*}

\textbf{Переменное поле}. Пусть теперь $E_i = E_i (t)$, тогда
\begin{equation*}
	\left(
		\frac{\partial }{\partial t} + e E_i (t) \frac{\partial }{\partial p_i} 
	\right) f(p, t) = - \frac{f(p, t) - \feq (p)}{\tau}.
\end{equation*}
Переходя к линейному отклику, находим
\begin{equation*}
		\frac{\partial }{\partial t} \delta f(p, t) + e E_i (t) \frac{\partial }{\partial p_i} 
	 \feq (p, t) = - \frac{\delta f(p, t)}{\tau},
\end{equation*}
или переходя к Фурье $\delta f(t) = \int \frac{\d \omega}{2\pi} e^{-i \omega t} \delta f(\omega)$, находим
\begin{equation*}
	 - i \omega \, \delta f(p, \omega) + e E_i (\omega) \frac{\partial }{\partial p_i}  \feq (p) = - \frac{\delta f(p, \omega)}{\tau},
\end{equation*}
тогда Фурье-образ поправки функции распределения будет равен
\begin{equation*}
	\delta f(p, \omega) = -  \frac{e E_i (\omega) }{1-i \omega \tau}  \frac{\partial \feq(p)}{\partial p_i} \tau .
\end{equation*}
Подставляя в выражение для тока $j$, получим
\begin{equation*}
	j_i (\omega) = \frac{\sigD}{1 - i \omega \tau} E_i (\omega) = \sigma(\omega) E_i (\omega),
\end{equation*}
c полюсом в нижней полуплоскости -- причинная функция Грина! Собственно, после обратного Фурье, находим
\begin{equation*}
	j_i = \int_{-\infty}^{t} \sigma(t-t') E_i (t) \d t',
	\hspace{0.5cm} \Rightarrow \hspace{0.5cm}
	\sigma(t) = \int \frac{\d \omega}{2\pi} e^{- i \omega t} \frac{\sigD}{1-i \omega \tau}  =  \sigD \cdot \frac{1}{\tau} e^{-t/\tau} \theta(t).
\end{equation*}


% T1
\section{Кинетическое уравнение Власова}


Начнём с уравнение Лиувилля, считая заданными $\vc{r}^N = (\vc{r}_1,\,  \ldots,\, \vc{r}_N)$ и $\vc{p}^N = (\vc{p}_1,\,  \ldots,\, \vc{p}_N)$
\begin{equation*}
	\dot{\vc{r}}_i = \frac{\partial H}{\partial \vc{p}_i},
	\hspace{10 mm} 
	\dot{\vc{p}}_i = -\frac{\partial H}{\partial \vc{r}_i},
\end{equation*}
где Гамильтониан запишется в виде
\begin{equation*}
	H = K(\vc{p}^N) + V(\vc{r}^N) + \Phi (\vc{r}^N),
	\hspace{10 mm} 
	K(\vc{p}^N) = \sum_{i=1}^{N} \frac{\vc{p}_i^2}{2m},
	\hspace{5 mm} 
	\Phi (\vc{r}^N) = \sum_{i=1}^{N} \varphi (\vc{r}_i).
\end{equation*}
Введём также функцию распределения $f^{[N]}(\vc{r}^N, \vc{p}^N, t)$ так чтобы $f^{[N]}(\vc{r}^N, \vc{p}^N, t) \d \vc{r}^N \d \vc{p}^N$ -- вероятность находиться в данной точке фазового пространства. Нормировка единичная. 

\textbf{Закон сохранения}.
Закон сохранения в дифференциальном виде запишется в виде
\begin{equation*}
	\frac{\partial \rho}{\partial t} + \div \vc{j} = 0,
\end{equation*}
где в нашем случае $\rho$ -- $f^{[N]}$, и $\vc{j} = \{f^{[N]} \dot{\vc{r}}_i, f^{[N]} \dot{\vc{p}}_i\}$, тогда
\begin{equation*}
	\frac{\partial f^{[N]}}{\partial t} + \sum_{i=1}^{N}\left( \frac{\partial }{\partial \vc{r}_i} \left[f^{[N]} \dot{\vc{r}}_i\right] + \frac{\partial }{\partial \vc{p}_i} \left[
				f^{[N]} \dot{\vc{p}}_i
			\right]\right) = \frac{d f^{[N]}}{d t} = 0, 
\end{equation*}
при подстановке уранений Гамильтона. 



\textbf{Редуцированная функция}. Редуцированная функция $f^{(n)}$ определяется как
\begin{equation*}
	f^{(n)} (\vc{r}^n,\, \vc{p}^n,\, t) = \frac{N!}{(N-n)!} \int f^{[N]} (\vc{r}^N,\, \vc{p}^N,\, t) \d \vc{r}^{(N-n)} \d \vc{p}^{(N-n)},
\end{equation*}
где $d \vc{r}^{(N-n)} = d \vc{r}_{n+1} \ldots \d \vc{r}_N$ и $d \vc{p}^{(N-n)} = d \vc{p}_{n+1} \ldots \d \vc{p}_N$. 

Работаем  приближение потенциального внешнего поля
\begin{equation*}
	\dot{\vc{p}}_i = \vc{X}_i + \sum_{j=1}^{N} \vc{F}_{ij} (\vc{r}_i,\, \vc{r}_j),
	\hspace{10 mm} 
	\vc{F}_{ii} = 0.
\end{equation*}
Тогда сохранение перепишется в виде
\begin{equation*}
	\frac{\partial f^{[N]}}{\partial t} + \sum_{i=1}^{N} \frac{\vc{p}_i}{m} \frac{\partial f^{[N]}}{\partial \vc{r}_i} + \sum_{i=1}^{N} \vc{X}_i \frac{\partial f^{[N]}}{\partial \vc{p}_i} = -
	\sum_{i=1}^{N} \sum_{j=1}^{N} \vc{F}_{ij} \frac{\partial f^{[N]}}{\partial \vc{p}_i}.
\end{equation*}
При редуцирование в силу ограниченности в фазовом пространстве, остаётся
\begin{equation*}
	\frac{\partial f^{(n)}}{\partial t} + \sum_{i=1}^{n} \frac{\vc{p}_i}{m} \frac{\partial f^{(n)}}{\partial \vc{r}_i} + \sum_{i=1}^{n} \vc{X}_i \frac{\partial f^{(n)}}{\partial \vc{p}_i} = - \sum_{i=1}^{n} \sum_{j=1}^{n} \vc{F}_{ij} \frac{\partial f^{(n)}}{\partial \dot{\vc{p}}_i} - \frac{N!}{(N-n)!} \sum_{i=1}^{n} \sum_{j=n+1}^{N} \int \vc{F}_{ij} \frac{\partial f^{[N]}}{\partial \vc{p}_i} \d \vc{r}^{(N-n)} \d \vc{p}^{(N-n)}.
\end{equation*}
С учетом симметричности функции распределения, последнее слагаемое можем переписать в виде
\begin{equation*}
	- \frac{N! (N-n)}{(N-n)!} \sum_{i=1}^{n} \int F_{i, n+1} \frac{\partial f^{[N]}}{\partial \vc{p}_i} \d \vc{r}^{(N-n-1)} \d \vc{p}^{(N-n-1)} \d \vc{r}_{n+1} \d \vc{p}_{n+1},
\end{equation*}
Так приходим к выражению, вида
\begin{equation*}
	\left(
		\frac{\partial }{\partial t} + \sum_{i=1}^{n} \frac{\vc{p}_i}{m} \frac{\partial }{\partial \vc{r}_i} + \sum_{i=1}^{n} \left[
			\vc{X}_i + \sum_{j=1}^{n} F_{ij}
		\right] \frac{\partial }{\partial \vc{p}_i} 
	\right) f^{(n)} = - \sum_{i=1}^{n} \int F_{i, n+1} \frac{\partial f^{(n+1)}}{\partial \vc{p}_i} \d \vc{r}_{n+1} \d \vc{p}_{n+1}.
\end{equation*}
Эта система уравнений называется цепочкой уравнений Боголюбова-Борна-Грина 
Обычно интерес представляют $n = 1, 2$, кстати $\int f^{(n)} \d \vc{r}^{n} \d \vc{p}^n = \frac{N!}{(N-n)!}$.

\subsection*{Одночастичный случай}

Для $n=1$ уравнение сведётся к
\begin{equation*}
	\left(
		\frac{\partial }{\partial t} + \frac{\vc{p}_1}{m} \frac{\partial }{\partial \vc{r}_1} + \vc{X}_1 \frac{\partial }{\partial \vc{p}_1} 
	\right) f^{(1)} (\vc{r}_1,\, \vc{p}_1,\, t) = - \int \vc{F}_{12} 
	\frac{\partial }{\partial \vc{p}_1} f^{(2)}(\vc{r}_1, \vc{p}_1, \vc{r}_2, \vc{p}_2, t) \d \vc{r}_2 \d \vc{p}_2.
\end{equation*}
В силу отсутствия корелляций между столкновениями попробуем сделать приближение
\begin{equation*}
 	f^{(2)}(\xi_1, \xi_2, t) = f^{(1)} (\xi_1^t)f^{(1)}(\xi_2^t).
\end{equation*}
Определяя
\begin{equation*}
	\tilde{\vc{F}} (\vc{r}, t) = \int \vc{F}_{12} (\vc{r}_1,\, \vc{r}_2) f^{(1)} (\vc{r}_2,\, \vc{p}_2,\, t) \d \vc{r}_2 \d \vc{p}_2,
\end{equation*}
приходим к бесстолкновительному уравнению Власова 
\begin{equation}
	\left(
		\frac{\partial }{\partial t} + \frac{\vc{p}_1}{m} \frac{\partial }{\partial \vc{r}_1}  + \left[
			\vc{X}_1 + \tilde{\vc{F}}
		\right] \frac{\partial }{\partial \vc{p}_1} 
	\right) f^{(1)} = 0.
\end{equation}
которое валидно при $n d^3 \gg 1$. 




\subsection*{Двухчастичный случай}

Для $n=2$:
\begin{equation*}
	\left(
		\frac{\partial }{\partial t} + \frac{\vc{p}_1}{m} \frac{\partial }{\partial \vc{r}_1} + \frac{\vc{p}_2}{m} \frac{\partial }{\partial \vc{r}_2} + \left[
			\vc{X}_1 + \vc{F}_{12}
		\right] \frac{\partial }{\partial \vc{p}_1} + 
		\left[
			\vc{X}_2 + \vc{F}_{21}
		\right] \frac{\partial }{\partial \vc{p}_2} 
	\right) f^{(2)} (\xi_1, \xi_2, t) =  - \int \left(
		\vc{F}_{13} \frac{\partial }{\partial \vc{p}_1} + \vc{F}_{23} \frac{\partial }{\partial \vc{p}_2} 
	\right) f^{(3)} \d \vc{r}_3 \d \vc{p}_3
\end{equation*}
Считая $n d^3 \ll 1$, можем игнорировать\footnote{
	Также будем считать, что $\vc{X}_i$ меняются слабо. 
}  трёхчастичные столкновения, тогда
\begin{equation*}
	\left(
		\frac{\vc{p}_1}{m} \frac{\partial }{\partial \vc{r}_1} + \frac{\vc{p}_2}{m} \frac{\partial }{\partial \vc{r}_2} + F_{12} \left[\frac{\partial }{\partial \vc{p}_1} - \frac{\partial }{\partial \vc{p}_2} \right]
	\right) f^{(2)} = 0.
\end{equation*}
Переходя к координатам, находим
\begin{equation*}
	\vc{F}_{12} \left(\frac{\partial }{\partial \vc{p}_1} - \frac{\partial }{\partial \vc{p}_2}  \right) f^{(2)} = - \left(
		\frac{\vc{p}_1}{m} \frac{\partial }{\partial \vc{r}_1} + \frac{\vc{p}_2}{m} \frac{\partial }{\partial \vc{r}_2} 
	\right) f^{(2)}.
\end{equation*}
Введём $\vc{r} = \vc{r}_1 - \vc{r}_2$, $\vc{R} = \frac{1}{2}(\vc{r}_1 + \vc{r}_2)$, тогда
\begin{equation*}
	\frac{\partial f^{(2)}}{\partial \vc{R}} \ll \frac{\partial f^{(2)}}{\partial \vc{r}}.
\end{equation*}

Возвращаемся к одночастичной функции, интегрируя находим
\begin{equation*}
	\left(
		\frac{\partial }{\partial t} + \frac{\vc{p}_1}{m} \frac{\partial }{\partial \vc{r}_1} + \vc{X}_1 \frac{\partial }{\partial \vc{p}_1} 
	\right) f^{(1)} (\vc{r}_1,\, \vc{p}_1,\, t) =  - \int \vc{F}_{12} \left(
		\frac{\partial }{\partial \vc{p}_1} - \frac{\partial }{\partial \vc{p}_2} 
	\right) f^{(2)} \d \xi_2 = \int 
		\left[
			\frac{\vc{p}_2}{m}-\frac{\vc{p}_2}{m}
		\right] \frac{\partial f^{(2)}}{\partial \vc{r}} \d \vc{r} \d \vc{p}_2,
\end{equation*}
продолжая с правой частью, вводя $\sub{\vc{v}}{отн} = \frac{\vc{p}_2}{m} - \frac{\vc{p}_1}{m}$ находим
\begin{equation*}
	\int \d p_2 \d^2 \sigma \d z  \sub{\vc{v}}{отн} \left(
		f^{(2)}(t_+) - f^{(2)}(t_-)
	\right).
\end{equation*}
После столкновения меняются импульсы частиц, тогда правую часть можем переписать в виде
\begin{equation}
	\int \d \vc{p}_2 \d^2 \sigma \sub{\vc{v}}{отн} \left(
		f^{(1)}(\vc{p}_2', \vc{r}, t) f^{(1)} (\vc{p}_1', \vc{r}, t) - f^{(1)} (\vc{p}_2, \vc{r}, t) f^{(1)}(\vc{p}_1, \vc{r}, t)
	\right), \text{\ \ -- интеграл столкновений}.
\end{equation}
Формально есть частицы прилетевшие и улетевшие.  К слову, $\d \vc{p}_1 \d \vc{p}_2 = \d \vc{p}_1' \d \vc{p}_2'$.



% T2
\section{x Колебания в плазме}
\input{parts/T2}

% T3
\section{Кондактанс}
\subsection{Общая идея}

Рассмотрим два куска металла между которыми существует 1D идеальный провод. Химпотенциалы соответственно равны
\begin{equation*}
	\mu_L = \mu + \frac{1}{2} eV, 
	\hspace{5 mm} 
	\mu_R = \mu - \frac{1}{2} eV,
\end{equation*}
ток можем найти как $I = I_R-I_L$:
\begin{equation*}
	I = \sum_{k > 0} e v_k \left(
		f_R(\varepsilon_k) - f_L (\varepsilon_k)
	\right),
\end{equation*}
где $f_{R, L} (\varepsilon_k) = f(\varepsilon_k - \mu_{R, L})$ -- числа заполнения. 
Подставляя в выражение для тока, находим
\begin{equation*}
	I = -\sum_{k > 0} e v_k \frac{\partial f}{\partial \varepsilon_k} \left(
		\delta \mu_L - \delta \mu_R
	\right) = - e^2 V \int_{k>0} \frac{d k}{2\pi} \frac{\partial \varepsilon_k}{\partial \hbar k}  \frac{\partial f}{\partial \varepsilon_k} = - \frac{e^2}{2\pi} V \int_{0}^{\infty} \d \varepsilon \frac{\partial f}{\partial \varepsilon} =  \frac{e^2}{2\pi \hbar} V.
\end{equation*}
где сделали подстановку $v_k = {\partial \varepsilon_k}/{\partial \hbar k}$, числа заполнения равны $f(\varepsilon=0)=1$ и $f(\varepsilon=\infty)= 0$ соответственно. Таким образом находим квант проводимости
\begin{equation*}
	G = \frac{I}{V} = \frac{e^2}{2 \pi \hbar}.
\end{equation*}
Если скажем, что электроны отражаются с коэффициентом $|t_i|^2$ и всего всего есть $N$ одномерных каналов, получим \textit{формулу Ландауэра}
\begin{equation*}
	G = \frac{e^2}{2\pi \hbar} \sum_{i=1}^N |t_i|^2.
\end{equation*}





\subsection{Подход Ландауэра}


Рассмотрим точечный контакт двух проводников. Пусть хим. потенциал резервуаров $\mu_1$ и $\mu_2$, функция распределения при температуре $\Theta$
\begin{equation*}
	f_{\alpha}(E) = \left(e^{(E-\mu_{\alpha})/\Theta}+1\right)^{-1},
	\hspace{5 mm} 
	\alpha = 1,2.
\end{equation*}
Будем считать систему двухмерной, ось $x$ вдоль течения тока, тогда уравнение Шрёдингера на стационарные волновые функции имеет вид
\begin{equation*}
	\hat{H} \psi = E \psi,
	\hspace{10 mm} 
	\hat{H} = - \frac{\hbar^2}{2m}(\partial_x^2 + \partial_y^2) + U(x, y).
\end{equation*}
Считаем, что расстояние между стенками меняется как $W(x)$, тогда для $W = \const $ можем явно найти $\psi(x, y) = \varphi(x) \varphi(y)$ и 
\begin{equation*}
	\varphi_n(y) = \sqrt{\frac{2}{W}} \sin\left(
		\pi n\left(\frac{y}{W} + \frac{1}{2}\right)
	\right),
\end{equation*}
для которых верно, что
\begin{equation*}
	-\left(\frac{\hbar^2}{2m} \partial_y^2 + E\right) \varphi(y) = - \tilde{E} \varphi(y),
	\hspace{10 mm} 
	\tilde{E} = E - U_n,
	\hspace{5 mm} 
	U_n = \frac{(\pi n \hbar)^2}{2 m W^2}.
\end{equation*}
Для адиабатического приближения можем подставить $W=W(x)$. Таким образом эффцективный потенциал имеет вид потенциального барьера высоты $E_n = (\pi n \hbar)^2/(2m W_0^2)$, где $W_0 = \min W(x)$. 


Введём вероятности отражения и прохождения $R_{nm}$ и $T_{nm}$ из канала $m$ в канал $n$, тогда
\begin{equation*}
	I = 2 \sum_{n,m} \int_{0}^{\infty} \frac{\d E}{2 \pi \hbar v_n} e v_n \left(
		f_1(E) (\delta_{nm}-R_{nm}) - f_2(E)T_{nm}
	\right).
\end{equation*}
Для линейного кондактанса $G = dI /dV$ при $V \to 0$, пределе нулевой температуры $\mu_1 = E_F$, $\mu_2 = E_F-eV$ и $f(E) = \theta(\mu-E)$ найдём
\begin{equation*}
	G = 2 \sum_{n.m} \int_{0}^{\infty} \frac{e^2 \d E}{2 \pi \hbar} \delta\left(
		E_F - eV - E
	\right) T_{nm} (E) = \frac{e^2}{\pi \hbar} \sum_{n,m} T_{nm}(E_F).
\end{equation*}
Появился $G_q = e^2 / (\pi \hbar)$ -- квантовый кондактанс. 



% \subsection{Метод вторичного квантования}

\subsection{Поток тепла}

Запишем в линейном приближение ток и поток тепла $I_Q$
\begin{equation*}
	I = \frac{2e}{h} \int_{0}^{\infty} (f_1(E)-f_2(E))T(E)\d E= G V + L \,\delta \Theta,
	\hspace{5 mm} 
	I_Q = \frac{2}{h} \int_{0}^{\infty} (f_1(E)-f_2(E))(E-\mu)T(E)\d E = L' V + K\, \delta \Theta,
	\hspace{10 mm} 
	\delta \Theta = \Theta_1 - \Theta_2,
	\hspace{5 mm} 
	\mu_1-\mu_2= eV,
\end{equation*}
где ввели кинетические коэффициенты $G,\, L,\, L'$ и $K$. 

Раскладываясь по температуре, находим
\begin{equation*}
	I_Q / \delta \Theta = K \approx \frac{G}{e^2} \int_{0}^{\infty}  \frac{\partial f(E)}{\partial \Theta} (E-\mu) \d E = \frac{\pi^2}{3} \frac{G \Theta}{e^2}.
\end{equation*}
По сути это закон Видемана-Франца в применении к точечному контакту. 

Аналогично находим $L' = L \Theta$, что является проявлением принципа симметрии кинетических коэффициентов Онсагера. И, наконец, для $L$:
\begin{equation*}
	L = G \frac{\pi^2}{3} \frac{\Theta}{e} \frac{\partial \ln T}{\partial E}.
\end{equation*}




% \subsection{Время релаксации}

% Из золотого правила Ферми
% \begin{equation*}
% 	I_{\vc{k}}[f] = \frac{2\pi}{\hbar}\sum_{\vc{k} \in \Omega} |\bk{\vc{k}'}[\mathcal U]{\vc{k}}^|^2 (f_{\vc{k}'} - f_{\vc{k}}) \delta(\varepsilon_{\vc{k}} - \varepsilon_{\vc{k}'}),
% \end{equation*}
% где ввели сумму потенциалов индивидуальных примесей
% \begin{equation*}
% 	\mathcal U(\vc{r}) = \sum_{j=1}^{\sub{N}{imp}} U(\vc{r}-\vc{R}_j).
% \end{equation*}
% Считая примеси нескоррелированными, можем получить
% \begin{equation*}
% 	\overline{|\bk{\vc{k}'}[\mathcal U]{\vc{k}}^|^2} = \frac{\sub{N}{imp}}{V^2}|\hat{U}(\vc{k}'-)
% \end{equation*}













% \begin{equation*}
% 	\hat{H} = -J \sum_n \left(\kb{n}{n+1}+\kb{n+1}{n}\right) + \sum_n \varepsilon_n \kb{n}{n},
% 	\hspace{5 mm} 
% 	\delta_n 
% \end{equation*}

% \begin{equation*}
% 	\varepsilon_n = \tfrac{1}{2}W \cos(\sigma n)
% \end{equation*}


% \begin{equation*}
% 	\varepsilon_n \in [0, W] \hspace{5 mm} 
% 	W=0.1 \hspace{5 mm} W=0.5
% \end{equation*}

% \begin{tabular}{cl}
%  $J$ & туннелирование \\
%  $U$ & не дружат в одном узле \\
%  $V$ & внешний потенциал \\
%  $\delta$ & шум
% \end{tabular}

% \begin{tabular}{ll}
%  $J$ & масщтаб энергии \\
%  $\tau=\hbar/J$ & масштаб времени 
% \end{tabular}

% T4
\section{Упругое рассеяние электронов на примесях}





Рассеяние электронов на примесях
\begin{equation*}
	\left(
		\frac{\partial }{\partial t} + \vc{v} \frac{\partial }{\partial \vc{r}} +  \dot{\vc{k}} \frac{\partial }{\partial \vc{k}} 
	\right) f = I_{\vc{k}} = - \frac{\delta f}{\tau}.
\end{equation*}
В данном случае для линеаризованного кинетического уравнения $\tau$-приближение является точным, где $\delta f = f_{k}-f_0$. 


Поработаем с самим интегралом столкновений
\begin{equation*}
	I_k = \frac{2\pi}{\hbar} \sum_{\vc{k}'} \left(
		|\bk{\vc{k}'}[\sub{U}{пол}]{\vc{k}}|^2 \delta (\varepsilon_{k}-\varepsilon_{k'}) \left[
			f_{k'}(1-f_{k}) - f_k (1-f_{k'})
		\right]
	\right),
\end{equation*}
где $f_{k'}(1-f_{k}) - f_k (1-f_{k'}) = f_{k'}-f_k$. Для матричного элемента
\begin{equation*}
	\sub{U}{пол} (\vc{r}) = \sum_{j=1}^{N} U(\vc{r}-\vc{R}_j),
	\hspace{10 mm} 
	\langle \vc{k}| = \frac{1}{\sqrt{V}} e^{i \vc{k} \vc{r}}.
\end{equation*}
Тогда для матричного эдлемента находим
\begin{equation*}
	|\bk{\vc{k}'}[\sub{U}{пол}]{\vc{k}}|^2 = \frac{1}{V^2} |\tilde{U}(\vc{k}-\vc{k}')|^2 \cdot \bigg|
		\sum_{j=1}^{N} e^{i(\vc{k}-\vc{k}')\vc{R}_j}
	\bigg|^2,
	\hspace{10 mm} 
	\tilde{U} (\vc{q}) = \int e^{i \vc{q} \vc{r}} U(\vc{r}) \d^3 \vc{r}.
\end{equation*}
Усредняя по случайному положению примесей
\begin{equation*}
	\left\langle \bigg| 
		\sum_{j=1}^{N} e^{i (\vc{k}-\vc{k}') R_j}
	\bigg| \right\rangle_{\text{прим}} = N + N (N-1) \delta_{\vc{k}, \vc{k}'}.
\end{equation*}
Для матричного элемента получили выражение
\begin{equation*}
	|\bk{\vc{k}'}[\sub{U}{пол}]{\vc{k}}|^2 = \frac{N}{V^2} |\tilde{U}(\vc{q})|^2 + \frac{N(N-1)}{V^2} |\tilde{U}(0)|^2 \delta_{\vc{k}, \vc{k}'},
	\hspace{10 mm} 
	\vc{q} = \vc{k}-\vc{k}'.
\end{equation*}
Итого для интеграла столкновений получаем выражение после замены $\sum_{\vc{k}} \to \int \frac{V \d^3 k}{(2 \pi)^3}$
\begin{equation*}
	I_k (f) = \frac{2\pi n}{\hbar} \int \frac{d^3 \vc{k}'}{(2\pi)^3} |\tilde{U} (\vc{q})|^2 \delta(\varepsilon_k - \varepsilon_{\vc{k}}) \cdot \left(
		\delta f_{k'} - \delta f_{k}
	\right),
\end{equation*}
где уже линеаризовали выражение. Здесь $n$ -- примесное.



Рассмотрим стационарный однородный случай, когда $\hbar \dot{\vc{k}} = - e \vc{E}$, где поле считаем малой поправкой, тогда
\begin{equation*}
	\dot{\vc{k}} \frac{\partial }{\partial \vc{k}}  = - e \vc{E} \cdot \vc{v} \frac{\partial }{\partial \varepsilon},
	\hspace{10 mm} 
	\delta f_k \overset{\mathrm{def}}{=}  \tau(\varepsilon) \left(e \vc{E} \cdot \vc{v}_{\vc{k}} \right) \frac{\partial f_0}{\partial \varepsilon},
\end{equation*}
то есть ищем решение в $\tau$-приближение. Получается уравнение
\begin{equation*}
	- (\vc{E} \cdot \vc{v}) \frac{\partial f_0}{\partial \varepsilon} = I_k = \frac{2\pi n}{\hbar} e \vc{E} \int \frac{d^3 \vc{k}'}{(2\pi)^3} |\tilde{U}(\vc{q})|^2 \delta(\varepsilon_k - \varepsilon_{k'}) \left(
		\tau (\varepsilon') \vc{v}' \frac{\partial f_0}{\partial \varepsilon} \bigg|_{\varepsilon'} 
		-
		\tau (\varepsilon) \vc{v} \frac{\partial f_0}{\partial \varepsilon} \bigg|_{\varepsilon}
	\right).
\end{equation*}
Сокращая $\partial_\varepsilon f_0$ и всё лишнее, находим
\begin{equation*}
	\vc{v} = \frac{\hbar \vc{k}}{m} = \frac{n \tau(\varepsilon[\vc{k}])}{4 \pi^2 \hbar} \int_{0}^{\infty} dk' \ (k')^2 \int d \Omega_{k'} |\tilde{U}(\vc{q})|^2 \cdot \frac{\delta(k-k')}{\hbar^2 k/m} \frac{\hbar}{m} (\vc{k}-\vc{k}').
\end{equation*}
Остаётся выражение
\begin{equation}
	\frac{1}{\tau(\varepsilon)} = \frac{m k n}{4 \pi^2 \hbar^3} \int d \Omega_k \ |\tilde{U}(\vc{q})|^2 (1 - \hat{\vc{k}} \cdot \hat{\vc{k}}'),
	\hspace{10 mm} 
	\hat{\vc{k}} = \frac{\vc{k}}{k}.
\end{equation}



\textbf{Дифференциальное сечение рассеяния}. Найдём выражение
\begin{equation*}
	\frac{\d \sigma}{\d \Omega} = \left(\frac{m}{2\pi \hbar^2}\right)^2 |\tilde{U}(\vc{q})|^2,
	\hspace{10 mm} 
	\vc{q} = \vc{k}-\vc{k}',
	\hspace{5 mm} 
	q^2 = 4 k^2 \sin^2 \left(\frac{\theta}{2}\right).
\end{equation*}
И интеграл столкновений перепишется в виде
\begin{equation}
	\frac{1}{\tau(\varepsilon)} = n v \int \frac{d \sigma}{d \Omega} (1-\cos \theta) \d \Omega = n v \sub{\sigma}{tr},
\end{equation}
где возникло новое $\sub{\sigma}{tr}$ с подавленным рассеянием на малых углах. 


Вспоминая формулу Друде, находим
\begin{equation*}
	\vc{j} = \sigma_D \vc{E},
	\hspace{10 mm} 
	\sigma_D = \frac{e^2 n_0 \sub{\tau}{tr}}{m},
\end{equation*}
где входит именно $\tau_{\text{tr}}$.





\textbf{Фурье-образ}. Для экранированного кулоновского потенциала 
\begin{equation*}
	U(r) = - e^{-r/\lambda} \frac{Z e^2}{r},
	\hspace{10 mm} 
	\tilde{U}(\vc{q}) = \int U(\vc{r}) e^{-i \vc{q} \vc{r}} \d V = \frac{4 \pi Z e^2}{q^2 + \lambda^{-2}}.
\end{equation*}
Для дифференциального сечения рассеяния находим
\begin{equation*}
	\frac{d \sigma}{d \Omega} = \frac{m^2}{4 \pi^2 \hbar^2} \left(
		\frac{4\pi Z e^2}{q^2+\lambda^{-2}}
	\right)^2 = \left(
		\frac{Z e^2}{4 E_F} \frac{1}{\sin^2 \frac{\theta}{2} + (2 k_F \lambda)^{-2}}
	\right)^2.
\end{equation*}
где $q = 2 k_F \sin \frac{\theta}{2}$.
Полное сечение рассеяния тогда получается
\begin{equation*}
	\sigma = \int \d \sigma = \int_{0}^{\pi} \left(
		\frac{Z e^2}{4 E_F} \frac{1}{\sin^2 \frac{\theta}{2} + (2 k_F \lambda)^{-2}}
	\right)^2 = \frac{2\pi Z^2 e^4}{4 E_F^2} \int_{0}^{2}  \frac{\d u}{(u + \frac{1}{2}(k_F \lambda)^{-2})^2},
\end{equation*}
где $u = 1 - \cos \theta$.  Итого, введя $\zeta \overset{\mathrm{def}}{=}  \frac{4}{\pi} (k_F \lambda)^2$, находим
\begin{equation*}
	\sigma = \frac{\pi Z^2 e^4}{2 E_F^2} \frac{(\pi \zeta)^2/2}{1 + \pi \zeta}.
\end{equation*}
Для транспортного $\sub{\sigma}{tr}$, находим
\begin{equation*}
	\sub{\sigma}{tr} = \frac{2\pi Z^2 e^4}{4 E_F^2} \int_{0}^{2}  \frac{u \d u}{(u + \frac{1}{2}(k_F \lambda)^{-2})^2} = \frac{\pi Z^2 e^4}{2 E_F^2} \left(
		\ln(1+\pi \zeta) - \frac{\pi \zeta}{1+\pi \zeta}
	\right).
\end{equation*}
Для проводимости $\rho$ можем найти
\begin{equation*}
	\rho = \frac{m}{n e^2 \sub{\tau}{tr}} = \frac{m n v_F}{n_0 e^2} \frac{\pi Z^2 e^4}{2 E_F^2} \zeta^3 F(\zeta),
	\hspace{10 mm} 
	F(\zeta) = \frac{1}{\zeta^3} \left(
		\ln(1+\pi \zeta) - \frac{\pi \zeta}{1+\pi \zeta}
	\right).
\end{equation*}
Итого, находим
\begin{equation}
	\rho = Z^2 R_q \sub{a}{B} \frac{n}{n_0} F(\zeta) \cdot \left[
		\frac{e^2}{2 \pi \hbar} \frac{m e^2}{\hbar^2} \frac{p_F}{e^2} \frac{\pi e^4}{p_F^2/2m^2} \frac{64 k_F^6 \lambda^6}{\pi^3}
	\right] = Z^2 R_q \sub{a}{B} \frac{n}{n_0} F(\zeta).
\end{equation}
где подставили $\lambda^2 = \frac{\pi \sub{a}{B}}{4 k_F}$.


% T5
\section{Рассеяние электронов на фононах}



\textbf{Эффект Иоффе-Регеля}. На высоких температурах $r^2 \sim T$ для ионов, тогда
\begin{equation*}
	\tau = \frac{1}{\sub{n}{ion} v \sigma} \sim \frac{1}{T},
	\hspace{10 mm} 
	\rho = \frac{m}{n e^2 \tau} \sim T.
\end{equation*}
Для $\tau v_F \sim \lambda_F$, можем записать с учётом $n \sim k_F^3$
\begin{equation*}
	\rho = \frac{m v_F}{n e^2 \tau v_F} = \frac{m v_F}{n e^2 \lambda_F} \sim \frac{\hbar}{k_F e^2},
\end{equation*}
что называется пределом Иоффе-Регеля, которые неплохо работает для легированных полупроводников. 


\textbf{Испускание фононов}. И снова запишем столкновительный интеграл в терминах приход-уход:
\begin{equation*}
	I_p = \sum_{\vc{p}'} w_{\vc{p} \vc{p}'} n_{\vc{p}'} (1-n_{\vc{p}}) - \sum_{\vc{p}'}  w_{\vc{p}' \vc{p}} n_{\vc{p}} (1-n_{\vc{p}'}).
\end{equation*}
Рассматриваем однородную ситуацию, тогда 
\begin{equation*}
	\dot{\vc{p}} \frac{\partial n}{\partial \vc{p}} = 
	- e (\vc{E} \cdot \vc{v}) \frac{\partial n_0}{\partial \varepsilon}
	= \sub{I}{ст}.
\end{equation*}
Учитвая что $w_q \sim q$, можем расписать
\begin{align*}
	\sub{I}{ст} = &\frac{2\pi}{\hbar V} \sum_{\vc{q}} \left(
		w_{\vc{q}} (1 + N_{\vc{q}}) n_{\vc{p} + \hbar \vc{q}} (1-n_{\vc{p}}) \delta (\varepsilon_p - \varepsilon_{\vc{p} + \hbar \vc{q}} + \hbar \omega_q) 
		+ w_q N_q n_{\vc{p}-\hbar \vc{q}} (1-n_{\vc{p}}) \delta(\varepsilon_p - \varepsilon_{\vc{p}-\hbar \vc{q}}- \hbar \omega_q) 
	\right) - \\
	- & \frac{2\pi}{\hbar V} \sum_{\vc{q}}
	\left(
		w_{\vc{q}} (1+N_{\vc{q}}) n_q (1-n_{\vc{p}-\hbar \vc{q}}) \delta (\varepsilon_p - \varepsilon_{\vc{p} - \hbar \vc{q}} - \hbar \omega_q) + w_q N_q (1-n_{\vc{p}+\hbar \vc{q}}) \delta(\varepsilon_p - \varepsilon_{\vc{p}+\hbar \vc{q}} + \hbar \omega_q)
	\right).
\end{align*}
% вставить формулу с фото
Будем считать, что фононы равновесные
\begin{equation*}
	N_q = N_{q}^0 = \frac{1}{e^{\hbar \omega_q/T}-1},
	\hspace{0.5cm} \Rightarrow \hspace{0.5cm}
	\frac{1+N_q}{N_q} = e^{\hbar \omega_q/T}.
\end{equation*}
Для электронов
\begin{equation*}
	n_p^0 = \frac{1}{e^{(\varepsilon_p-\mu)/T}+1},
	\hspace{0.5cm} \Rightarrow \hspace{0.5cm}
	\frac{1-n_p^0}{n_p^0} = e^{(\varepsilon_p-\mu)/T}.
\end{equation*}
Преобразуем выражение из квадратных скобок *
\begin{equation}
	(1+N_q)(1-n_p)(1-n_{p+\hbar q}) \left(
		\frac{n_{\vc{p}+\hbar \vc{q}}}{1-n_{\vc{p}+\hbar \vc{q}}}
		- \frac{N_q}{1+N_q} \frac{n_p}{1-n_p}
	\right),
	\label{eqsq}
\end{equation}
которое очевидно зануляется для равновесных функций. 


Решение будем искать в виде
\begin{equation*}
	n_p = n_{p}^0 + \delta n_p = n_p^0 - \frac{\partial n^0_p}{\partial \varepsilon_p} \Phi_p = n_p^0 + \frac{n^0(\varepsilon_p)(1-n^0(\varepsilon_p))}{T} \Phi_p.
\end{equation*}
Возвращаясь к \eqref{eqsq}, получаем линеаризуя
\begin{align*}
	&(1+N_q)(1-n_p^0) (1-n_{\vc{p} + \hbar \vc{q}}^0) \left[
		\frac{\delta n_{\vc{p} + \hbar \vc{q}}}{(1-n_{\vc{p} + \hbar \vc{q}}^0)^2} - \frac{N_q}{1+N_q} \frac{\delta n_p}{(1-n_p^0)^2}
	\right] = \\
	= & +\frac{1}{T} (1+N_q)(1-n_p^0) n_{\vc{p} + \hbar \vc{q}}^0 \left[
		\Phi_{\vc{p} + \hbar \vc{q}}-\Phi_p
	\right] = \\
	= & - \frac{1}{T} (1+N_q) N_q (n_{\vc{p} + \hbar \vc{q}}^0-n_p^0) \left[
		\Phi_{\vc{p} + \hbar \vc{q}}-\Phi_p
	\right]
	.
\end{align*}
Аналогично преобразуется второе слагаемое в *, откуда находим линеаризованный интеграл столкновений:
\begin{align*}
	\sub{I}{ст}(\Phi_p) &= - \frac{2\pi}{\hbar V} \sum_{\vc{q}} w_q \frac{(1+N_q)N_q (n_{\vc{p}+\hbar \vc{q}}^0-n_p^0)}{T} \left[
		\Phi_{\vc{p}+\hbar \vc{q}} - \Phi_{\vc{p}}
	\right] \times \left[
		\delta(\varepsilon_p - \varepsilon_{\vc{p}+\hbar \vc{q}} + \hbar \omega_q) - \delta(\varepsilon_p - \varepsilon_{\vc{p}+\hbar \vc{q}} - \hbar \omega_q)
	\right]
\end{align*}
Выделим физ. смысл в слагаемых
\begin{align*}
		\sub{I}{ст}(\Phi_p) = - \frac{2 \pi}{\hbar V} \sum_{\vc{q}} w_q \frac{(1+N_q) N_q}{T} \bigg(
		&\left[
			n^0(\varepsilon_p + \hbar \omega_q) - n^0(\varepsilon_p)
		\right] 
		\delta(\varepsilon_p - \varepsilon_{\vc{p}+\hbar \vc{q}} + \hbar \omega_q) 
		- \\ - &
		\left[
			n^0(\varepsilon_p - \hbar \omega_q) - n^0(\varepsilon_p)
		\right] \delta(\varepsilon_p - \varepsilon_{\vc{p}+\hbar \vc{q}} - \hbar \omega_q)
	\bigg) \left[
		\Phi_{\vc{p}+\hbar \vc{q}} - \Phi_{\vc{p}}
	\right]
	.
\end{align*}
Учтём, что мы живём вблизи поверхности Ферми, тогда $\hbar \omega_q$ мало по сравнению с $\varepsilon_p$, приходим к выражению
\begin{equation*}
	\sub{I}{ст}(\Phi_p) = - \frac{\partial n^0}{\partial \varepsilon} \frac{2\pi}{\hbar V} \sum_q w_q \frac{2 \hbar \omega_q (1+N_q) N_q}{T} \delta(\varepsilon_{\vc{p}+\hbar \vc{q}}- \varepsilon_{\vc{p}}) \left[
		\Phi_{\vc{p}+\hbar \vc{q}} - \Phi_{\vc{p}}
	\right].
\end{equation*}
Аргумент $\delta$-функции можем расписать в виде
\begin{equation*}
	\varepsilon_p - \varepsilon_{\vc{p}+\hbar \vc{q}} \pm \hbar \omega_q = 
	\frac{2 p \hbar q \cos \theta}{2 m} + \frac{\hbar^2 q^2}{2m} \pm \hbar c_L q = \frac{\hbar p q}{m}\left(
		\cos \theta + \frac{\hbar q}{2p} \pm \frac{m c_L}{p}
	\right),
\end{equation*}
где $c_L \ll v_F$, поэтому можем опустить последнее слагаемое. 



\textbf{Кинетическое уравнение}. Итого, будем решать кинетическое уравнение на $\Phi_{\vc{p}}$ вида
\begin{equation}
	- e (\vc{E} \cdot \vc{v}) \frac{\partial n^0}{\partial \varepsilon} = \sub{I}{ст}(\Phi_p) = - \frac{\partial n_0}{\partial \varepsilon} \frac{2\pi}{\hbar V} \sum_q w_q \frac{2 \hbar \omega_q (1+N_q) N_q}{T} \delta(\varepsilon_{\vc{p}+\hbar \vc{q}}- \varepsilon_{\vc{p}}) \left[
		\Phi_{\vc{p}+\hbar \vc{q}} - \Phi_{\vc{p}}
	\right].
\end{equation}
Решение аналогично будем искать в виде $\Phi_p = - e (\vc{E} \cdot \vc{v}) \sub{\tau}{tr} (\varepsilon_p)$, что соответствует $\tau$-приближению: $\sub{I}{ст} = - \delta n_p / \tau$. Таким образом остаётся найти $\sub{\tau}{tr}$, и найти остальные величины по формуле Друде.  Выражая из двух уравнений $(\vc{E} \cdot \vc{v})$, находим
% в данном случае метод моментов сводится к одному слагаемому
\begin{equation*}
	(\vc{E} \cdot \vc{v}) = - \frac{4 \pi}{\hbar V} \sum_{\vc{q}} w_q \frac{ \hbar \omega_q (1+N_q) N_q}{T} \delta(\varepsilon_{\vc{p}+\hbar \vc{q}}- \varepsilon_{\vc{p}}) 
	\frac{\hbar (\vc{q} \cdot \vc{E})}{m} \sub{\tau}{tr}(\varepsilon_p).
\end{equation*}
Переходя к интегрированию, нахоим
\begin{equation*}
	(\vc{E} \cdot \vc{v}) = - \frac{4 \pi}{\hbar} \int \frac{q^2 \d q \d \Omega_q}{(2\pi)^3} w_q \frac{\hbar \omega_q (1+N_q) N_q}{T} \delta(\varepsilon_{\vc{p} + \hbar \vc{q}} - \varepsilon_p) \frac{\hbar (\vc{q} \cdot \vc{E})}{m} \sub{\tau}{tr}(\varepsilon_p).
\end{equation*}
Проведём интегрирование, введя полярную ось и расписав
\begin{align*}
	\vc{q}  &= (q \sin \theta \cos \varphi,\, q \sin \theta \sin \varphi,\, q \cos \theta), \\
	\vc{E} &= (E \sin \theta_E \cos \varphi_E,\, E \sin \theta_E \sin \varphi_E,\, E \cos \theta_E).
\end{align*}
Тогда скалярное произведение перепишется в виде
\begin{equation*}
	(\vc{q} \cdot \vc{E}) = q E \left(
		\cos \theta \cos \theta_E + \sin \theta \sin \theta_E \cos(\varphi-\varphi_E)
	\right),
\end{equation*}
где после интегрирование второе слагаемое зануляется. Также подставляя $(\vc{E} \cdot \vc{v}) = E v \cos \theta_E$, тогда
\begin{equation*}
	\frac{p}{m \sub{\tau}{tr}(\varepsilon_p)} = - \frac{4 \pi}{T} \int_0^{q_D} \frac{q^2 \d q \sin \theta \d \theta}{(2\pi)^2} w_q \omega_q (1 + N_q) N_q \times \delta\left(
		\tfrac{\hbar q p}{m}\left(\cos \theta + \tfrac{\hbar q}{2p}\right)
	\right) \times \frac{\hbar q}{m} \cos \theta,
\end{equation*}
где $q_D$ -- максимальный дебаевский импульс. Таким образом
\begin{equation*}
	\frac{1}{\sub{\tau}{tr}(\varepsilon_p)} = \frac{4 \pi m}{T p^2} \int_{0}^{q_D} \frac{q^2 \d q}{(2\pi)^2} w_q \omega_q (1+N_q)N_q \int_{-1}^{1} dx\ x \times \delta\left(x + \tfrac{\hbar q}{2p}\right),
\end{equation*}
где ввели $x = \cos \theta$.

Вообще $q_D = \sqrt[3]{6 \pi^2 n}$, $p_F = \sqrt[3]{3 \pi^2 n}$, тогда $\frac{\hbar q_D}{2 p_F} < 1$. Учитывая, что $w_q \propto \omega_q \propto q$, находим
\begin{equation*}
	\frac{1}{\sub{\tau}{tr}(\varepsilon_p)} \propto \frac{1}{T} \int_{0}^{q_D} q^5 \d q \ \frac{e^{\hbar \omega_q/T}}{(e^{\hbar \omega_q/T}-1)^2}. 
\end{equation*}
Введём $z = \frac{\hbar \omega_q}{T} = \frac{T_D}{T} \frac{q}{q_D}$, где $T_D = \hbar c_L q_D$. Таким образом 
\begin{equation*}
	\frac{1}{\sub{\tau}{tr}(\varepsilon_p)} \propto \frac{1}{T} \left(
		\frac{T}{T_D}
	\right)^6 \int_{0}^{T_D/T} \frac{e^z z^5 \d z}{(e^z-1)^2},
\end{equation*}
где из-за разности скоростей возникла пятая степень вместо четвертой. Итого, искомое выражение 
\begin{equation}
	\frac{1}{\sub{\tau}{tr}(\varepsilon_p)} \propto \left(\frac{T}{T_D}\right)^5 \int_{0}^{T_D/T} \frac{z^5 \d z}{\sh^2 \frac{z}{2}}.
\end{equation}


\textbf{Формула Друде}. Вспоминая, что
\begin{equation*}
	\sigma = \sigma_D = \frac{e^2 n \sub{\tau}{tr}}{m},
\end{equation*}
находим 
\begin{equation*}
	\frac{\sub{\rho}{e-ph}(T)}{\sub{\rho}{e-ph}(T_D)} = \frac{\sub{\sigma}{e-ph}(T)}{\sub{\sigma}{e-ph}(T_D)} = \left(\frac{T}{T_D}\right)^5 \int_{0}^{T_D/T} \frac{z^5 \d z}{\sh^2 \frac{z}{2}} \bigg/ \int_{0}^{1} \frac{z^5 \d z}{\sh^2 \frac{z}{2}}.
\end{equation*}
Для $T \ll T_D$ получится
\begin{equation*}
	\frac{\sub{\rho}{e-ph}(T)}{\sub{\rho}{e-ph}(T_D)} = 526 \left(\frac{T}{T_D}\right)^5.
\end{equation*}
И в обратную сторону, для $T \gg T_D$, раскладываясь в ряд, находим
\begin{equation*}
	\frac{\sub{\rho}{e-ph}(T)}{\sub{\rho}{e-ph}(T_D)} = 1.06 \left(\frac{T}{T_D}\right).
\end{equation*}



% T6
\section{Электроны в магнитном поле}
\input{parts/T6}

% T7
\section{Модель диффузии Лоренца}


\textbf{Несохранение числа частиц}. 
В $\tau$-приближении:
\begin{equation*}
	\frac{\partial f}{\partial t}  + \vc{v} \cdot \frac{\partial f}{\partial \vc{r}} = - \frac{f-f_0}{\tau},
	\hspace{10 mm} 
	\delta n = \int \delta f \d^3 \vc{r},
	\hspace{5 mm} 
	F(\vc{v}, t) \overset{\mathrm{def}}{=} \int \d^3 \vc{r}\ f(\vc{r}, \vc{v}, t).
\end{equation*}
Проинтегрируем уравнение Больцмана по координатам:
\begin{equation*}
	\frac{\partial F}{\partial t} = - \frac{F-F_0}{\tau},
\end{equation*}
Введя $\delta F (\vc{v}, t) = F(\vc{v}, t) - F_0 (\vc{v})$, найдём
\begin{equation*}
	\delta F(\vc{v}, t) = \delta F(\vc{v}, 0) e^{- t/\tau},
\end{equation*}
таким образом $\tau$-приближение не сохраняет число частиц, релаксируя к равновесному. 


\textbf{Модификация}. Исправим эту проблему следующим образом
\begin{equation*}
	\frac{\partial f}{\partial t} + \vc{v} \frac{\partial f}{\partial \vc{r}}  = \frac{1}{\tau} \left[
		- f + \int \frac{\d \Omega_v}{4\pi} f
	\right] = \frac{1}{\tau} \left(Pf - f\right),
	\hspace{10 mm} 
	P f = \int \frac{\d \Omega_v}{4\pi} f(\vc{r}, \vc{v}, t).
\end{equation*}
что называется моделью Лоренца, случай легкой примеси в тяжелом газе, а именно слабо-ионизированный газ. Здесь $Pf$ -- члены прихода.  Электроны рассеиваются\footnote{
	См. ЛЛX.
}  на тяжелых частицах.
Забавный факт -- тут возникает диффузия, а ещё эта модель имеет точное решение. 


\textbf{Проверка}. Аналогично перейдём к функции $F$, тогда
\begin{equation*}
	\frac{\partial F}{\partial t} = \frac{1}{t}\left(
		P F(v, t) - F(\vc{v}, t)
	\right),
\end{equation*}
тогда, после применения проекции $P$, находим
\begin{equation*}
	\frac{\partial (PF)}{\partial t}  = \frac{1}{\tau}\left[P^2 F - P F\right] = 0,
	\hspace{0.5cm} \Rightarrow \hspace{0.5cm}
	PF(v, t) = \Phi(v).
\end{equation*}
Так находим, что
\begin{equation*}
	F(\vc{v}, t) = \Phi(v) + \left[
		F_0 (\vc{v}) - \Phi(v)
	\right] e^{- t/\tau}.
\end{equation*}



\textbf{Лаплас}. Рассмотрим уравнение
\begin{equation*}
	\frac{\partial f}{\partial t}  + \vc{\nabla} \cdot \frac{\partial f}{\partial \vc{r}} = - \frac{1}{\tau}\left(
		f - \langle f\rangle
	\right).
\end{equation*}
Сдлаем преобразование Фурье в пространстве и преобразование Лапласа по времени:
\begin{equation*}
	\hat{f} (\vc{k}, \vc{v}, s) = \int_{0}^{\infty} e^{-st}\d t \int d^3 r\ e^{- i \vc{k} \vc{r}} f(\vc{r}, \vc{v}, t).
\end{equation*}
\textbf{вставить из фото}.


Приходим к интегралу
\begin{equation*}
	\frac{1}{2} \int_{-1}^{1} dx\ \frac{(1+s \tau)  - i v k \tau x}{(i + s \tau)^2 + (v k \tau x)^2} = \frac{1}{v k \tau} \arctg \frac{v k \tau}{1+s \tau}.
\end{equation*}
Подставляем всё в $P \hat{f}$
\begin{equation*}
	P \hat{f}(\vc{k}, \vc{v}, s) = \left[
		1 - \frac{1}{vk\tau} \arctg \frac{v k \tau}{1 + s \tau}
	\right] \int \frac{\d \Omega_v}{4\pi} \frac{f(\vc{k}, \vc{v}, t=0)}{s + i \vc{k} \cdot \vc{v} + \tau^{-1}},
\end{equation*}
находим
\begin{equation*}
	\hat{f}(\vc{k}, \vc{v}, s) = \frac{\tau^{-1}}{s + i \vc{v} \cdot \vc{k} + \tau^{-1}} \left[
		1 - \frac{1}{vk\tau} \arctg \frac{v k \tau}{1 + s \tau}
	\right] \int \frac{\d \Omega_v}{4\pi} \frac{f(\vc{k}, \vc{v}, t=0)}{s + i \vc{k} \cdot \vc{v} + \tau^{-1}} +  \frac{f(\vc{k}, \vc{v}, t=0)}{s + i \vc{k} \cdot \vc{v} + \tau^{-1}}.
\end{equation*}


Конкретизируем начальные условия:
\begin{equation*}
	f(\vc{r}, \vc{v}, t=0) = \delta(\vc{r}) \delta(\vc{v}-\vc{v}_0),
	\hspace{0.5cm} \Rightarrow \hspace{0.5cm}
	f(\vc{k}, \vc{}, t=0) = \delta(\vc{v}-\vc{v}_0).
\end{equation*}
Подставляя в интеграл по телесному углу, находим
\begin{equation*}
	\int \frac{\d \Omega_v}{4\pi}  \frac{f(\vc{k}, \vc{v}, t=0)}{s + i \vc{k} \cdot \vc{v} + \tau^{-1}} 
	= \int \frac{\d \Omega_v}{4\pi} \frac{\delta(\vc{v} - \vc{v}_0)}{s + i \vc{k} \cdot \vc{v} + \tau^{-1}}
	= \frac{1}{s + i \vc{k} \cdot \vc{v} + \tau^{-1}} \frac{\delta(v-v_0)}{4 \pi v_0^2}.
\end{equation*}




\textbf{Диффузия}. Рассматриваем время $t \gg \tau$, тогда малые $s \tau \ll 1$, и можем разложиться
\begin{equation*}
	1 - \frac{1}{vk\tau} \arctg \frac{v k \tau}{1 + s \tau} = 1 - \frac{1}{1 + s \tau} + \frac{1}{3} \frac{(v k \tau)^2}{(1+s \tau)^3} \approx s \tau + \frac{1}{3} v^2 k^2 \tau^2 + \ldots 
\end{equation*}
Подставляя в выражение для $\hat{f}$, находим
\begin{equation*}
	\hat{f}(\vc{k}, \vc{v}, s) =  \left(
		\frac{\tau^{-1}}{s + i \vc{v} \cdot \vc{k} + \tau^{-1}} 
	\right)^2 \frac{1}{s + \frac{1}{3} v^2 k^2 \tau^2}  \frac{\delta(v-v_0)}{4 \pi v_0^2} + \frac{\delta(\vc{v}-\vc{v}_0)}{s + i (\vc{k} \cdot \vc{v}_0) + \tau^{-1}}.
\end{equation*}
Смотрим большие времена и большие расстояния, тогда самое большое это $\tau^{-1}$, и можем переписать функцию распределения $\hat{f}$ в виде
\begin{equation*}
	\hat{f}(\vc{k}, \vc{v}, s) \approx \frac{1}{s + D k^2} \frac{\delta(v-v_0)}{4 \pi v_0^2},
	\hspace{10 mm} 
	D = \frac{1}{3} v_0^2 \tau.
\end{equation*}
Возвращаясь к обратному Фурье-образу, находим
\begin{equation*}
	f(\vc{r}, \vc{v}, t) = \int_{s^* - i \infty}^{s^* + i \infty} \frac{e^{st \d s}}{2 \pi i} \int \frac{d^3 k}{(2\pi)^3} e^{i \vc{k} \vc{r}} \hat{f}(\vc{k}, \vc{v}, s).
\end{equation*}
Считая по вычетам, находим
\begin{equation*}
	f(\vc{r}, \vc{v}, t) = \frac{\delta(v-v_0)}{4 \pi v_0^2} \left[
		\int_{-\infty}^{+\infty} \frac{\d k_x}{2\pi}  \exp\left(
			- Dt( k_x - \frac{i x}{2 D t})^2 -\frac{x^2}{4 D t}
		\right)
	\right] \left[ \int_{-\infty}^{+\infty}  \frac{\d k_y}{2\pi} \ldots \right] \left[ \int_{-\infty}^{+\infty}  \frac{\d k_z}{2\pi} \ldots \right],
\end{equation*}
так приходим к явной диффузии
\begin{equation}
	f(\vc{r}, \vc{v}, t) =  \frac{1}{(4 \pi D t)^{3/2}}  \frac{\delta(v-v_0)}{4 \pi v_0^2} e^{-r^2/4Dt},
	\hspace{10 mm} 
	D = \frac{1}{3} v_0^2 \tau.
\end{equation}





% T8
\section{Электронный газ}
\subsection*{Т8. Электронный газ}


И снова смотрим на уравнение Больцмана, ищём решение в виде $f = f_0 + \delta f$, смотрим на $\tau$-приближение, равновесным будет распределение Ферми:
\begin{equation*}
	f_0 = \frac{1}{e^{\frac{\varepsilon-\mu}{T}} + 1},
	\hspace{10 mm} 
	\mu = \mu (t, \vc{r}),
	\hspace{5 mm} 
	T = T(t, \vc{r}).
\end{equation*}
Будем решать уравнение рассматривая стационарный случай
\begin{equation*}
	\vc{v} \cdot \frac{\partial f_0}{\partial \vc{r}} - e \vc{E} \cdot \vc{v} \frac{\partial f_0}{\partial \varepsilon}  = - \frac{\delta f}{\tau}.
\end{equation*}
Можем переписать 
\begin{equation*}
	\frac{\partial f_0}{\partial \vc{r}} = \frac{\partial f_0}{\partial T}  \vc{\nabla} T + \frac{\partial f_0}{\partial \mu} \vc{\nabla} \mu = - \frac{\varepsilon-\mu}{T} \frac{\partial f}{\partial \varepsilon} \vc{\nabla} T - \frac{\partial f_0}{\partial \varepsilon} \nabla \mu.
\end{equation*}
Тогда, после подстановки, левая часть уравнения может быть найдена в виде
\begin{equation*}
	\delta f = \tau \left(
		\frac{\varepsilon-\mu}{T} (\vc{v} \cdot \vc{\nabla} T) + \vc{v} \cdot (\vc{\nabla} \mu + e \vc{E})
	\right) \frac{\partial f_0}{\partial \varepsilon} .
\end{equation*}


\textbf{Металл}. Достаточно рассмотреть $- \frac{\partial f_0}{\partial \varepsilon} \approx \delta(\varepsilon - \varepsilon_F)$. Для тока $\vc{j}$ находим
\begin{equation*}
	\vc{j} = - e \int \vc{v} (f_0 + \delta f) \frac{2 \d^3 p}{(2\pi \hbar)^3} = \frac{e}{3} (\vc{\nabla} \mu + e \vc{E}) \int \tau v^2 \left(
		- \frac{\partial f_0}{\partial \varepsilon} 
	\right) \frac{2 \d^3 p}{(2\pi \hbar)^3} + \frac{e}{3} \frac{\vc{\nabla} T}{T} \int \tau v^2 (\varepsilon-\mu) \left(-\frac{\partial f_0}{\partial \varepsilon} \right) \frac{2 \d^3 p}{(2\pi \hbar)^3}.
\end{equation*}
Для плотности потока энергии $\vc{q}$ 
\begin{equation*}
	\vc{q} = \int \vc{v} (\varepsilon- e \varphi) (f_0 + \delta f) \frac{2 \d^3 p}{(2\pi \hbar)^3} = - \frac{\vc{j}}{e} (\mu - e \varphi) - \ldots.
\end{equation*}
% вставить из фото!
Введём диссипативную часть $\vc{q}'$
\begin{equation*}
	\vc{q}' = \vc{q} + \frac{\vc{j}}{e} (\mu- e \varphi).
\end{equation*}
Также определим усреднение в виде
\begin{equation*}
	\langle F(\varepsilon)\rangle = \frac{m}{3n} \int \frac{2 \d^3 p}{(2 \pi \hbar)^3} v^2  \left(- \frac{\partial f_0}{\partial \varepsilon} \right)  F(\varepsilon) = \frac{2}{3n} \int_{0}^{\infty} \varepsilon \left(- \frac{\partial f_0}{\partial \varepsilon} \right) F(\varepsilon) g(\varepsilon) \d \varepsilon,
	\hspace{10 mm} 
	n = \int_{0}^{\infty} \varepsilon \left(- \frac{\partial f_0}{\partial \varepsilon} \right) g(\varepsilon) \d \varepsilon.
\end{equation*}
Тогда уравнение перепишется в виде
\begin{equation*}
	\vc{E} + \frac{\vc{\nabla} \mu}{e} = \frac{m \vc{j}}{n e^2 \langle \tau\rangle} - \frac{\vc{\nabla} T}{ e T} \frac{\langle (\varepsilon -\mu) \tau\rangle}{\langle \tau\rangle} = \frac{\vc{j}}{\sigma} + \alpha \vc{\nabla} T.
\end{equation*}
Тогда для потока энергии
\begin{equation*}
	\vc{q}' = - \frac{\langle (\varepsilon-\mu) \tau\rangle}{e \langle  \tau\rangle} \vc{j} + \frac{\vc{\nabla} T}{m T} \frac{n \langle (\varepsilon-\mu) \tau\rangle^2}{\langle \tau\rangle} - 
	\frac{\vc{\nabla} T}{m T} n \langle (\varepsilon-\mu)^2 \tau\rangle 
	= 
	\alpha T \vc{j} - \varkappa \vc{\nabla} T
	.
\end{equation*}
Где коэффициенты соответственно равны
\begin{equation}
	\alpha = - \frac{\langle (\varepsilon-\mu) \tau \rangle}{e T \langle \tau\rangle},
	\hspace{10 mm} 
	\varkappa = \frac{n \langle \tau\rangle}{m T}\left[
		\frac{\langle (\varepsilon-\mu)^2 \tau\rangle}{\langle \tau\rangle} - \frac{\langle (\varepsilon-\mu) \tau\rangle^2}{\langle \tau\rangle^2}
	\right],
	\hspace{10 mm} 
	\sigma = \frac{n e^2 \langle \tau\rangle}{m}.
\end{equation}
где $\varkappa$ -- коэффициент теплопроводности, $\alpha$ -- термоэлектрический коэффициентр, $\sigma$ -- проводимость. 


% Соотношения Онзагера можем получить, записав с фото
% \begin{equation*}
% 	\vc{j} = 
% \end{equation*}




\textbf{Полупроводник}. Здесь можем написать, что $\frac{\partial f_0}{\partial \varepsilon} = - \frac{\partial f_0}{\partial T}$, так как $f_0 \approx e^{(\mu-\varepsilon)/T}$. Тогда усредение можем переписать в виде
\begin{equation*}
	\langle F(\varepsilon)\rangle  = \frac{m}{3 n T} \frac{2 \d^3 p}{(2 \pi \hbar)^3} f_0 v^2 F(\varepsilon).
\end{equation*}
Считая, что $\tau(\varepsilon) \propto v^k \propto \varepsilon^{k/2}$ и что $f_0 \propto e^{-\frac{m v^2}{2T}}$, находим
\begin{equation*}
	\langle v^k\rangle \propto \left(\frac{2 T}{m}\right)^{k/2} \Gamma\left(\frac{3+k}{2}\right).
\end{equation*}
Так, например, для $\alpha$ получится
\begin{equation*}
	\alpha = \frac{1}{e} \left(
		\frac{\mu}{T} - \frac{\langle \tau v^2 \varepsilon\rangle}{\langle  \tau v^2\rangle}
	\right) = \frac{1}{e} \left(
		\frac{\mu}{T} - \frac{\Gamma\left(\frac{1+k}{2}\right)}{\Gamma\left(\frac{3+k}{2}\right)}
	\right) = \frac{1}{e} \left(
		\frac{\mu}{T} - \frac{5+k}{2}
	\right).
\end{equation*}
% êîýôôèöèåíòû Ïåëüòüå

% Для полупроводника получается коэффициент плетье гораздо больше.
% ДЗ: найти коэффициент плетье в металле, должно получиться \mu/E_F 


% число Лоренца -- винеман -франц закон вывести
% 

% T9
\section{Уравнения Навье-Стокса}
% со вторым заданием всё разбирать не будем

Пишем уравнение Больцмана для двухчастичных столкновений
\begin{equation*}
	\frac{\partial f}{\partial t}  + \vc{v} \frac{\partial f}{\partial \vc{r}}  + \vc{F} \cdot \frac{\partial f}{\partial \vc{p}} = \int d^3 p_1\,   \sub{v}{отн} \d \sigma_{p p_1} (f' f_1' - f f_1).
\end{equation*} 
Столкновения упругие: $\vc{p} + \vc{p}_1 = \vc{p}' + \vc{p}_1'$.
Умножим уравнение на некоторую $\varphi(\vc{p})$ и проинтегрируем по импульсам:
\begin{equation*}
	\int d^3 \vc{p}\ \varphi(\vc{p}) \ldots = \frac{1}{4} \iint d^3\vc{p}\, d^3\vc{p}_1\, \sub{v}{отн} \d \sigma_{p p_1} (f' f_1' -f f_1) \times \left(
		\varphi(\vc{p}) + \varphi(\vc{p}_1) - \varphi(\vc{p}') - \varphi(\vc{p}_1')
	\right).
\end{equation*}

\textbf{Частицы}. 
Можем вспомнить законы сохранения и подставить $\varphi(\vc{p}) = \left[1,\, p_\alpha,\, \frac{p^2}{2m}\right]$, получим следующее выражение для $\varphi(p)=1$
\begin{equation}
	\frac{1}{n} \int d^3p\ \vc{v} f = \langle \vc{v}\rangle = \vc{u}(t, \vc{r}),
	\hspace{5 mm} 
	\int d^3 \vc{p}\ f = n(t, \vc{r}),
	\hspace{0.5cm} \Rightarrow \hspace{0.5cm}
	\frac{\partial n}{\partial t} + \div(n \vc{u}) = 0.
\end{equation}

\textbf{Импульс}. 
Теперь рассмотрим $\varphi(\vc{p}) = m v_\alpha$:
\begin{equation*}
	\frac{\partial }{\partial t} (m n u_\alpha) + \frac{\partial \Pi_{\alpha \beta}}{\partial x_\beta} = n F_\alpha,
\end{equation*}
где ввели тензор потока импульса
\begin{align*}
	\Pi_{\alpha \beta} &\overset{\mathrm{def}}{=} \int d^3\vc{p}\ m v_\alpha v_\beta f = n m \langle v_\alpha v_\beta \rangle 
	= m n u_\alpha u_\beta + mn \langle (v_\alpha - u_\alpha)(v_\beta-u_\beta)\rangle 
	= \\ &=
	m n u_\alpha u_\beta + mn \left\langle 
		(v_\alpha - u_\alpha)(v_\beta-u_\beta) - \tfrac{1}{3} \delta_{\alpha \beta} (\vc{v}-\vc{u})^2
	\right\rangle + \tfrac{1}{3} \delta_{\alpha \beta} mn  \langle (\vc{v}-\vc{u})^2\rangle
	= \\ &= 
	m n u_\alpha u_\beta  + P \delta_{\alpha \beta} - \sigma_{\alpha \beta}',
\end{align*}
где ввели для удобства величины тензора вязких напряжений и давления
\begin{equation*}
	\sigma_{\alpha \beta}' =  - mn \left\langle 
		(v_\alpha - u_\alpha) (v_\beta-u_\beta) - \tfrac{1}{3} \delta_{\alpha \beta} (\vc{v} - \vc{u})^2
	\right\rangle,
	\hspace{10 mm} 
	P  = \tfrac{1}{3} mn \left\langle (\vc{v}-\vc{u})^2\right\rangle.
\end{equation*}
Тогда уравнение перепишется в виде
\begin{equation*}
	\frac{\partial (mn u_\alpha)}{\partial t} + \frac{\partial P}{\partial x_\alpha} + \frac{\partial (mn u_\alpha u_\beta)}{\partial x_\beta}  = \frac{\partial \sigma_{\alpha\beta}'}{\partial x_\beta} + n F_\alpha.
\end{equation*}
В силу уравнения непрерывности часть слагаемых сократится, тогда 
\begin{equation*}
	m n \left(
		\frac{\partial u_\alpha}{\partial t}  + u_\beta \frac{\partial u_\alpha}{\partial x_\beta} 
	\right) + \frac{\partial p}{\partial x_\alpha}  = \frac{\partial \sigma_{\alpha \beta}'}{\partial x_\beta} + n F_\alpha.
\end{equation*}
Введем плотность $\rho \overset{\mathrm{def}}{=}  m n$, тогда
\begin{equation}
	\frac{d }{d t}  = \frac{\partial }{\partial t}  + (\vc{u} \cdot \vc{\nabla}),
	\hspace{10 mm} 
	\rho \frac{d u_\alpha}{d t} = \rho \left(
		\frac{\partial \vc{u}}{\partial t} + (\vc{u} \cdot \vc{\nabla}) \vc{u}
	\right)_\alpha = - \frac{\partial P}{\partial x_\alpha} + \frac{\partial \sigma_{\alpha \beta}'}{\partial x_\beta}  + n F_\alpha.
\end{equation}

\textbf{Энергия}. Подставим $\varphi(\vc{p}) = \frac{p^2}{2m}$, тогда
\begin{equation}
	\frac{\partial \varepsilon}{\partial t} + \div \vc{q} = n (\vc{F} \cdot \vc{u}),
	\hspace{10 mm} 
	\varepsilon = n \left\langle \tfrac{m \vc{v}^2}{2}\right\rangle,
	\hspace{5 mm} 
	\vc{q} = n \left\langle \vc{v} \tfrac{m \vc{v}^2}{2}\right\rangle.
\end{equation}
Вообще малостью будем считать $\vc{v} - \vc{u}$, тогда
\begin{align}
	\varepsilon &= \frac{mn}{2} \left(
		\left\langle \left(\vc{v}-\vc{u}\right)^2\right\rangle + \vc{u}^2
	\right) = \frac{3}{2} P + \frac{1}{2} mn u^2, \\
	q_\alpha  &=  u_\alpha \left( \frac{5}{2} P + \frac{1}{2} n m u^2\right) + q_\alpha' - \sigma_{\alpha \beta}' u_\beta,
\end{align}
где диссипативная часть плотности потока энергии $\vc{q}'$ имеет вид
\begin{equation*}
	q_\alpha' = \frac{mn}{2} \langle (v-u)_\alpha (\vc{v}-\vc{u})^2\rangle.
\end{equation*}
Подставляя и сокращая, находим
\begin{equation*}
	\frac{3}{2} \frac{\partial P}{\partial t}  + \frac{5}{2} P \div \vc{u} + \frac{3}{2} u_\alpha \frac{\partial P}{\partial x_\alpha} = \sigma_{\alpha \beta}' \frac{\partial u_\beta}{\partial x_\alpha}  - \div \vc{q}'.
\end{equation*}
Объединяя в $\frac{d }{d t} $, можем переписать в виде
\begin{equation}
	\frac{3}{2} \frac{\partial P}{\partial t}  + \frac{5}{2} P \div \vc{u} = \sigma_{\alpha \beta}' \frac{\partial u_\beta}{\partial x_\alpha} - \div \vc{q}'.
\end{equation}

\textbf{Температура}. Введём для одноатомного газа
\begin{equation*}
	\frac{3}{2} T = \left\langle  \frac{m (\vc{v}-\vc{u})^2}{2}\right\rangle,
	\hspace{0.5cm} \Rightarrow \hspace{0.5cm}
	P(t, \vc{r}) = n(t, \vc{r}) T(t, \vc{r}).
\end{equation*}
Снова учитывая уравнение непрерывности можем перписать выражение в виде
\begin{equation}
	\frac{3n}{2} \frac{d T}{d t} + n T \div \vc{u} = \sigma_{\alpha \beta}' \frac{\partial u_\alpha}{\partial x_\alpha}  - \div \vc{q}',
\end{equation}
что является уравнением на температуру. Три основные уравнения -- (9.1), (9.3), (9.7), которые мы получили из уравнения Больцмана. Осталось замкнуть эти уравнения. 

\textbf{$\tau$-приближение}. Будем решать систему уравнение в $\tau$-приближение, когда $\sub{I}{ст} = - \frac{f-f_0}{\tau}$. Выбираем функцию $f_0$ в виде локально равновестного распределения
\begin{equation*}
	f_0 = \frac{n(t, \vc{r})}{\left(2 \pi m T(t, \vc{r})\right)^{3/2}} \exp\left(
		- \frac{(\vc{p}-m \vc{u}(t, \vc{r}))^2}{2 m T(t, \vc{r})}
	\right),
	\hspace{10 mm} 
	\int f_0 \d^3 \vc{p} = n (t, \vc{r}).
\end{equation*}
Мы учли, что длина пробега $l \ll L$, характерных размеров системы. Число $\text{Kn} = \frac{l}{L}$ -- число Кнудсена, которое и характеризует то что достаточно часто происходят столкновения. 

Подставляя в левую часть $f_0$, находим поправку
\begin{equation*}
	\frac{\partial f_0}{\partial t}  + \vc{v} \frac{\partial f_0}{\partial \vc{r}} + \vc{F} \frac{\partial f_0}{\partial \vc{p}} = - \frac{\delta f}{\tau}.
\end{equation*}
Верно, что $\sigma_{\alpha\beta}'$ будет зануляться для локально равновесного распределения. После выражения временных производных из бездиссипативных уравнений, получается
\begin{equation*}
	\delta f = - \tau \frac{f_0}{T} \left(
		(\vc{v} \cdot \vc{\nabla} T) \left(
			\frac{m (v')^2}{2T} - \frac{5}{2}
		\right) + \frac{m}{2} v_\alpha' v_\beta' U_{\alpha \beta}
	\right),
\end{equation*}
где ввели переменные
\begin{equation*}
	\vc{v}' = \vc{v} - \vc{u},
	\hspace{10 mm} 
	U_{\alpha\beta} = \frac{\partial u_\alpha}{\partial x_\beta} + \frac{\partial u_\beta}{\partial x_\alpha} - \frac{2}{3} \delta_{\alpha \beta} \div \vc{u}.
\end{equation*}

\subsection*{Кинетические коэффициенты}

\textbf{Теплопроводность}. Главное здесь найти
\begin{equation*}
	q_\alpha' = \int d^3 \vc{p}\, 
	\delta f \, v_\alpha' \frac{m (v')^2}{2}.
\end{equation*}
Подставляя поправку $\delta f$, находим
\begin{equation*}
	q_\alpha' = - \nabla_\alpha T \int d^3 \vc{p}\ \frac{f_0 \tau m}{6 T}\left(
		\frac{m (v')^6}{2T} - \frac{5 (v')^4}{2}
	\right) = - \kappa n_\alpha T,
\end{equation*}
где \textit{коэффициент теплопроводности} $\kappa$ равен
\begin{equation*}
	\kappa = \frac{nm}{6T} \left\langle 
		\tau(v') \left(
			\frac{m (v')^6}{2T} - \frac{5 (v')^4}{2} 
		\right)
	\right\rangle.
\end{equation*}
Нам понадобятся интегралы, вида
\begin{equation*}
	\langle (v')^{2n} \rangle = \frac{(T/m)^n}{(2\pi)^{3/2}} \int d^3 \vc{x}\  x^{2n} e^{-x^2/2}
	=  (2n+1)!! \left(\frac{T}{m}\right)^n.
\end{equation*}
Собирая всё вместе, находим
\begin{equation}
	\kappa = \frac{5 n T}{2m}\tau,
\end{equation}
где мы считали $\tau = \const$.


\textbf{Тензор вязких напряжений}. Для тензора вязких напряжений
\begin{equation*}
	\sigma_{\alpha \beta}' =  - m \int d^3 \vc{p}\ \delta f \left(
		v_\alpha' v_\beta' - \frac{\delta_{\alpha\beta}}{3} (v')^2
	\right) = \frac{n m^2}{2T} U_{\mu \nu} \left\langle 
		\tau(v') v_\mu' v_\nu' \left(
			v_\alpha' v_\beta' - \frac{\delta_{\alpha \beta}}{3} (v')^2
		\right)
	\right\rangle.
\end{equation*}
Вспоминаем теорию поля
\begin{equation*}
	\langle v_\alpha' v_\beta' v_\mu' v_\nu' \rangle = \frac{1}{15} \left(
		\delta_{\alpha \beta} \delta_{\mu \nu} + \delta_{\alpha \mu} \delta_{\alpha \beta} + \delta_{\alpha \nu} \delta_{\beta \mu}
	\right).
\end{equation*}
Получается тензор, вида
\begin{equation*}
	\sigma_{\alpha \beta}' = \frac{n m^2}{2T} \cdot \frac{\langle \tau (v')^4\rangle_0}{15} 2 U_{\alpha \beta} = \eta \left(
		\frac{\partial u_\alpha}{\partial x_\beta} + \frac{\partial u_\beta}{\partial x_\alpha} - \frac{2}{3} \delta_{\alpha \beta} \div \vc{u}
	\right),
\end{equation*}
где \textit{коэффицент вязкости} $\eta$ равен
\begin{equation}
	\eta = \frac{n m^2}{T} \cdot \frac{\langle \tau (v')^4\rangle_0}{15}= n \tau T.
\end{equation}
Заметим, что
\begin{equation*}
	\eta = \frac{m \kappa}{c_p}, 
	\hspace{10 mm} 
	c_p = \frac{5}{2}.
\end{equation*}


\textbf{Уравнения Навье-Стокса}. Для $\sigma_{\alpha \beta}'$ обычно можем переписать её в виде
\begin{equation*}
	\sigma_{\alpha \beta}' = \eta \left(
		\frac{\partial u_\alpha}{\partial x_\beta} + \frac{\partial u_\beta}{\partial x_\alpha} - \frac{2}{3} \delta_{\alpha \beta} \div \vc{u}
	\right) + \zeta \delta_{\alpha \beta} \div \vc{u},
\end{equation*}
где в нашем случае для малой плотности $\zeta = 0$, в отличие от сдвиговой вязкости $\eta$. Считая $\zeta, \eta = \const$
\begin{equation*}
	\frac{\partial \sigma_{\alpha \beta}'}{\partial x_\beta}  = \eta\left(
		\nabla^2 u_\alpha + \nabla_\alpha \div \vc{u} - \frac{2}{3} \nabla_\alpha \div \vc{u}
	\right) + \zeta \nabla_\alpha \div \vc{u},
\end{equation*}
и подставляя в исходное уравнение, нахходим уравнение Навье-Стокса
\begin{equation}
	\rho \frac{d \vc{u}}{d t}  = - \vc{\nabla} P + \eta \nabla^2 \vc{u} + \left(
		\zeta + \frac{\eta}{3}
	\right) \grad \div \vc{u} + n \vc{F}.
\end{equation}


\textbf{Беспорядок}. Для наличия вмороженного беспорядка получим дополнительное слагаемое
\begin{equation*}
	\rho \frac{d \vc{u}}{d t} =\ldots - \rho \frac{\vc{u}}{\tau}.
\end{equation*}
Для получения формулы Друде, можем увидеть что почти всё в стационарном случае занулится, и получится
\begin{equation*}
	\vc{u} = \frac{n \tau}{\rho } \vc{F},
	\hspace{10 mm}
	\vc{j} = - n e \vc{u} = - n e \frac{n \tau}{\rho} (-e \vc{E}) = \frac{e^2 n \tau}{m} \vc{E}.
\end{equation*}







%%%%%%%%%% ВТОРОЕ ЗАДАНИЕ %%%%%%%%%%%%%%%%%%%%%%%

% Т12
\setcounter{section}{9}
\section{Холловская проводимость}
Рассмотрим систему
\begin{align*}
	\dot{x}_\alpha + \varepsilon_{\alpha \beta \gamma} \dot{k}_\beta \Omega_n^\gamma = v_n^\alpha \\ 
	\dot{k}_\alpha + \frac{e}{\hbar c} \varepsilon_{\alpha \beta \gamma} \dot{x}_\beta B_\gamma = - \frac{e}{\hbar} E_\alpha.
\end{align*}
Решение можем найти, переписа в виде
\begin{equation*}
	\begin{pmatrix}
	    \delta_{\alpha \beta} & \varepsilon_{\alpha \beta \gamma} \Omega^\gamma_n  \\
	    \frac{e}{\hbar c}\varepsilon_{\alpha \beta \gamma} B^\gamma & \delta_{\alpha \beta}  \\
	\end{pmatrix} \begin{pmatrix}
		\dot{x}_\alpha \\ \dot{k}^\beta
	\end{pmatrix} = \begin{pmatrix}
		v_n^{\alpha} \\  -\frac{e}{\hbar} E^{\alpha}
	\end{pmatrix},
\end{equation*}
для координаты и импульса
\begin{align*}
	\dot{x}_\alpha &= (1 + \frac{e}{\hbar c} \vc{B} \cdot \vc{\Omega}_n)^{-1}\left(
		v_n^\alpha + \frac{e}{\hbar c} (\vc{v}_n \cdot \vc{\Omega}_n) B^\alpha + \frac{e}{\hbar} \varepsilon_{\alpha \beta \gamma} E^\beta \Omega_n^\gamma
	\right) \\
	\dot{k}_\alpha &= -\frac{e}{\hbar} \left(1 + \frac{e}{\hbar c} \vc{B} \cdot \vc{\Omega}_n\right)^{-1} \left(
		E^\alpha + \frac{e}{\hbar c} \left(\vc{E} \cdot \vc{B}\right) \Omega_n^\alpha + \frac{1}{c} \varepsilon_{\alpha \beta \gamma} v_n^\beta B^\gamma
	\right).
\end{align*}

\textbf{Несохранение фазового объема}. Заметим, что
\begin{equation*}
	\frac{\partial \dot{x}_\alpha}{\partial x_\alpha}  + \frac{\partial \dot{k}_\alpha}{\partial k_\alpha} = - \frac{d \ln D_n}{d t},
	\hspace{10 mm} 
	D_n(\vc{r}, \vc{k}, t) = 1 + \frac{e}{\hbar c} \vc{B}(\vc{r}, t) \cdot \vc{\Omega}_n (\vc{k}).
\end{equation*}
Таким образом фазовый объем увеличивается в соответствии с 
\begin{equation*}
	\frac{d \ln \Delta V}{d t}  = \vc{\nabla}_{\vc{r}} \cdot \dot{\vc{r}} + \vc{n}_{\vc{k}} \cdot \dot{\vc{k}} = - \frac{\d}{dt} \ln D_n (\vc{r}, \vc{k}, t),
\end{equation*}
где $\Delta V = \Delta \vc{r} \ \Delta \vc{k}$, и тогда $\Delta V(t) = \Delta V(0) / D_n (\vc{r}, \vc{k}, t)$.
Это можно исправить заменой
\begin{equation*}
	\d \mu = \frac{\d^3 r \d^3 k}{(2\pi)^3} 
	\hspace{5 mm} 
	\to
	\hspace{5 mm} 
	\d \tilde{\mu} \overset{\mathrm{def}}{=} D_n(\vc{r}, \vc{k}, t) \frac{\d^3 r \d^3 k}{(2\pi)^3} .
\end{equation*}
и в дальнейшем интегрировать уже в новой метрике.

\textbf{Проводимость}. Среднее для любой локальной наблюдаемой может быть получено в виде
\begin{align*}
	\langle \mathcal O\rangle(\vc{r}, t) = \sum_n \int \d \tilde{\mu} f_n (\vc{r}, \vc{k}, t) \langle u_{n \vc{k}}| \mathcal O | u_{n \vc{k}}\rangle \delta(\vc{r}-\vc{r}')
	.
\end{align*}
В равновесном случае $f_n(\vc{k})$ -- функция Ферми $f(E_n(\vc{k}) - \mu)$. Для тока тогда
\begin{equation*}
	j^\alpha_n (\vc{r}, t) = -e \int  \frac{\d^3 k}{(2\pi)^3} \left(
		v_n^\alpha + \frac{e}{\hbar c} (\vc{v}_n \cdot \vc{\Omega}_n) B^\alpha + \frac{e}{\hbar} \varepsilon_{\alpha \beta \gamma} E^\beta \Omega_n^\gamma
	\right) f_n (\vc{k}).
\end{equation*}
Для случая $\vc{B} = 0$ явно можем найти
\begin{equation*}
	\vc{j}_n = -\frac{e^2}{\hbar} \vc{E} \times \int \frac{\d^3 k}{(2\pi)^3} \vc{\Omega}_n (\vc{k}).
\end{equation*}



% Т12
\setcounter{section}{11}
\section{Тяжелая частица в лёгком газе}
\textbf{Уравнениве Фоккера-Планка}.
Заметим, что
\begin{equation*}
	\frac{\partial f(t, \vc{p})}{\partial t} =  \int d^3 \vc{q}\
	\left(
		w(\vc{p}+\vc{\vc{q},q}) f(t, \vc{p}+\vc{q}) - w(\vc{p}, \vc{q}) f(t, \vc{p})
	\right),
\end{equation*}
раскладываясь до второго порядка малости по $\vc{q}$, находим
\begin{equation*}
	\frac{\partial f(t, \vc{p})}{\partial t}  = 
	\frac{\partial }{\partial p_\alpha} \left(
		\tilde{A}_\alpha f + \frac{\partial }{\partial p_\beta} (B_{\alpha \beta} f)
	\right),
\end{equation*}
где ввели
\begin{equation*}
	\tilde{A}_\alpha = \int q_\alpha w(\vc{p}, \vc{q}) \d^3 \vc{q} = \frac{\sum_{\delta t} q_\alpha}{\delta t},
	\hspace{10 mm} 
	B_{\alpha \beta} = \frac{1}{2} \int q_\alpha q_\beta q(\vc{p}, \vc{q}) \d^3 \vc{q} = \frac{\sum_{\delta t} q_\alpha q_\beta}{2 \delta t}.
\end{equation*}
Перепишем уравнение в виде
\begin{equation*}
	\frac{\partial f}{\partial t}  = - \frac{\partial s_\alpha}{\partial p_\alpha},
	\hspace{5 mm} \Leftrightarrow \hspace{5 mm} 
	\frac{\partial f}{\partial t}  + \div_{\vc{p}} \vc{s} = 0,
\end{equation*}
где величина $\vc{s}$ -- плотность потока в импульсном пространстве
\begin{equation*}
	s_\alpha = - \tilde{A}_\alpha f - \frac{\partial }{\partial p_\beta} (B_{\alpha \beta} f) = - A_\alpha f - B_{\alpha \beta} - B_{\alpha \beta} \frac{\partial f}{\partial p_\beta},
	\hspace{10 mm} 
	A_\alpha = \tilde{A}_\alpha + \frac{\partial B_{\alpha \beta}}{\partial p_\beta}.
\end{equation*}
Итого в общем виде уравнение Фоккера-Планка можем иметь вид
\begin{equation}
	\frac{\partial f}{\partial t} + \vc{v} \frac{\partial f}{\partial \vc{r}} + \vc{F} \frac{\partial f}{\partial \vc{p}} + \div_{\vc{p}} \vc{s} = 0.
\end{equation}
Считатать обычно проще $B_{\alpha \beta}$, а потом они друг через друга выражаются с учётом того, что в равновесии поток $\vc{s}^{(0)}$ зануляется.



\textbf{Тяжелые частицы}. Будем считать, что
\begin{equation*}
	f^{(0)} \sim \exp\left(
		- \frac{p^2}{2MT}
	\right).
\end{equation*}
Начнём с вычисления коэффициентов $B_{\alpha \beta}$:
\begin{equation*}
	B_{\alpha \beta} = B \delta_{\alpha \beta},
	\hspace{10 mm} 
	B = \frac{\sum_{\delta t} q^2}{6 \delta t},
\end{equation*}
и кинетическое уравнение перепишется в виде
\begin{equation*}
	\frac{\partial f(t, \vc{p})}{\partial t} = B \div_{\vc{p}} \left(
		\frac{\vc{p} f}{MT} + \nabla_{\vc{p}} f
	\right).
\end{equation*}
% уравнение Полуховского
% D = b T -- соотношение эйнштейна, из зануления потока в равновесии
Таким образом $B$ иммет смысл коэффициента диффузии в импульсном пространстве. 

Для определения величины $B$ выразим
\begin{equation*}
	\vc{q} = \Delta \vc{p}_b = \vc{p}_b - \bar{p}_b' = \vc{p}_a' - \vc{p}_a.
\end{equation*}
Считая, что $p_a = p_a'$, находим
\begin{equation*}
	q^2 = 2 p_a^2 - 2 p_a^2 \cos \theta = 2 p_a^2 (1-\cos \theta).
\end{equation*}
И тогда можем посчитать интеграл вида
\begin{equation*}
	B = \frac{1}{6} \frac{\sum_{\delta t} q^2}{\delta t} = \frac{1}{6} \int 2 p_a^2 (1-\cos \theta) f_a^{(0)} (\vc{p}_a) v_a \d \sigma \d^3 \vc{p}_\alpha.
\end{equation*}
Вводя $n = \int f^{(0)}(\vc{p}_\alpha) \d^3 \vc{p}_a$, находим
\begin{equation*}
	B = \frac{n_a}{3m} \langle p_a^3 \sigma_t(v_a)\rangle = \frac{m^2 n_a}{3} \langle v_a^3 \sigma_t \rangle,
\end{equation*}
где $m$ -- масса легкой частица, $\sigma_t$ -- транспортное сечение рассеяния легких частиц на тяжелых. 
% Тяжелые частицы диффузируют в легких


\textbf{Диффузия}. Добавив силу $\vc{F}$ можем найти
\begin{equation*}
	\frac{\partial f}{\partial t} + \frac{\partial }{\partial \vc{p}} \left(
		\left(\vc{F} - \frac{B \vc{p}}{MT}\right) f - B \frac{\partial f}{\partial \vc{p}} 
	\right) = 0.
\end{equation*}
На больших $\vc{p}$ функция распределения обращается в ноль. Для стационарного случая
\begin{equation*}
	\left(\vc{F} - \frac{B \vc{p}}{MT}\right) f - B \frac{\partial f}{\partial \vc{p}}  = \const = 0.
\end{equation*}
Функция распределения при внешней силе модифицируется к виду
\begin{equation*}
	f \sim \exp\left(
		- \frac{(\vc{p} - \vc{F} M T/B)^2}{2MT}
	\right) = \exp\left(
		- \frac{(\vc{p} - M \vc{u})^2}{2MT}
	\right),
	\hspace{10 mm} 
	\vc{u} = \frac{T}{B} \vc{F},
\end{equation*}
где $\vc{u}$ -- средняя потоковая скорость. Вообще подвижность $b$ определяется из $\vc{u} = b \vc{F}$, откуда находим $b = T/B$ и $D = b T = T^2/B$.




% Т12
% 
% со звёздочкой, не обязательно

% Т14 
% \newpage
\setcounter{section}{13}
\section{x Броуновское движение}
% #12 по гайду

Добавим случайную силу к уравнению движения
\begin{equation*}
	m \frac{d \vc{v}}{d t} = \Frand (t) + \vc{F}(t),
	\hspace{0.5cm} \Rightarrow \hspace{0.5cm}	
	m \frac{d \langle \vc{v}\rangle}{d t} = \langle \Frand (t) \rangle + \vc{F}(t).
\end{equation*}
Вообще $\langle \vc{v}\rangle = b \vc{F}$, при этом $\langle \Frand(t)\rangle + \vc{F} = 0$, тогда
\begin{equation*}
	\Frand (t) = - \frac{\vc{v}(t)}{b} + \frand(t),
	\hspace{10 mm} 
	\langle \frand(t) \rangle = 0.
\end{equation*}
Тогда уравнение движения перепишется в виде
\begin{equation*}
	\frac{d \vc{v}}{d t} = - \gamma \vc{v} + \frac{1}{M}\left(
		\frand(t) + \vc{F}(t)
	\right),
	\hspace{10 mm} 
	\gamma = \frac{1}{bM}.
\end{equation*}
Уже можем сказать, что
\begin{equation*}
	\langle \frand_i(t) \frand_k(t')\rangle = \kappa \delta_{ik} \delta(t-t').
\end{equation*}
Введём также $\ffull = \frand + \vc{F}(t)$. Теперь перейдём к Фурье-образу
\begin{equation*}
	\vc{v}(t) = \int_{-\infty}^{+\infty} v_\omega e^{- i \omega t} \frac{\d \omega}{2\pi},
	\hspace{10 mm} 
	\ffull(t) = \int_{-\infty}^{+\infty} \ffull_\omega e^{- i \omega t} \frac{\d \omega}{2\pi}.
\end{equation*}
Тогда уравнение перепишется в виде
\begin{equation*}
	- i \omega \vc{v}_\omega = - \gamma \vc{v}_\omega + \frac{1}{M} \ffull_\omega,
	\hspace{0.5cm} \Rightarrow \hspace{0.5cm}	
	\vc{v}_\omega = \frac{\ffull_\omega}{M(\gamma-i \omega)}.
\end{equation*}
Полезно ввести отклик системы $\vc{r}_\omega$
\begin{equation*}
	\vc{r}_\omega = \frac{\vc{v}_\omega}{-i \omega} = \chi(\omega) \ffull_\omega,
	\hspace{10 mm} 
	\chi(\omega) = \frac{i \gamma/\omega - 1}{M(\gamma^2 + \omega^2)},
	\hspace{5 mm} 
	|\chi|^2 = \frac{1}{M^2 \omega^2 (\gamma^2 + \omega^2)} = \frac{\Im \chi}{M \omega \gamma}.
\end{equation*}

\textbf{Диссипативная теорема}. Рассмотрим
\begin{equation*}
	x(t) = \int_{-\infty}^{+\infty} x_\omega e^{- i \omega t} \frac{\d \omega}{2\pi},
\end{equation*}
тогда коррелятор
\begin{equation*}
	\langle x(t) x(t')\rangle = \int_{-\infty}^{+\infty} \int_{-\infty}^{+\infty} \langle x_{\omega} x_{\omega'}\rangle e^{- i \omega t - i \omega' t} \frac{\d \omega \d \omega'}{(2\pi)^2}.
\end{equation*}
Учитывая, что $\langle x_\omega x_{\omega'}\rangle = 2 \pi (x^2)_\omega\delta(\omega+\omega')$, где $(x^2)_\omega$ -- спектральная плотность, находим
\begin{equation*}
	\langle x(t) x(t')\rangle = \int_{-\infty}^{+\infty} (x^2)_\omega e^{- i \omega(t-t')} \frac{\d \omega}{2\pi}.
\end{equation*}
Для $t=t'$ просто
\begin{equation*}
	\langle x^2(t)\rangle = \int_{-\infty}^{+\infty} (x^2)_\omega \frac{\d \omega}{2\pi}.
\end{equation*}
Но мы знаем, что $\vc{v} = d \vc{r} / dt$, и тогда
\begin{equation*}
	(v_i v_k)_\omega = - i \omega \chi(\omega) \cdot i \omega \chi(-\omega) \cdot \left(
		\frand_i \frand_k
	\right)_\omega = \omega^2 |\chi(\omega)|^2 (\frand_i \frand_k)_\omega.
\end{equation*}

Для нахождения константы удобно посмотреть на одновременной коррелятор
\begin{equation*}
	\langle v_i (t) v_k (t)\rangle = \delta_{ik} \frac{T}{M} = \int_{-\infty}^{+\infty} (v_i v_k)_\omega \frac{\d \omega}{2\pi} = \int_{-\infty}^{+\infty} \omega^2 |\chi(\omega)|^2 (\frand_i \frand_k)_\omega \frac{\d \omega}{2\pi}.
\end{equation*}
Итак, получаем уравнение 
\begin{equation*}
	\delta_{ik} \frac{T}{M} = \delta_{ik} \kappa  \int_{-\infty}^{+\infty} \omega^2 \frac{1+\gamma^2/\omega^2}{M^2(\gamma^2+\omega^2)^2} \frac{\d \omega}{2\pi}.
\end{equation*}
Интеграл равен $\pi/\gamma$ и тогда
\begin{equation}
	\kappa = 2 \gamma M T.
\end{equation}
Искомый коррелятор тогда равен
\begin{equation*}
	\langle \frand_i(t) \frand_k(t')\rangle =  2 \gamma M T \delta_{ik} \delta(t-t').
\end{equation*}
Аналогично можем найти
\begin{equation*}
	\langle v_i v_k\rangle_\omega = \omega^2 |\chi(\omega)|^2 \cdot 2 \gamma MT \delta_{ik},
	\hspace{10 mm} 
	\langle x_i x_k\rangle_\omega = |\chi(\omega)|^2 \cdot 2 \gamma MT \delta_{ik}.
\end{equation*}
Подставляя через мнимую часть отклика, находим
\begin{equation*}
	\langle v_i v_k\rangle_\omega = 2 \delta_{ik} T \omega \Im \chi,
	\hspace{10 mm} 
	\langle x_i x_k\rangle_\omega = 2 \delta_{ik} \frac{T}{\omega} \Im \chi.
\end{equation*}
Более явно можем найти 
\begin{equation*}
	\langle v_i(t) v_k(t')\rangle = \int_{-\infty}^{+\infty} (v_i v_k)_\omega e^{- i \omega(t-t')} \frac{d \omega}{2\pi} = \delta_{ik} \int_{-\infty}^{+\infty} \frac{2T \gamma e^{- i \omega(t-t')}}{M(\gamma^2+\omega^2)} \frac{\d \omega}{2\pi} = \delta_{ik} \frac{T}{M} e^{-\gamma|t-t'|}.
\end{equation*}
Интегрируя полученное выражение по $t$, получим
\begin{equation*}
	\int_{t'}^{\infty} \langle  v_i(t) v_k(t')\rangle \d t = \delta_{ik} \frac{T}{\gamma M} = \delta_{ik} \cdot \frac{T}{M} \cdot b M = \delta_{ik} b T = D \delta_{ik}.
\end{equation*} 


% 15, 16, 18, 19 -- решены у лектора, часть из смотреть в бурмистрове


\subsection{Среднеквадратичное отклонение}

Частица двигается случайным образом и хотим найти $\Delta \vc{r} (t) = \vc{r}(t+t_0) - \vc{r}(t_0)$. Умеем выражать коррелятор через спектральную плотность:
\begin{equation*}
	\left\langle (\Delta \vc{r}(t))^2\right\rangle = 
	2 \int_{-\infty}^{+\infty} (1-e^{-i \omega t}) (\vc{r}^2)_\omega \frac{\d \omega}{2\pi}  = 2 \int_{-\infty}^{+\infty} 
	(1-e^{- i \omega t}) \frac{6T}{\omega} \Im \chi
	\frac{\d \omega}{2\pi}.
\end{equation*}
Подставляя $\Im \chi$, находим
\begin{equation*}
	\left\langle (\Delta \vc{r}(t))^2\right\rangle = \frac{6 T t}{M \gamma} \left(1 - \frac{1-e^{-\gamma t}}{\gamma t}\right).
\end{equation*}
Таким образом есть два предела: при $\gamma t \gg 1$:
\begin{equation*}
	\left\langle (\Delta \vc{r}(t))^2\right\rangle = 6 D t,
\end{equation*}
где $\gamma = 1/ bM$, $T b = D$.  И для $\gamma t \ll 1$ получается
\begin{equation*}
	\left\langle (\Delta \vc{r}(t))^2\right\rangle = \frac{3 T}{M} t^2 = \langle \vc{v}^2\rangle t^2,
\end{equation*}
то есть просто свободное движение со средней тепловой скоростью. 


% \subsection{Уравнение Фоккера–Планка для броуновского движения.}

% Вспоминая уравнение
% \begin{equation*}
% 	\frac{\partial n}{\partial t} = \frac{\partial }{\partial x_\alpha} \left(
% 		\tilde{A}_\alpha n + \frac{\partial (B_{\alpha \beta} n)}{\partial x_\beta} 
% 	\right),
% 	\hspace{10 mm} 
% 	B_{\alpha \beta} = \delta_{\alpha \beta} D.
% \end{equation*}


% ландау лифшиц, 
 % #12 по гайду



% Т17
\setcounter{section}{16}
\section{Уравнение Смолуховского}
\subsection{Сведение к осциллятору}

Работаем примерно с уравнением 
\begin{equation*}
	\frac{\partial n}{\partial t} = D \Delta n + \div(b n \nabla U),
\end{equation*}
точнее с уравнением вида
\begin{equation*}
	\frac{\partial P}{\partial t} = \frac{D}{T} k P + \frac{D}{T} k x \frac{\partial P}{\partial x}  + D \frac{\partial^2 P}{\partial x^2} ,
\end{equation*}
где подставили потенциал $U = kx^2/2$.

Введём $g=2 \gamma T$ и $\tau = b t$, тогда можем сделать подстановку
\begin{equation*}
	P(x, \tau) = e^{-\gamma k x^2/2g} \psi(x, \tau),
\end{equation*}
получаем уравнение вида
\begin{equation*}
	\frac{1}{k} \frac{\partial \psi}{\partial \tau} = \left(
		\frac{1}{2} - \frac{x^2}{4A}
	\right) \psi + A \frac{\partial^2 \psi}{\partial x^2},
	\hspace{10 mm} 
	A = \frac{g}{2k \gamma} = \frac{T}{k}. 
\end{equation*}
Таким образом пришли к гамильтониану гармонического осциллятора, с собственными функциями в виде полиномов эрмита
\begin{equation*}
	\varphi_n (x) = \frac{1}{\sqrt{2^n n! \sqrt{2\pi A}}} H_n\left(\frac{x}{\sqrt{2A}}\right) e^{-x^2/4A}.
\end{equation*}
Итого, искомая вероятность
\begin{equation*}
	P(x, \tau) = e^{-\gamma k x^2/2g} \psi(x, \tau) = e^{-x^2/4A} \psi(x, \tau) = \sum_{n=0}^{\infty} a_n e^{- n k \tau} \varphi_0(x) \varphi_n(x).
\end{equation*}


\subsection{Забываются начальные условия}

Забавный факт:
\begin{equation*}
	\sum_{n=0}^{\infty} \frac{H_n(x) H_n(y)}{n!} \left(\frac{U}{2}\right)^n = \frac{1}{\sqrt{1-U^2}} \exp\left(
		\frac{2 u xy}{1+u} - \frac{u^2 (x-y)^2}{1-u^2}
	\right),
\end{equation*}
и воспользуемся соотношением ортогональности, что найти эволюцию от $P(x, 0) = \delta(x-x_0)$:
\begin{equation*}
	a_n = \frac{1}{\sqrt{2^n n!}} H_n\left(\frac{x_0}{\sqrt{2A}}\right) = \frac{\varphi_n(x_0)}{\varphi_0(x_0)}.
\end{equation*}
Итого, эволюция запишется в виде
\begin{align*}
	P(x, \tau) &= \sum_{n=0}^{\infty} e^{-n k \tau} \frac{\varphi_0(x)}{\varphi_0(x_0)} \varphi_n(x) \varphi_n(x_0)
	= \frac{1}{\sqrt{2\pi A}} e^{-x^2/2A} \sum_{n=0}^{\infty} \frac{1}{2^n n!} e^{-n k \tau} H_n\left(\frac{x_0}{\sqrt{2A}}\right) H_n \left(\frac{x}{\sqrt{2A}}\right) \\ 
	&= \frac{1}{\sqrt{2\pi A(1-e^{-2k \tau})}} \exp\left(
		- \frac{(x-x_0 e^{- k \tau})}{2A(1-e^{-2 k \tau})}
	\right),
\end{align*}
где подставили ту сумму с $u = e^{- k \tau}$. Таким образом начальные условия забываются!


% Теперь работаем с непрерывным пространство, поэтому ПОВМ будем искать в виде
% \begin{align*}
% 	\Pi_0 &= \int p(0, x) \kb{0}{x} \d x, \\
% 	\Pi_1 &= \int p(1, x) \kb{1}{x} \d x.
% \end{align*}
% Для удобства введём функцию $I_\Delta (x)$:
% \begin{equation*}
% 	I_\Delta (x) = \left\{\begin{aligned}
% 	    &0, &|x|>\Delta, \\
% 	    &1, &|x|<\Delta.
% 	\end{aligned}\right.
% \end{equation*}
% Тогда искомые вероятности запишутся в виде
% \begin{align*}
% 	p(1, x) &= p \cdot I_\Delta(x), \\
% 	p(0, x) &= (1-p)I_\Delta(x) + (1-I_\Delta(x)).
% \end{align*}
% \begin{align*}
% 	\hat{\Omega}_{0} &= \sqrt{p} \kb{0}{0} + \sqrt{1-p} \kb{1}{1} \\
% 	\hat{\Omega}_{1} &= \sqrt{1-p} \kb{0}{0} + \sqrt{p} \kb{1}{1} \\
% \end{align*}

% \begin{equation*}
% 	P_0 = P_1 = \frac{1}{2}\left(\frac{1}{2} - \sqrt{(1-p)(p)}\right)
% \end{equation*}

Состояние с заданной $x$-компонентной спина -- собственное для $\hat{\sigma}_x$: $\psi \sim (\pm 1, 1)$, а значит 
\begin{equation*}
	\psi(t) \sim \begin{pmatrix}
		\pm e^{\mp i \Omega t/2} \\ e^{\mp i \Omega t/2}
	\end{pmatrix},
	|\psi(t)|^2 = \const,
\end{equation*}
то есть ситема будет равновероятно наблюдаться в состояние $\ket{0}$ или $\ket{1}$. 

Для постоянных измерений будем работать с системой, вида
\begin{equation*}
	\dot{\rho}_x = - 2 \gamma \rho_x,
	\hspace{5 mm} 
	\dot{\rho}_y = -\Omega \rho_z - 2 \gamma \rho_y,
	\hspace{5 mm} 
	\dot{\rho}_z = \Omega \rho_y,
\end{equation*}
которая очевидно для $\rho_x = 1$ будет иметь решение, вида
\begin{equation*}
	\rho_x (t) = e^{-2\gamma t}.
\end{equation*}




% зависимость коэффициентов теплопроводности по Т8 для полупроводников и металлов. 
% первые две пары, ещё можно после 15:20.

% после 15:20 здесь или на кафедре, 526


% У
\setcounter{section}{19}
\section{Упражнения}
\subsection{У1}

Матрица перехода
\begin{equation*}
	T = \begin{pmatrix}
	    a & b  \\
	    1-a & 1-b  \\
	\end{pmatrix} = \begin{pmatrix}
	    0.85 & 0.5  \\
	    0.15 & 0.5  \\
	\end{pmatrix}.
\end{equation*}
Собственный вектор c $\lambda=1$:
\begin{equation*}
	\vc{q} = \left(
		\frac{b}{1-a+b},\ 
		\frac{a-1}{a-1-b}
	\right) \approx \begin{pmatrix}
		0.77 \\ 0.23
	\end{pmatrix}.
\end{equation*}
Для него выполняется детальный баланс $T_{j \neq i} q_i = T_{i \neq j} q_j$. Таким образом данный процесс обратимый.



\subsection{У2}

Рассмотрим одномерное случайное блуждение с для разных $p$. Матрица перехода имеет вид
\begin{equation*}
	T = \begin{pmatrix}
		q & q & 0 & 0 & \ldots \\
		p & 0 & q & 0 & \ldots \\
		0 & p & 0 & q & \ldots \\
		0 & 0 & p & 0 & \ldots \\
		\ldots &&&&
	\end{pmatrix}
\end{equation*}
И снова ищем $T \vc{x} = \vc{x}$, что значит 
\begin{equation*}
	x_0 p = x_1 q, 
	\hspace{5 mm} 
	x_1 p = x_2 q, 
	\hspace{5 mm} 
	\ldots
\end{equation*}
откуда находим
\begin{equation*}
	\frac{1}{x_0} =
		1 + \sum_{i=0}^{N-1} \left(\frac{p}{1-p}\right)^{i}
	= 1 + \frac{1-(p/q)^{N}}{1-p/q} = \left\{\begin{aligned}
	    &\infty, &p \geq 0.5 \\
	    & \tfrac{2-3p}{1-2p}, &p<0.5
	\end{aligned}\right.
\end{equation*}
Видно, что при $p \geq 0.5$ сумма расходится и $x_0 \to 0$, а при $x_0(p < 0.5)$ конечна, гарантировано возвращаемся. 
