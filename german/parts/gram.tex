\section{Граматика}



\subsection{Perfect}

% haben  | +  Partizip II
% sein   |


% 1) слабые глаголы образуют Partizip II при помощи добавления приставки ge- и суффикса -t, например:

% machen – gemacht (делать)
% lachen – gelacht (смеяться)
% arbeiten – gearbeitet (работать)

% 2) Сильные глаголы

% gehen – gegangen (идти)
% stehen – gestanden (стоять)
% schlafen – geschlafen (спать)



% Еще примеры Partizip II глаголов с отделяемой приставкой:


% umziehen - umgezogen (переезжать)
% mitbringen – mitgebracht (приносить)
% ankommen – angekommen (прибывать)

% Если приставка неотделяемая, то приставки ge- просто не будет, но не забывайте обращать внимание на то, слабый глагол или сильный!

% Например:

% beginnen - begonnen (начинать)
% erzählen - erzählt (рассказывать)
% vergessen – vergessen (забывать)


% возвратные глаголы
% https://www.de-online.ru/index/0-95

% модальные глаголы
% https://www.de-online.ru/modalnie_slova


% глаголы: 
	% сильные (меняется корневая гласная, своя форма прошедшего)/слабые
	% правильные/неправильные
% https://www.de-online.ru/gram-tabelle/tablica_Deutsch-online.pdf

% Слабые глаголы образуют форму Präteritum по общему правилу: путем добавления к основе суффикса –(e)te, а форму Partizip II - путем добавления к основе приставки ge-  и суффикса -(e)t к основе глагола, например: arbeiten (работать) – arbeitete – gearbeitet. 


% Времена:
% Perfect/Partizip

