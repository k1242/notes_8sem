\section*{Семинар 4}


\subsection*{Т4. Упругое рассеяние электронов на примесях}


Рассеяние электронов на примесях
\begin{equation*}
	\left(
		\frac{\partial }{\partial t} + \vc{v} \frac{\partial }{\partial \vc{r}} +  \dot{\vc{k}} \frac{\partial }{\partial \vc{k}} 
	\right) f = I_{\vc{k}} = - \frac{\delta f}{\tau}.
\end{equation*}
В данном случае для линеаризованного кинетического уравнения $\tau$-приближение является точным, где $\delta f = f_{k}-f_0$. 


Поработаем с самим интегралом столкновений
\begin{equation*}
	I_k = \frac{2\pi}{\hbar} \sum_{\vc{k}'} \left(
		|\bk{\vc{k}'}[\sub{U}{пол}]{\vc{k}}|^2 \delta (\varepsilon_{k}-\varepsilon_{k'}) \left[
			f_{k'}(1-f_{k}) - f_k (1-f_{k'})
		\right]
	\right),
\end{equation*}
где $f_{k'}(1-f_{k}) - f_k (1-f_{k'}) = f_{k'}-f_k$. Для матричного элемента
\begin{equation*}
	\sub{U}{пол} (\vc{r}) = \sum_{j=1}^{N} U(\vc{r}-\vc{R}_j),
	\hspace{10 mm} 
	\langle \vc{k}| = \frac{1}{\sqrt{V}} e^{i \vc{k} \vc{r}}.
\end{equation*}
Тогда для матричного эдлемента находим
\begin{equation*}
	|\bk{\vc{k}'}[\sub{U}{пол}]{\vc{k}}|^2 = \frac{1}{V^2} |\tilde{U}(\vc{k}-\vc{k}')|^2 \cdot \bigg|
		\sum_{j=1}^{N} e^{i(\vc{k}-\vc{k}')\vc{R}_j}
	\bigg|^2,
	\hspace{10 mm} 
	\tilde{U} (\vc{q}) = \int e^{i \vc{q} \vc{r}} U(\vc{r}) \d^3 \vc{r}.
\end{equation*}
Усредняя по случайному положению примесей
\begin{equation*}
	\left\langle \bigg| 
		\sum_{j=1}^{N} e^{i (\vc{k}-\vc{k}') R_j}
	\bigg| \right\rangle_{\text{прим}} = N + N (N-1) \delta_{\vc{k}, \vc{k}'}.
\end{equation*}
Для матричного элемента получили выражение
\begin{equation*}
	|\bk{\vc{k}'}[\sub{U}{пол}]{\vc{k}}|^2 = \frac{N}{V^2} |\tilde{U}(\vc{q})|^2 + \frac{N(N-1)}{V^2} |\tilde{U}(0)|^2 \delta_{\vc{k}, \vc{k}'},
	\hspace{10 mm} 
	\vc{q} = \vc{k}-\vc{k}'.
\end{equation*}
Итого для интеграла столкновений получаем выражение после замены $\sum_{\vc{k}} \to \int \frac{V \d^3 k}{(2 \pi)^3}$
\begin{equation*}
	I_k (f) = \frac{2\pi n}{\hbar} \int \frac{d^3 \vc{k}'}{(2\pi)^3} |\tilde{U} (\vc{q})|^2 \delta(\varepsilon_k - \varepsilon_{\vc{k}}) \cdot \left(
		\delta f_{k'} - \delta f_{k}
	\right),
\end{equation*}
где уже линеаризовали выражение. Здесь $n$ -- примесное.



Рассмотрим стационарный однородный случай, когда $\hbar \dot{\vc{k}} = - e \vc{E}$, где поле считаем малой поправкой, тогда
\begin{equation*}
	\dot{\vc{k}} \frac{\partial }{\partial \vc{k}}  = - e \vc{E} \cdot \vc{v} \frac{\partial }{\partial \varepsilon},
	\hspace{10 mm} 
	\delta f_k \overset{\mathrm{def}}{=}  \tau(\varepsilon) \left(e \vc{E} \cdot \vc{v}_{\vc{k}} \right) \frac{\partial f_0}{\partial \varepsilon},
\end{equation*}
то есть ищем решение в $\tau$-приближение. Получается уравнение
\begin{equation*}
	- (\vc{E} \cdot \vc{v}) \frac{\partial f_0}{\partial \varepsilon} = I_k = \frac{2\pi n}{\hbar} e \vc{E} \int \frac{d^3 \vc{k}'}{(2\pi)^3} |\tilde{U}(\vc{q})|^2 \delta(\varepsilon_k - \varepsilon_{k'}) \left(
		\tau (\varepsilon') \vc{v}' \frac{\partial f_0}{\partial \varepsilon} \bigg|_{\varepsilon'} 
		-
		\tau (\varepsilon) \vc{v} \frac{\partial f_0}{\partial \varepsilon} \bigg|_{\varepsilon}
	\right).
\end{equation*}
Сокращая $\partial_\varepsilon f_0$ и всё лишнее, находим
\begin{equation*}
	\vc{v} = \frac{\hbar \vc{k}}{m} = \frac{n \tau(\varepsilon[\vc{k}])}{4 \pi^2 \hbar} \int_{0}^{\infty} dk' \ (k')^2 \int d \Omega_{k'} |\tilde{U}(\vc{q})|^2 \cdot \frac{\delta(k-k')}{\hbar^2 k/m} \frac{\hbar}{m} (\vc{k}-\vc{k}').
\end{equation*}
Остаётся выражение
\begin{equation}
	\frac{1}{\tau(\varepsilon)} = \frac{m k n}{4 \pi^2 \hbar^3} \int d \Omega_k \ |\tilde{U}(\vc{q})|^2 (1 - \hat{\vc{k}} \cdot \hat{\vc{k}}'),
	\hspace{10 mm} 
	\hat{\vc{k}} = \frac{\vc{k}}{k}.
\end{equation}



\textbf{Дифференциальное сечение рассеяния}. Найдём выражение
\begin{equation*}
	\frac{\d \sigma}{\d \Omega} = \left(\frac{m}{2\pi \hbar^2}\right)^2 |\tilde{U}(\vc{q})|^2,
	\hspace{10 mm} 
	\vc{q} = \vc{k}-\vc{k}',
	\hspace{5 mm} 
	q^2 = 4 k^2 \sin^2 \left(\frac{\theta}{2}\right).
\end{equation*}
И интеграл столкновений перепишется в виде
\begin{equation}
	\frac{1}{\tau(\varepsilon)} = n v \int \frac{d \sigma}{d \Omega} (1-\cos \theta) \d \Omega = n v \sub{\sigma}{tr},
\end{equation}
где возникло новое $\sub{\sigma}{tr}$ с подавленным рассеянием на малых углах. 


Вспоминая формулу Друде, находим
\begin{equation*}
	\vc{j} = \sigma_D \vc{E},
	\hspace{10 mm} 
	\sigma_D = \frac{e^2 n_0 \sub{\tau}{tr}}{m},
\end{equation*}
где входит именно $\tau_{\text{tr}}$.





\textbf{Фурье-образ}. Для экранированного кулоновского потенциала 
\begin{equation*}
	U(r) = - e^{-r/\lambda} \frac{Z e^2}{r},
	\hspace{10 mm} 
	\tilde{U}(\vc{q}) = \int U(\vc{r}) e^{-i \vc{q} \vc{r}} \d V = \frac{4 \pi Z e^2}{q^2 + \lambda^{-2}}.
\end{equation*}
Для дифференциального сечения рассеяния находим
\begin{equation*}
	\frac{d \sigma}{d \Omega} = \frac{m^2}{4 \pi^2 \hbar^2} \left(
		\frac{4\pi Z e^2}{q^2+\lambda^{-2}}
	\right)^2 = \left(
		\frac{Z e^2}{4 E_F} \frac{1}{\sin^2 \frac{\theta}{2} + (2 k_F \lambda)^{-2}}
	\right)^2.
\end{equation*}
где $q = 2 k_F \sin \frac{\theta}{2}$.
Полное сечение рассеяния тогда получается
\begin{equation*}
	\sigma = \int \d \sigma = \int_{0}^{\pi} \left(
		\frac{Z e^2}{4 E_F} \frac{1}{\sin^2 \frac{\theta}{2} + (2 k_F \lambda)^{-2}}
	\right)^2 = \frac{2\pi Z^2 e^4}{4 E_F^2} \int_{0}^{2}  \frac{\d u}{(u + \frac{1}{2}(k_F \lambda)^{-2})^2},
\end{equation*}
где $u = 1 - \cos \theta$.  Итого, введя $\zeta \overset{\mathrm{def}}{=}  \frac{4}{\pi} (k_F \lambda)^2$, находим
\begin{equation*}
	\sigma = \frac{\pi Z^2 e^4}{2 E_F^2} \frac{(\pi \zeta)^2/2}{1 + \pi \zeta}.
\end{equation*}
Для транспортного $\sub{\sigma}{tr}$, находим
\begin{equation*}
	\sub{\sigma}{tr} = \frac{2\pi Z^2 e^4}{4 E_F^2} \int_{0}^{2}  \frac{u \d u}{(u + \frac{1}{2}(k_F \lambda)^{-2})^2} = \frac{\pi Z^2 e^4}{2 E_F^2} \left(
		\ln(1+\pi \zeta) - \frac{\pi \zeta}{1+\pi \zeta}
	\right).
\end{equation*}
Для проводимости $\rho$ можем найти
\begin{equation*}
	\rho = \frac{m}{n e^2 \sub{\tau}{tr}} = \frac{m n v_F}{n_0 e^2} \frac{\pi Z^2 e^4}{2 E_F^2} \zeta^3 F(\zeta),
	\hspace{10 mm} 
	F(\zeta) = \frac{1}{\zeta^3} \left(
		\ln(1+\pi \zeta) - \frac{\pi \zeta}{1+\pi \zeta}
	\right).
\end{equation*}
Итого, находим
\begin{equation}
	\rho = Z^2 R_q \sub{a}{B} \frac{n}{n_0} F(\zeta) \cdot \left[
		\frac{e^2}{2 \pi \hbar} \frac{m e^2}{\hbar^2} \frac{p_F}{e^2} \frac{\pi e^4}{p_F^2/2m^2} \frac{64 k_F^6 \lambda^6}{\pi^3}
	\right] = Z^2 R_q \sub{a}{B} \frac{n}{n_0} F(\zeta).
\end{equation}
где подставили $\lambda^2 = \frac{\pi \sub{a}{B}}{4 k_F}$.




\subsection*{Т5. Рассеяние электронов на фононах}


\textbf{Эффект Иоффе-Регеля}. На высоких температурах $r^2 \sim T$ для ионов, тогда
\begin{equation*}
	\tau = \frac{1}{\sub{n}{ion} v \sigma} \sim \frac{1}{T},
	\hspace{10 mm} 
	\rho = \frac{m}{n e^2 \tau} \sim T.
\end{equation*}
Для $\tau v_F \sim \lambda_F$, можем записать с учётом $n \sim k_F^3$
\begin{equation*}
	\rho = \frac{m v_F}{n e^2 \tau v_F} = \frac{m v_F}{n e^2 \lambda_F} \sim \frac{\hbar}{k_F e^2},
\end{equation*}
что называется пределом Иоффе-Регеля, которые неплохо работает для легированных полупроводников. 


\textbf{Испускание фононов}. И снова запишем столкновительный интеграл в терминах приход-уход:
\begin{equation*}
	I_p = \sum_{\vc{p}'} w_{\vc{p} \vc{p}'} n_{\vc{p}'} (1-n_{\vc{p}}) - \sum_{\vc{p}'}  w_{\vc{p}' \vc{p}} n_{\vc{p}} (1-n_{\vc{p}'}).
\end{equation*}
Рассматриваем однородную ситуацию, тогда 
\begin{equation*}
	\dot{\vc{p}} \frac{\partial n}{\partial \vc{p}} = 
	- e (\vc{E} \cdot \vc{v}) \frac{\partial n_0}{\partial \varepsilon}
	= \sub{I}{ст}.
\end{equation*}
Учитвая что $w_q \sim q$, можем расписать
\begin{align*}
	\sub{I}{ст} = &\frac{2\pi}{\hbar V} \sum_{\vc{q}} \left(
		w_{\vc{q}} (1 + N_{\vc{q}}) n_{\vc{p} + \hbar \vc{q}} (1-n_{\vc{p}}) \delta (\varepsilon_p - \varepsilon_{\vc{p} + \hbar \vc{q}} + \hbar \omega_q) 
		+ w_q N_q n_{\vc{p}-\hbar \vc{q}} (1-n_{\vc{p}}) \delta(\varepsilon_p - \varepsilon_{\vc{p}-\hbar \vc{q}}- \hbar \omega_q) 
	\right) - \\
	- & \frac{2\pi}{\hbar V} \sum_{\vc{q}}
	\left(
		w_{\vc{q}} (1+N_{\vc{q}}) n_q (1-n_{\vc{p}-\hbar \vc{q}}) \delta (\varepsilon_p - \varepsilon_{\vc{p} - \hbar \vc{q}} - \hbar \omega_q) + w_q N_q (1-n_{\vc{p}+\hbar \vc{q}}) \delta(\varepsilon_p - \varepsilon_{\vc{p}+\hbar \vc{q}} + \hbar \omega_q)
	\right).
\end{align*}
% вставить формулу с фото
Будем считать, что фононы равновесные
\begin{equation*}
	N_q = N_{q}^0 = \frac{1}{e^{\hbar \omega_q/T}-1},
	\hspace{0.5cm} \Rightarrow \hspace{0.5cm}
	\frac{1+N_q}{N_q} = e^{\hbar \omega_q/T}.
\end{equation*}
Для электронов
\begin{equation*}
	n_p^0 = \frac{1}{e^{(\varepsilon_p-\mu)/T}+1},
	\hspace{0.5cm} \Rightarrow \hspace{0.5cm}
	\frac{1-n_p^0}{n_p^0} = e^{(\varepsilon_p-\mu)/T}.
\end{equation*}
Преобразуем выражение из квадратных скобок *
\begin{equation}
	(1+N_q)(1-n_p)(1-n_{p+\hbar q}) \left(
		\frac{n_{\vc{p}+\hbar \vc{q}}}{1-n_{\vc{p}+\hbar \vc{q}}}
		- \frac{N_q}{1+N_q} \frac{n_p}{1-n_p}
	\right),
	\label{eqsq}
\end{equation}
которое очевидно зануляется для равновесных функций. 


Решение будем искать в виде
\begin{equation*}
	n_p = n_{p}^0 + \delta n_p = n_p^0 - \frac{\partial n^0_p}{\partial \varepsilon_p} \Phi_p = n_p^0 + \frac{n^0(\varepsilon_p)(1-n^0(\varepsilon_p))}{T} \Phi_p.
\end{equation*}
Возвращаясь к \eqref{eqsq}, получаем линеаризуя
\begin{align*}
	&(1+N_q)(1-n_p^0) (1-n_{\vc{p} + \hbar \vc{q}}^0) \left[
		\frac{\delta n_{\vc{p} + \hbar \vc{q}}}{(1-n_{\vc{p} + \hbar \vc{q}}^0)^2} - \frac{N_q}{1+N_q} \frac{\delta n_p}{(1-n_p^0)^2}
	\right] = \\
	= & +\frac{1}{T} (1+N_q)(1-n_p^0) n_{\vc{p} + \hbar \vc{q}}^0 \left[
		\Phi_{\vc{p} + \hbar \vc{q}}-\Phi_p
	\right] = \\
	= & - \frac{1}{T} (1+N_q) N_q (n_{\vc{p} + \hbar \vc{q}}^0-n_p^0) \left[
		\Phi_{\vc{p} + \hbar \vc{q}}-\Phi_p
	\right]
	.
\end{align*}
Аналогично преобразуется второе слагаемое в *, откуда находим линеаризованный интеграл столкновений:
\begin{align*}
	\sub{I}{ст}(\Phi_p) &= - \frac{2\pi}{\hbar V} \sum_{\vc{q}} w_q \frac{(1+N_q)N_q (n_{\vc{p}+\hbar \vc{q}}^0-n_p^0)}{T} \left[
		\Phi_{\vc{p}+\hbar \vc{q}} - \Phi_{\vc{p}}
	\right] \times \left[
		\delta(\varepsilon_p - \varepsilon_{\vc{p}+\hbar \vc{q}} + \hbar \omega_q) - \delta(\varepsilon_p - \varepsilon_{\vc{p}+\hbar \vc{q}} - \hbar \omega_q)
	\right]
\end{align*}
Выделим физ. смысл в слагаемых
\begin{align*}
		\sub{I}{ст}(\Phi_p) = - \frac{2 \pi}{\hbar V} \sum_{\vc{q}} w_q \frac{(1+N_q) N_q}{T} \bigg(
		&\left[
			n^0(\varepsilon_p + \hbar \omega_q) - n^0(\varepsilon_p)
		\right] 
		\delta(\varepsilon_p - \varepsilon_{\vc{p}+\hbar \vc{q}} + \hbar \omega_q) 
		- \\ - &
		\left[
			n^0(\varepsilon_p - \hbar \omega_q) - n^0(\varepsilon_p)
		\right] \delta(\varepsilon_p - \varepsilon_{\vc{p}+\hbar \vc{q}} - \hbar \omega_q)
	\bigg) \left[
		\Phi_{\vc{p}+\hbar \vc{q}} - \Phi_{\vc{p}}
	\right]
	.
\end{align*}
Учтём, что мы живём вблизи поверхности Ферми, тогда $\hbar \omega_q$ мало по сравнению с $\varepsilon_p$, приходим к выражению
\begin{equation*}
	\sub{I}{ст}(\Phi_p) = - \frac{\partial n^0}{\partial \varepsilon} \frac{2\pi}{\hbar V} \sum_q w_q \frac{2 \hbar \omega_q (1+N_q) N_q}{T} \delta(\varepsilon_{\vc{p}+\hbar \vc{q}}- \varepsilon_{\vc{p}}) \left[
		\Phi_{\vc{p}+\hbar \vc{q}} - \Phi_{\vc{p}}
	\right].
\end{equation*}
Аргумент $\delta$-функции можем расписать в виде
\begin{equation*}
	\varepsilon_p - \varepsilon_{\vc{p}+\hbar \vc{q}} \pm \hbar \omega_q = 
	\frac{2 p \hbar q \cos \theta}{2 m} + \frac{\hbar^2 q^2}{2m} \pm \hbar c_L q = \frac{\hbar p q}{m}\left(
		\cos \theta + \frac{\hbar q}{2p} \pm \frac{m c_L}{p}
	\right),
\end{equation*}
где $c_L \ll v_F$, поэтому можем опустить последнее слагаемое. 



\textbf{Кинетическое уравнение}. Итого, будем решать кинетическое уравнение на $\Phi_{\vc{p}}$ вида
\begin{equation}
	- e (\vc{E} \cdot \vc{v}) \frac{\partial n^0}{\partial \varepsilon} = \sub{I}{ст}(\Phi_p) = - \frac{\partial n_0}{\partial \varepsilon} \frac{2\pi}{\hbar V} \sum_q w_q \frac{2 \hbar \omega_q (1+N_q) N_q}{T} \delta(\varepsilon_{\vc{p}+\hbar \vc{q}}- \varepsilon_{\vc{p}}) \left[
		\Phi_{\vc{p}+\hbar \vc{q}} - \Phi_{\vc{p}}
	\right].
\end{equation}
Решение аналогично будем искать в виде $\Phi_p = - e (\vc{E} \cdot \vc{v}) \sub{\tau}{tr} (\varepsilon_p)$, что соответствует $\tau$-приближению: $\sub{I}{ст} = - \delta n_p / \tau$. Таким образом остаётся найти $\sub{\tau}{tr}$, и найти остальные величины по формуле Друде.  Выражая из двух уравнений $(\vc{E} \cdot \vc{v})$, находим
% в данном случае метод моментов сводится к одному слагаемому
\begin{equation*}
	(\vc{E} \cdot \vc{v}) = - \frac{4 \pi}{\hbar V} \sum_{\vc{q}} w_q \frac{ \hbar \omega_q (1+N_q) N_q}{T} \delta(\varepsilon_{\vc{p}+\hbar \vc{q}}- \varepsilon_{\vc{p}}) 
	\frac{\hbar (\vc{q} \cdot \vc{E})}{m} \sub{\tau}{tr}(\varepsilon_p).
\end{equation*}
Переходя к интегрированию, нахоим
\begin{equation*}
	(\vc{E} \cdot \vc{v}) = - \frac{4 \pi}{\hbar} \int \frac{q^2 \d q \d \Omega_q}{(2\pi)^3} w_q \frac{\hbar \omega_q (1+N_q) N_q}{T} \delta(\varepsilon_{\vc{p} + \hbar \vc{q}} - \varepsilon_p) \frac{\hbar (\vc{q} \cdot \vc{E})}{m} \sub{\tau}{tr}(\varepsilon_p).
\end{equation*}
Проведём интегрирование, введя полярную ось и расписав
\begin{align*}
	\vc{q}  &= (q \sin \theta \cos \varphi,\, q \sin \theta \sin \varphi,\, q \cos \theta), \\
	\vc{E} &= (E \sin \theta_E \cos \varphi_E,\, E \sin \theta_E \sin \varphi_E,\, E \cos \theta_E).
\end{align*}
Тогда скалярное произведение перепишется в виде
\begin{equation*}
	(\vc{q} \cdot \vc{E}) = q E \left(
		\cos \theta \cos \theta_E + \sin \theta \sin \theta_E \cos(\varphi-\varphi_E)
	\right),
\end{equation*}
где после интегрирование второе слагаемое зануляется. Также подставляя $(\vc{E} \cdot \vc{v}) = E v \cos \theta_E$, тогда
\begin{equation*}
	\frac{p}{m \sub{\tau}{tr}(\varepsilon_p)} = - \frac{4 \pi}{T} \int_0^{q_D} \frac{q^2 \d q \sin \theta \d \theta}{(2\pi)^2} w_q \omega_q (1 + N_q) N_q \times \delta\left(
		\tfrac{\hbar q p}{m}\left(\cos \theta + \tfrac{\hbar q}{2p}\right)
	\right) \times \frac{\hbar q}{m} \cos \theta,
\end{equation*}
где $q_D$ -- максимальный дебаевский импульс. Таким образом
\begin{equation*}
	\frac{1}{\sub{\tau}{tr}(\varepsilon_p)} = \frac{4 \pi m}{T p^2} \int_{0}^{q_D} \frac{q^2 \d q}{(2\pi)^2} w_q \omega_q (1+N_q)N_q \int_{-1}^{1} dx\ x \times \delta\left(x + \tfrac{\hbar q}{2p}\right),
\end{equation*}
где ввели $x = \cos \theta$.

Вообще $q_D = \sqrt[3]{6 \pi^2 n}$, $p_F = \sqrt[3]{3 \pi^2 n}$, тогда $\frac{\hbar q_D}{2 p_F} < 1$. Учитывая, что $w_q \propto \omega_q \propto q$, находим
\begin{equation*}
	\frac{1}{\sub{\tau}{tr}(\varepsilon_p)} \propto \frac{1}{T} \int_{0}^{q_D} q^5 \d q \ \frac{e^{\hbar \omega_q/T}}{(e^{\hbar \omega_q/T}-1)^2}. 
\end{equation*}
Введём $z = \frac{\hbar \omega_q}{T} = \frac{T_D}{T} \frac{q}{q_D}$, где $T_D = \hbar c_L q_D$. Таким образом 
\begin{equation*}
	\frac{1}{\sub{\tau}{tr}(\varepsilon_p)} \propto \frac{1}{T} \left(
		\frac{T}{T_D}
	\right)^6 \int_{0}^{T_D/T} \frac{e^z z^5 \d z}{(e^z-1)^2},
\end{equation*}
где из-за разности скоростей возникла пятая степень вместо четвертой. Итого, искомое выражение 
\begin{equation}
	\frac{1}{\sub{\tau}{tr}(\varepsilon_p)} \propto \left(\frac{T}{T_D}\right)^5 \int_{0}^{T_D/T} \frac{z^5 \d z}{\sh^2 \frac{z}{2}}.
\end{equation}


\textbf{Формула Друде}. Вспоминая, что
\begin{equation*}
	\sigma = \sigma_D = \frac{e^2 n \sub{\tau}{tr}}{m},
\end{equation*}
находим 
\begin{equation*}
	\frac{\sub{\rho}{e-ph}(T)}{\sub{\rho}{e-ph}(T_D)} = \frac{\sub{\sigma}{e-ph}(T)}{\sub{\sigma}{e-ph}(T_D)} = \left(\frac{T}{T_D}\right)^5 \int_{0}^{T_D/T} \frac{z^5 \d z}{\sh^2 \frac{z}{2}} \bigg/ \int_{0}^{1} \frac{z^5 \d z}{\sh^2 \frac{z}{2}}.
\end{equation*}
Для $T \ll T_D$ получится
\begin{equation*}
	\frac{\sub{\rho}{e-ph}(T)}{\sub{\rho}{e-ph}(T_D)} = 526 \left(\frac{T}{T_D}\right)^5.
\end{equation*}
И в обратную сторону, для $T \gg T_D$, раскладываясь в ряд, находим
\begin{equation*}
	\frac{\sub{\rho}{e-ph}(T)}{\sub{\rho}{e-ph}(T_D)} = 1.06 \left(\frac{T}{T_D}\right).
\end{equation*}

