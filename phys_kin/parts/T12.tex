\textbf{Уравнениве Фоккера-Планка}.
Заметим, что
\begin{equation*}
	\frac{\partial f(t, \vc{p})}{\partial t} =  \int d^3 \vc{q}\
	\left(
		w(\vc{p}+\vc{\vc{q},q}) f(t, \vc{p}+\vc{q}) - w(\vc{p}, \vc{q}) f(t, \vc{p})
	\right),
\end{equation*}
раскладываясь до второго порядка малости по $\vc{q}$, находим
\begin{equation*}
	\frac{\partial f(t, \vc{p})}{\partial t}  = 
	\frac{\partial }{\partial p_\alpha} \left(
		\tilde{A}_\alpha f + \frac{\partial }{\partial p_\beta} (B_{\alpha \beta} f)
	\right),
\end{equation*}
где ввели
\begin{equation*}
	\tilde{A}_\alpha = \int q_\alpha w(\vc{p}, \vc{q}) \d^3 \vc{q} = \frac{\sum_{\delta t} q_\alpha}{\delta t},
	\hspace{10 mm} 
	B_{\alpha \beta} = \frac{1}{2} \int q_\alpha q_\beta q(\vc{p}, \vc{q}) \d^3 \vc{q} = \frac{\sum_{\delta t} q_\alpha q_\beta}{2 \delta t}.
\end{equation*}
Перепишем уравнение в виде
\begin{equation*}
	\frac{\partial f}{\partial t}  = - \frac{\partial s_\alpha}{\partial p_\alpha},
	\hspace{5 mm} \Leftrightarrow \hspace{5 mm} 
	\frac{\partial f}{\partial t}  + \div_{\vc{p}} \vc{s} = 0,
\end{equation*}
где величина $\vc{s}$ -- плотность потока в импульсном пространстве
\begin{equation*}
	s_\alpha = - \tilde{A}_\alpha f - \frac{\partial }{\partial p_\beta} (B_{\alpha \beta} f) = - A_\alpha f - B_{\alpha \beta} - B_{\alpha \beta} \frac{\partial f}{\partial p_\beta},
	\hspace{10 mm} 
	A_\alpha = \tilde{A}_\alpha + \frac{\partial B_{\alpha \beta}}{\partial p_\beta}.
\end{equation*}
Итого в общем виде уравнение Фоккера-Планка можем иметь вид
\begin{equation}
	\frac{\partial f}{\partial t} + \vc{v} \frac{\partial f}{\partial \vc{r}} + \vc{F} \frac{\partial f}{\partial \vc{p}} + \div_{\vc{p}} \vc{s} = 0.
\end{equation}
Считатать обычно проще $B_{\alpha \beta}$, а потом они друг через друга выражаются с учётом того, что в равновесии поток $\vc{s}^{(0)}$ зануляется.



\textbf{Тяжелые частицы}. Будем считать, что
\begin{equation*}
	f^{(0)} \sim \exp\left(
		- \frac{p^2}{2MT}
	\right).
\end{equation*}
Начнём с вычисления коэффициентов $B_{\alpha \beta}$:
\begin{equation*}
	B_{\alpha \beta} = B \delta_{\alpha \beta},
	\hspace{10 mm} 
	B = \frac{\sum_{\delta t} q^2}{6 \delta t},
\end{equation*}
и кинетическое уравнение перепишется в виде
\begin{equation*}
	\frac{\partial f(t, \vc{p})}{\partial t} = B \div_{\vc{p}} \left(
		\frac{\vc{p} f}{MT} + \nabla_{\vc{p}} f
	\right).
\end{equation*}
% уравнение Полуховского
% D = b T -- соотношение эйнштейна, из зануления потока в равновесии
Таким образом $B$ иммет смысл коэффициента диффузии в импульсном пространстве. 

Для определения величины $B$ выразим
\begin{equation*}
	\vc{q} = \Delta \vc{p}_b = \vc{p}_b - \bar{p}_b' = \vc{p}_a' - \vc{p}_a.
\end{equation*}
Считая, что $p_a = p_a'$, находим
\begin{equation*}
	q^2 = 2 p_a^2 - 2 p_a^2 \cos \theta = 2 p_a^2 (1-\cos \theta).
\end{equation*}
И тогда можем посчитать интеграл вида
\begin{equation*}
	B = \frac{1}{6} \frac{\sum_{\delta t} q^2}{\delta t} = \frac{1}{6} \int 2 p_a^2 (1-\cos \theta) f_a^{(0)} (\vc{p}_a) v_a \d \sigma \d^3 \vc{p}_\alpha.
\end{equation*}
Вводя $n = \int f^{(0)}(\vc{p}_\alpha) \d^3 \vc{p}_a$, находим
\begin{equation*}
	B = \frac{n_a}{3m} \langle p_a^3 \sigma_t(v_a)\rangle = \frac{m^2 n_a}{3} \langle v_a^3 \sigma_t \rangle,
\end{equation*}
где $m$ -- масса легкой частица, $\sigma_t$ -- транспортное сечение рассеяния легких частиц на тяжелых. 
% Тяжелые частицы диффузируют в легких


\textbf{Диффузия}. Добавив силу $\vc{F}$ можем найти
\begin{equation*}
	\frac{\partial f}{\partial t} + \frac{\partial }{\partial \vc{p}} \left(
		\left(\vc{F} - \frac{B \vc{p}}{MT}\right) f - B \frac{\partial f}{\partial \vc{p}} 
	\right) = 0.
\end{equation*}
На больших $\vc{p}$ функция распределения обращается в ноль. Для стационарного случая
\begin{equation*}
	\left(\vc{F} - \frac{B \vc{p}}{MT}\right) f - B \frac{\partial f}{\partial \vc{p}}  = \const = 0.
\end{equation*}
Функция распределения при внешней силе модифицируется к виду
\begin{equation*}
	f \sim \exp\left(
		- \frac{(\vc{p} - \vc{F} M T/B)^2}{2MT}
	\right) = \exp\left(
		- \frac{(\vc{p} - M \vc{u})^2}{2MT}
	\right),
	\hspace{10 mm} 
	\vc{u} = \frac{T}{B} \vc{F},
\end{equation*}
где $\vc{u}$ -- средняя потоковая скорость. Вообще подвижность $b$ определяется из $\vc{u} = b \vc{F}$, откуда находим $b = T/B$ и $D = b T = T^2/B$.


