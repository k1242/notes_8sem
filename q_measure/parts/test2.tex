\section{Сверхпроводимость}
% Валерий Владимирович Рязанов

Описываем макроскопическое квантовое явление сверхпроводимости единой волновой функции
\begin{equation*}
	\psi = |\psi| e^{i \theta},
	\hspace{5 mm} 
	|\psi|^2 = n_s.
\end{equation*}


% Открытие Камерлинга Оннеса 1911 года, в 1908 получил гелий
% в 1933 году открытие Майсснера и Оксенфельда, выталкивание магнитного поля

% 1934 год -- двухжидкостная модель, Гортер и Казимир
% 1935 -- фен теор Лондонов
% 1950 -- квантование магнитного потока в кольце Ф Лондон, 1961 эксп набл
% 1950-1953 -- А. Пиппард, теория нелокальности сверхпроводников и когерентности электронов в сверхпроводниках
% 1950 -- теория Гинзбурга-Ландау

% 1957 -- Абрикосов и scII рода и вихревое смешанное состояние сверпроводников

% 1956 -- феномен Купера
% 1950 -- изотопический эффект

% 1957 -- БКШ
% 1958 -- теория неоднородных сверхроводников Горькова

% 1962 -- предсказание Джозефсоном теннельных эффектов


% 1985 -- открытие высокотемпературных сверхпроводников



% Абрикосов, Гинзбург -- знакомые Рязанова




% Александр Лукин, сын брата Михаила Лукина



% Гинзбурговские теоретики, Максимов, не может быть сверхпроводимости выше 30К

Так вот, в квазиклассике
\begin{equation*}
	\vc{p} = \hbar \nabla \theta = m \vc{v} + e \vc{A},
\end{equation*}
тогда в рамках квантования Бора-Зоммерфельда
\begin{equation*}
	\oint p \d r = 
	\hbar \oint \nabla \theta \d r
	= 
	 \left(n + \frac{1}{2}\right) h,
\end{equation*}
откуда находим
\begin{equation*}
	\hbar \oint \nabla \theta \d r = 2 \pi \hbar n.
\end{equation*}
А вообще для сверхпроводника
\begin{equation*}
	\hbar \nabla \theta = 2 m v_s + 2 e A.
\end{equation*}
При малых полях и токах, в массивном односвязнос сверхпроводнике $\nabla \theta = 0$, поэтому 
\begin{equation*}
	\Lambda \vc{j}_s (r) = - A(r),
\end{equation*}
так приходим к лондоновской линейной электродинамике. Это случай $\lambda \gg \xi_0$, где $\xi_0$ -- размер куперовской пары. 


% эффект Аарнова-Бома, 1959 год


\textbf{Пиппардовские сверхпроводники}. Считая $\lambda \ll \xi_0$, придём к
\begin{equation*}
	\lambda \approx \sqrt[3]{\lambda_L^2 \xi_0}
\end{equation*}
Феноменологично наблюдалось
\begin{equation*}
	n_s(T) = n \left(
		1- \left[\tfrac{T}{T_c}\right]^4
	\right),
\end{equation*}
а в грязном сверхпроводнике
\begin{equation*}
	\lambda = \lambda_L \sqrt{\xi_0/l},
\end{equation*}
таким образом пиппардовские сверхпроводники становятся сверхпроводниками II рода при малых температурах.

% теорема Андерсона
% грязные сверхпроводники
% кинетическая индуктивность -- супериндуктивности
% 