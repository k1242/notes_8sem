% ПОСТАНОВКА ЗАДАЧИ
% Здесь надо максимально формально описать суть задачи, которую потребуется решить, так, чтобы можно было потом понять, в какой степени полученное в результате работы решение ей соответствует. Текст главы должен быть написан в стиле
% технического задания, т.е. содержать как описание задачи, так и некоторый набор
% требований к решению


% \startp
% \upar{Печь}
% На данный момент хочется провести калибровку по температуре для печи. Для этого смотрим на мощность флюоресценции на выходе из печи, как на функцию температуры. Сравниваем с исходной калибровкой и калибровкой, полученной по точкам плавления меди и алюминия. 

% \upar{Зееман}
% Варьируя параметры отстройки и мощности, добиться такой же загрузки МОЛ при сохранение количества атомов в МОЛ. Найти, как от параметров зеемана зависит $\sub{v}{crit}$, $\sub{v}{cool}$, $\sub{\dot{N}}{cool}$.


Целями данной работы являлись оптимизация количества атомов ${}^{169}$Tm в магнитооптической ловушке, работающей на длине волны 532 нм: увеличение длительности работы источника атомов (печи), повышение эффективности процесса замедления атомов. Проектирование двухмерной магнитооптическую ловушку в качестве источника атомов ${}^{169}$Tm.

В рамках работы были поставлены и решены следующие задачи 
\begin{enumerate*}
	\item С помощью спектроскопии атомарного пучка откалибровать температуру используемой в установке печи. Оптимизировать температуру печи.
	\item Построить модель замедления атомов в ЗЗ. Определить оптимальные параметры мощности лазерного луча, отстройки и значения токов в катушках ЗЗ. Измерить значение потока загрузки МОЛ с помощью ЗЗ.
	\item Построить модель формирования атомарного пучка в двухмерной магнитооптической ловушке. Определить оптимальные параметры мощности, отстройки, размеров пучка. Расчитать ожидаемое значение потока загрузки МОЛ с помощью 2D-МОЛ.
\end{enumerate*}


