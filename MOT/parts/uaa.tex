В области ультрахолодных атомов можно выделить две принципиальные области применений: создание сверхточных измерительных приборов и квантовая симуляция многочастичных систем. Создание квантовых симуляторов позволяет исследовать процессы, недоступные к аналитическому описанию или численному моделированию, в связи с экспоненциальным ростом сложности вычислений многочастичных задач в квантовой механике. Высокая точность измерений связана с возможностью работать с системами в их основном состоянии и наблюдению интерференционных явлений.и





% про измерения
Физика ультрахолодных атомов позволяет добиваться сверхточного измерения времени. Стандарт секунды определяется переходом в атоме ${}^{133}$Cs, реализация часов на основе лазерного охлаждения позволяет достигать точности порядка $10^{-16}$ \cite{schmittberger2020review, 799241}. На Sr и Yb получены точности порядка $10^{-18}$ \cite{schmittberger2020review, Bloom_2014}. 

Измерение гравитационных эффектов с помощью ультрахолодных атомов находит применение в фундаментальных исследованиях \cite{Tino_2021} измерение гравитационной постоянной $G$, исследование гравитации на малых масштабах, измерение параметра Этвёша; развиваются детекторы гравитационных волн на основе атомных интерферометров \cite{Dimopoulos_2009}. Измерение ускорения свободного падения может использоваться для практических задач, например поиска месторождений полезных ископаемых \cite{Tino_2021}.




% про симуляторы в общем
Основой квантовых симуляторов на ультрахолодных атомах является возможность в широком диапазоне настраивать различные параметры системы, такие как сила взаимодействия атомов \cite{bloch2012quantum}
, структура и глубина потенциала решетки \cite{lewenstein_ultracold_2007, gross_quantum_2017, tsyganok2023boseeinstein}, в которую помещаются охлажденные атомы, температуру и концентрацию. В зависимости от используемых атомв возможна симуляция ферми или бозе систем, а также их смесей \cite{yamamoto2012lattice}. С использованием объективов с большой числовой апертурой возможно получение разрешения в один узел оптической решётки \cite{Sherson_2010}, что позволяет напрямую наблюдать исследуемые явления на микроскопическом масштабе, увеличивая точность экспериментов и качественно меняя доступные к измерениям эффекты.  

В исследуемых с помощью квантовых симуляторов особенно можно выделить многочастичные задачи в оптических решётках \cite{bloch_many-body_2008}, формально реализующие модель ферми-хаббарда и бозе-хаббарда (с реализацией, например, перехода от сверхтекучести к моттовскому изолятору \cite{Greiner2002}). Экспериментально наблюдались вихри во вращающемся бозе-конденсате, формирование вихрями решётки \cite{Klaus_2022}.  Возможность настройки взаимодействия через резонанс Фешбаха позволяет исследовать переход от сверхтекучести БКШ, когда притяжение слабое и спаривание проявляется только в импульсном пространстве, к конденсату Бозе-Эйнштейна тесно связанных пар в реальном пространстве \cite{bloch_many-body_2008}.

Особый интерес представляет исследование условий, когда система не термализуется \cite{abanin_colloquium_2019}, так как это является важным шагом на пути к пониманию новых состояний материи, которые могут возникать в сильно неравновесных квантовых системах. Основным путём к термолизации является рассеивание энергии по доступным степеням свободы, что требует переноса между разными частями системы. Соответсвенно нарушение эргодичности происход в изолирующих системах. Примерами такого изолюирующего поведения, исследуемого с помощью квантовых симуляторов на ультрахолодных атомах, являются андерсоновская локализация \cite{roati_anderson_2008} и многочастичная локализация \cite{choi_exploring_2016}.
