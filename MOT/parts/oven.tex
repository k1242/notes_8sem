



\startp
\upar{Расход атомов}
В печи нагревается тулий до температуры $T$, вылетает из сопла диаметра $\sub{D}{ov}$, площади $\sub{S}{ov} = \pi \sub{D}{ov}^2/4$, длины $\sub{L}{ov}$. Полный поток атомов тулия \cite{tiecke_high-flux_2009} может быть определён, как 
\begin{equation}
	\sub{\Phi}{tot} = \frac{1}{4} \sub{n}{sat} \bar{v} \sub{S}{ov},
	\label{oven}
\end{equation}
где $\bar{v} = \sqrt{{8 \sub{k}{B} T}/{\pi m}} \approx 1.13 \alpha$ -- средняя тепловая скорость, $\sub{n}{sat} = \sub{P}{sat} / \sub{k}{B} T$ -- концентрация атомов в печи, зависимость $\sub{n}{sat} (T)$ для тулия приведена в \eqref{nTp}. Время работы печи тогда может найти, как $\sub{t}{life} = \sub{N}{tm} / \sub{\Phi}{tot}$, где $\sub{N}{Tm}$ -- изначальное количество атомов Tm в печи. В нашей установке $\sub{D}{ov} \approx 7\,\text{мм}$ и $\sub{t}{life} \sim 1$ месяца непрерывной работы.



\textbf{Концентрация}.  Считая, что в тигле достигается динамическое равновесие, концентрацию $n$ можно найти из зависимости давления насыщенных паров от температуры \cite{alcock_vapour_1983, svp} для атомов Tm:
\begin{equation}
	\sub{n}{sat}(T) = \frac{101325\,\text{Па}}{\kB T} 10^{
		8.882 - 12270\, T^{-1} - 0.9564 \log_{10} T
	}
	\label{nTp}
\end{equation}
где температура $T$ указана в Кельвинах.



\upar{Поток атомов на выходе}
В соответсвии с \cite{ramsey_molecular_1985}, вероятность вылететь из печи пропорциональна скорости $v$, поэтому максвелловское распределение модифицируется. Исходное распределение может быть записано в виде
\begin{equation}
	f(v_z, v_r) = \frac{4}{\alpha^4} v_r e^{-(v_r/\alpha)^2} v_z e^{-(v_z/\alpha)^2},
\end{equation}
где $\alpha = \sqrt{2 \sub{k}{B} T/m}$ -- наиболее вероятная скорость. Поток атомов долетающих до ЗЗ может быть найден \cite{tiecke_high-flux_2009}, с учётом того что
$v_r / v_z < \varphi$:
\begin{equation}
	f(v_z) = \frac{4}{\alpha^4} \int_{0}^{\infty}  v_r e^{-(v_r/\alpha)^2} v_z e^{-(v_z/\alpha)^2} \theta(\varphi - v_r/v_z) \d v_r = \frac{2}{\alpha^4} \varphi^2 v_z^3 e^{-(v_z/\alpha)^2}.
\end{equation}
Аналогично можем посмотреть на распределение в радиальном направлении
\begin{equation}
	f(v_r) = \frac{4}{\alpha^4} \int_{0}^{\infty}  v_r e^{-(v_r/\alpha)^2} v_z e^{-(v_z/\alpha)^2} \theta(\varphi - v_r/v_z) \d v_z = \frac{2}{\alpha^2} v_r e^{-(v_r/\alpha \varphi)^2}.
\end{equation}
Выражение для полного потока, влетающего в ЗЗ, может быть оценено, как
\begin{equation}
	\sub{\Phi}{sol} = \varphi^2 \cdot \sub{\Phi}{tot},
	\label{phisol}
\end{equation}
где для нашего ЗЗ $\varphi \sim 1/40$.  


















% \begin{equation}
% 	\sub{\Phi}{sol} = 
% 	\int_{0}^{\sub{\Omega}{sol}} d \Omega \frac{\cos \theta}{4\pi} \frac{1}{\mathcal N} 
% 	\int_{0}^{\infty} v^3 e^{-(v/\alpha)^2} \d v \approx
% 	\frac{\sub{\Omega}{sol}}{4 \pi} \frac{1}{\mathcal N} 
% 	\int_{0}^{\sub{v}{crit}} v^3 e^{-(v/\alpha)^2} \d v
% 	,
% 	\label{vcrit}
% \end{equation}
% где $\alpha = \sqrt{2 \sub{k}{B} T/m}$ -- наиболее вероятная скорость, $\mathcal N = \int v^2 e^{-(v/\alpha)^2} \d v = \frac{\sqrt{\pi}}{4} \alpha^3$ -- нормирующий множитель распределения по скоростям. 
