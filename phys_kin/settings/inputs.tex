% 11 18(202) 19
% 14

% ФП (129)

% document's head

\begin{center}
    \LARGE \textsc{Задание по курсу <<Физическая кинетика>>}
\end{center}

\hrule

\phantom{42}

\begin{flushright}
    \begin{tabular}{rr}
    % written by:
        % \textbf{Источник}: 
        % & \href{__ссылка__}{__название__} \\
        % & \\
        % \textbf{Лектор}: 
        % & _ФИО_ \\
        % & \\
        \textbf{Автор заметок}: 
        & Хоружий Кирилл \\
        & \\
    % date:
        \textbf{От}: &
        \textit{\today}\\
    \end{tabular}
\end{flushright}

\thispagestyle{empty}
\tableofcontents
\newpage




\section*{Проводимость Друде}


\textbf{Общефизическое рассмотрение}. 
Рассмотрим движение электронов под действием электрического поля
\begin{equation*}
	m (\ddot{x} + \gamma \dot{x}) = e E,
\end{equation*}
в установившемся режиме $\ddot{x} = 0$, $\gamma = 1/\tau$, где $\tau$ -- время столкновений. Так находим
\begin{equation*}
	v = \frac{e \tau}{m}E,
	\hspace{5 mm} 
	j = e n v = \frac{n e^2}{m} \tau E,
	\hspace{0.5cm} \Rightarrow \hspace{0.5cm}
	\sigD = \frac{n e^2}{m} \tau.
\end{equation*}

\textbf{$\tau$-приближение}. Воспользуемся $\tau$-приближением для $\sub{I}{st}$
\begin{equation*}
	\left(
		\frac{\partial }{\partial t} + v_i \frac{\partial }{\partial r_i} + e E_i \frac{\partial }{\partial p_i} 
	\right) f(r, p, t) = - \frac{f(r, p, t) - \feq (p)}{\tau}.
\end{equation*}
Рассматривая однородную стационарную задачу приходим к уравнению, вида
\begin{equation*}
	e E_i \frac{\partial }{\partial p_i} f(p) = - \frac{\delta f(p)}{\tau},
\end{equation*}
где $\delta f = f - \sub{f}{eq}$, а хотим найти $j = e \int \frac{\d p}{(2 \pi \hbar)^d} \delta f(p) v(p)$.

Рассматривая задачу в предположение о линейном отклике, находим
\begin{equation*}
	f(p) = \feq (p) + \chi_i (p) E_i + \ldots,
	\hspace{0.25cm} \Rightarrow \hspace{0.25cm}
	\chi_i (p) = - e \tau \frac{\partial }{\partial p_i} \feq (p),
\end{equation*}
и подставляя это в выражение для $j$, находим
\begin{equation*}
	j_i = - e^2 \tau E_s \int \frac{\d p}{(2 \pi \hbar)^d} \frac{p_i}{m} \frac{\partial }{\partial p_s} \feq (p) = \frac{e^2 \tau E_i}{m} \int \frac{\d p}{(2 \pi \hbar)^d} \feq (p),
\end{equation*}
где мы проинтегрировали по частям. Таким образом приходим к выражению для проводимости Друде
\begin{equation*}
	 j_i = \frac{\sub{n}{eq} e^2 \tau}{m} E_i = \sigD E_i,
	 \hspace{10 mm} 
	 \sigD = \frac{\sub{n}{eq} e^2}{m} \tau.
\end{equation*}

\textbf{Переменное поле}. Пусть теперь $E_i = E_i (t)$, тогда
\begin{equation*}
	\left(
		\frac{\partial }{\partial t} + e E_i (t) \frac{\partial }{\partial p_i} 
	\right) f(p, t) = - \frac{f(p, t) - \feq (p)}{\tau}.
\end{equation*}
Переходя к линейному отклику, находим
\begin{equation*}
		\frac{\partial }{\partial t} \delta f(p, t) + e E_i (t) \frac{\partial }{\partial p_i} 
	 \feq (p, t) = - \frac{\delta f(p, t)}{\tau},
\end{equation*}
или переходя к Фурье $\delta f(t) = \int \frac{\d \omega}{2\pi} e^{-i \omega t} \delta f(\omega)$, находим
\begin{equation*}
	 - i \omega \, \delta f(p, \omega) + e E_i (\omega) \frac{\partial }{\partial p_i}  \feq (p) = - \frac{\delta f(p, \omega)}{\tau},
\end{equation*}
тогда Фурье-образ поправки функции распределения будет равен
\begin{equation*}
	\delta f(p, \omega) = -  \frac{e E_i (\omega) }{1-i \omega \tau}  \frac{\partial \feq(p)}{\partial p_i} \tau .
\end{equation*}
Подставляя в выражение для тока $j$, получим
\begin{equation*}
	j_i (\omega) = \frac{\sigD}{1 - i \omega \tau} E_i (\omega) = \sigma(\omega) E_i (\omega),
\end{equation*}
c полюсом в нижней полуплоскости -- причинная функция Грина! Собственно, после обратного Фурье, находим
\begin{equation*}
	j_i = \int_{-\infty}^{t} \sigma(t-t') E_i (t) \d t',
	\hspace{0.5cm} \Rightarrow \hspace{0.5cm}
	\sigma(t) = \int \frac{\d \omega}{2\pi} e^{- i \omega t} \frac{\sigD}{1-i \omega \tau}  =  \sigD \cdot \frac{1}{\tau} e^{-t/\tau} \theta(t).
\end{equation*}


% T1
\section{Кинетическое уравнение Власова}


Начнём с уравнение Лиувилля, считая заданными $\vc{r}^N = (\vc{r}_1,\,  \ldots,\, \vc{r}_N)$ и $\vc{p}^N = (\vc{p}_1,\,  \ldots,\, \vc{p}_N)$
\begin{equation*}
	\dot{\vc{r}}_i = \frac{\partial H}{\partial \vc{p}_i},
	\hspace{10 mm} 
	\dot{\vc{p}}_i = -\frac{\partial H}{\partial \vc{r}_i},
\end{equation*}
где Гамильтониан запишется в виде
\begin{equation*}
	H = K(\vc{p}^N) + V(\vc{r}^N) + \Phi (\vc{r}^N),
	\hspace{10 mm} 
	K(\vc{p}^N) = \sum_{i=1}^{N} \frac{\vc{p}_i^2}{2m},
	\hspace{5 mm} 
	\Phi (\vc{r}^N) = \sum_{i=1}^{N} \varphi (\vc{r}_i).
\end{equation*}
Введём также функцию распределения $f^{[N]}(\vc{r}^N, \vc{p}^N, t)$ так чтобы $f^{[N]}(\vc{r}^N, \vc{p}^N, t) \d \vc{r}^N \d \vc{p}^N$ -- вероятность находиться в данной точке фазового пространства. Нормировка единичная. 

\textbf{Закон сохранения}.
Закон сохранения в дифференциальном виде запишется в виде
\begin{equation*}
	\frac{\partial \rho}{\partial t} + \div \vc{j} = 0,
\end{equation*}
где в нашем случае $\rho$ -- $f^{[N]}$, и $\vc{j} = \{f^{[N]} \dot{\vc{r}}_i, f^{[N]} \dot{\vc{p}}_i\}$, тогда
\begin{equation*}
	\frac{\partial f^{[N]}}{\partial t} + \sum_{i=1}^{N}\left( \frac{\partial }{\partial \vc{r}_i} \left[f^{[N]} \dot{\vc{r}}_i\right] + \frac{\partial }{\partial \vc{p}_i} \left[
				f^{[N]} \dot{\vc{p}}_i
			\right]\right) = \frac{d f^{[N]}}{d t} = 0, 
\end{equation*}
при подстановке уранений Гамильтона. 



\textbf{Редуцированная функция}. Редуцированная функция $f^{(n)}$ определяется как
\begin{equation*}
	f^{(n)} (\vc{r}^n,\, \vc{p}^n,\, t) = \frac{N!}{(N-n)!} \int f^{[N]} (\vc{r}^N,\, \vc{p}^N,\, t) \d \vc{r}^{(N-n)} \d \vc{p}^{(N-n)},
\end{equation*}
где $d \vc{r}^{(N-n)} = d \vc{r}_{n+1} \ldots \d \vc{r}_N$ и $d \vc{p}^{(N-n)} = d \vc{p}_{n+1} \ldots \d \vc{p}_N$. 

Работаем  приближение потенциального внешнего поля
\begin{equation*}
	\dot{\vc{p}}_i = \vc{X}_i + \sum_{j=1}^{N} \vc{F}_{ij} (\vc{r}_i,\, \vc{r}_j),
	\hspace{10 mm} 
	\vc{F}_{ii} = 0.
\end{equation*}
Тогда сохранение перепишется в виде
\begin{equation*}
	\frac{\partial f^{[N]}}{\partial t} + \sum_{i=1}^{N} \frac{\vc{p}_i}{m} \frac{\partial f^{[N]}}{\partial \vc{r}_i} + \sum_{i=1}^{N} \vc{X}_i \frac{\partial f^{[N]}}{\partial \vc{p}_i} = -
	\sum_{i=1}^{N} \sum_{j=1}^{N} \vc{F}_{ij} \frac{\partial f^{[N]}}{\partial \vc{p}_i}.
\end{equation*}
При редуцирование в силу ограниченности в фазовом пространстве, остаётся
\begin{equation*}
	\frac{\partial f^{(n)}}{\partial t} + \sum_{i=1}^{n} \frac{\vc{p}_i}{m} \frac{\partial f^{(n)}}{\partial \vc{r}_i} + \sum_{i=1}^{n} \vc{X}_i \frac{\partial f^{(n)}}{\partial \vc{p}_i} = - \sum_{i=1}^{n} \sum_{j=1}^{n} \vc{F}_{ij} \frac{\partial f^{(n)}}{\partial \dot{\vc{p}}_i} - \frac{N!}{(N-n)!} \sum_{i=1}^{n} \sum_{j=n+1}^{N} \int \vc{F}_{ij} \frac{\partial f^{[N]}}{\partial \vc{p}_i} \d \vc{r}^{(N-n)} \d \vc{p}^{(N-n)}.
\end{equation*}
С учетом симметричности функции распределения, последнее слагаемое можем переписать в виде
\begin{equation*}
	- \frac{N! (N-n)}{(N-n)!} \sum_{i=1}^{n} \int F_{i, n+1} \frac{\partial f^{[N]}}{\partial \vc{p}_i} \d \vc{r}^{(N-n-1)} \d \vc{p}^{(N-n-1)} \d \vc{r}_{n+1} \d \vc{p}_{n+1},
\end{equation*}
Так приходим к выражению, вида
\begin{equation*}
	\left(
		\frac{\partial }{\partial t} + \sum_{i=1}^{n} \frac{\vc{p}_i}{m} \frac{\partial }{\partial \vc{r}_i} + \sum_{i=1}^{n} \left[
			\vc{X}_i + \sum_{j=1}^{n} F_{ij}
		\right] \frac{\partial }{\partial \vc{p}_i} 
	\right) f^{(n)} = - \sum_{i=1}^{n} \int F_{i, n+1} \frac{\partial f^{(n+1)}}{\partial \vc{p}_i} \d \vc{r}_{n+1} \d \vc{p}_{n+1}.
\end{equation*}
Эта система уравнений называется цепочкой уравнений Боголюбова-Борна-Грина 
Обычно интерес представляют $n = 1, 2$, кстати $\int f^{(n)} \d \vc{r}^{n} \d \vc{p}^n = \frac{N!}{(N-n)!}$.

\subsection*{Одночастичный случай}

Для $n=1$ уравнение сведётся к
\begin{equation*}
	\left(
		\frac{\partial }{\partial t} + \frac{\vc{p}_1}{m} \frac{\partial }{\partial \vc{r}_1} + \vc{X}_1 \frac{\partial }{\partial \vc{p}_1} 
	\right) f^{(1)} (\vc{r}_1,\, \vc{p}_1,\, t) = - \int \vc{F}_{12} 
	\frac{\partial }{\partial \vc{p}_1} f^{(2)}(\vc{r}_1, \vc{p}_1, \vc{r}_2, \vc{p}_2, t) \d \vc{r}_2 \d \vc{p}_2.
\end{equation*}
В силу отсутствия корелляций между столкновениями попробуем сделать приближение
\begin{equation*}
 	f^{(2)}(\xi_1, \xi_2, t) = f^{(1)} (\xi_1^t)f^{(1)}(\xi_2^t).
\end{equation*}
Определяя
\begin{equation*}
	\tilde{\vc{F}} (\vc{r}, t) = \int \vc{F}_{12} (\vc{r}_1,\, \vc{r}_2) f^{(1)} (\vc{r}_2,\, \vc{p}_2,\, t) \d \vc{r}_2 \d \vc{p}_2,
\end{equation*}
приходим к бесстолкновительному уравнению Власова 
\begin{equation}
	\left(
		\frac{\partial }{\partial t} + \frac{\vc{p}_1}{m} \frac{\partial }{\partial \vc{r}_1}  + \left[
			\vc{X}_1 + \tilde{\vc{F}}
		\right] \frac{\partial }{\partial \vc{p}_1} 
	\right) f^{(1)} = 0.
\end{equation}
которое валидно при $n d^3 \gg 1$. 




\subsection*{Двухчастичный случай}

Для $n=2$:
\begin{equation*}
	\left(
		\frac{\partial }{\partial t} + \frac{\vc{p}_1}{m} \frac{\partial }{\partial \vc{r}_1} + \frac{\vc{p}_2}{m} \frac{\partial }{\partial \vc{r}_2} + \left[
			\vc{X}_1 + \vc{F}_{12}
		\right] \frac{\partial }{\partial \vc{p}_1} + 
		\left[
			\vc{X}_2 + \vc{F}_{21}
		\right] \frac{\partial }{\partial \vc{p}_2} 
	\right) f^{(2)} (\xi_1, \xi_2, t) =  - \int \left(
		\vc{F}_{13} \frac{\partial }{\partial \vc{p}_1} + \vc{F}_{23} \frac{\partial }{\partial \vc{p}_2} 
	\right) f^{(3)} \d \vc{r}_3 \d \vc{p}_3
\end{equation*}
Считая $n d^3 \ll 1$, можем игнорировать\footnote{
	Также будем считать, что $\vc{X}_i$ меняются слабо. 
}  трёхчастичные столкновения, тогда
\begin{equation*}
	\left(
		\frac{\vc{p}_1}{m} \frac{\partial }{\partial \vc{r}_1} + \frac{\vc{p}_2}{m} \frac{\partial }{\partial \vc{r}_2} + F_{12} \left[\frac{\partial }{\partial \vc{p}_1} - \frac{\partial }{\partial \vc{p}_2} \right]
	\right) f^{(2)} = 0.
\end{equation*}
Переходя к координатам, находим
\begin{equation*}
	\vc{F}_{12} \left(\frac{\partial }{\partial \vc{p}_1} - \frac{\partial }{\partial \vc{p}_2}  \right) f^{(2)} = - \left(
		\frac{\vc{p}_1}{m} \frac{\partial }{\partial \vc{r}_1} + \frac{\vc{p}_2}{m} \frac{\partial }{\partial \vc{r}_2} 
	\right) f^{(2)}.
\end{equation*}
Введём $\vc{r} = \vc{r}_1 - \vc{r}_2$, $\vc{R} = \frac{1}{2}(\vc{r}_1 + \vc{r}_2)$, тогда
\begin{equation*}
	\frac{\partial f^{(2)}}{\partial \vc{R}} \ll \frac{\partial f^{(2)}}{\partial \vc{r}}.
\end{equation*}

Возвращаемся к одночастичной функции, интегрируя находим
\begin{equation*}
	\left(
		\frac{\partial }{\partial t} + \frac{\vc{p}_1}{m} \frac{\partial }{\partial \vc{r}_1} + \vc{X}_1 \frac{\partial }{\partial \vc{p}_1} 
	\right) f^{(1)} (\vc{r}_1,\, \vc{p}_1,\, t) =  - \int \vc{F}_{12} \left(
		\frac{\partial }{\partial \vc{p}_1} - \frac{\partial }{\partial \vc{p}_2} 
	\right) f^{(2)} \d \xi_2 = \int 
		\left[
			\frac{\vc{p}_2}{m}-\frac{\vc{p}_2}{m}
		\right] \frac{\partial f^{(2)}}{\partial \vc{r}} \d \vc{r} \d \vc{p}_2,
\end{equation*}
продолжая с правой частью, вводя $\sub{\vc{v}}{отн} = \frac{\vc{p}_2}{m} - \frac{\vc{p}_1}{m}$ находим
\begin{equation*}
	\int \d p_2 \d^2 \sigma \d z  \sub{\vc{v}}{отн} \left(
		f^{(2)}(t_+) - f^{(2)}(t_-)
	\right).
\end{equation*}
После столкновения меняются импульсы частиц, тогда правую часть можем переписать в виде
\begin{equation}
	\int \d \vc{p}_2 \d^2 \sigma \sub{\vc{v}}{отн} \left(
		f^{(1)}(\vc{p}_2', \vc{r}, t) f^{(1)} (\vc{p}_1', \vc{r}, t) - f^{(1)} (\vc{p}_2, \vc{r}, t) f^{(1)}(\vc{p}_1, \vc{r}, t)
	\right), \text{\ \ -- интеграл столкновений}.
\end{equation}
Формально есть частицы прилетевшие и улетевшие.  К слову, $\d \vc{p}_1 \d \vc{p}_2 = \d \vc{p}_1' \d \vc{p}_2'$.



% T2
\section{x Колебания в плазме}
Рассмотрим частицы в силе Лоренца
\begin{equation*}
	\hat{F} = q \left(
		\vc{E} + \frac{1}{c} \left[ \vc{v} \times  \vc{B} \right]
	\right).
\end{equation*}
Запишем
\begin{equation*}
	\left(
		\frac{\partial }{\partial t} + \vc{v} \frac{\partial }{\partial \vc{r}} - e \left[
			\vc{E} + \frac{1}{c} \left[\vc{v} \times  \vc{B}\right]
		\right] \frac{\partial }{\partial \vc{p}} 
	\right) f = 0.
\end{equation*}
Учтём, что $M \gg m$, тогда $f_i = f_{io}$  и $f = f_0 + \delta f$. В линейном отклике
\begin{equation*}
	\rho = - e \int \delta f \d \Gamma,
	\hspace{10 mm} 
	\vc{j} = - e \int \vc{v} \, \delta f  \d \Gamma,
\end{equation*}
где уже учли отсутствие вклада равновесных слагаемых. Равновесная функция распределения $f_0 = f_0 (\varepsilon(\vc{p}))$, подставляя, находим
\begin{equation*}
	\frac{\partial \delta f}{\partial t} + \vc{v} \frac{\partial \delta f}{\partial \vc{r}} - e\left(
		\vc{E} + \frac{1}{c}  \left[\vc{v} \times  \vc{B}\right]
	\right) \frac{\partial f_0}{\partial \vc{p}} = 0,
	\hspace{10 mm} 
	\frac{\partial f_0}{\partial \vc{p}} = \vc{v} \frac{\partial f_0}{\partial \varepsilon}.
\end{equation*}
Итого остаётся 
\begin{equation*}
	\frac{\partial \delta f}{\partial t}  + \vc{v} \frac{\partial \delta f}{\partial \vc{r}} = e (\vc{v} \cdot \vc{E}) \frac{\partial f_0}{\partial \varepsilon} - \frac{\delta f}{\tau},
\end{equation*}
где добавление $- \frac{\delta f}{\tau}$ приводит к причинности дальнейшего выражения $+ i \delta = + i / \tau$. 

Рассмотрим $\vc{E} = E_{k, \omega}e^{i\left(\vc{k} \vc{r} - \omega t\right)}$, и тогда $\delta f = \delta f_{k, \omega} e^{i\left(\vc{k} \vc{r} - \omega t\right)}$, подставляя находим 
\begin{equation*}
	(\omega - \vc{k} \vc{v}) \delta f_{k \omega} = i e (\vc{v} \cdot \vc{E}_{k \omega}) \frac{\partial f_0}{\partial \varepsilon},
\end{equation*}
и выражение на Фурье образ первой поправки
\begin{equation*}
	\delta f_{k \omega} (\vc{p}) = \frac{i e (\vc{v} \cdot \vc{E}_{k, \omega})}{\omega - \vc{k} \cdot \vc{v} + i \delta} \frac{\partial f_0}{\partial \varepsilon}.
\end{equation*}
Вспомним, что $\vc{D} = \vc{E} + 4 \pi \vc{P}$, тогда
\begin{equation*}
	- e \int \frac{i e \vc{v} (\vc{v} \cdot \vc{E}_{k \omega})}{\omega - \vc{k} \vc{v} + i \delta} \frac{\partial f_0}{\partial \varepsilon} \d \Gamma = j_{k \omega} = - i \omega P_{k \omega},
\end{equation*}
откуда находим поляризацию
\begin{equation*}
	P_{\alpha, k \omega} = \frac{e^2}{\omega} E_\beta \int \frac{v_\alpha v_\beta}{\omega - \vc{k} \cdot \vc{v}} \frac{\partial f_0}{\partial \varepsilon} \d \Gamma = \chi_{\alpha \beta} E_\beta.
\end{equation*}
Для трёхмерного случая итого находим
\begin{equation*}
	D_\alpha = E_\alpha + 4 \pi P_\alpha = \varepsilon_{\alpha \beta} E_\beta,
	\hspace{0.5cm} \Rightarrow \hspace{0.5cm}
	\varepsilon_{\alpha \beta} (\omega, \vc{k}) = \delta_{\alpha \beta} + \frac{4 \pi e^2}{\omega} \int \frac{v_\alpha v_\beta}{\omega - \vc{k} \cdot \vc{v}}	\frac{\partial f_0}{\partial \varepsilon} \d \Gamma.
\end{equation*}
Перейдём к переменным $\hat{\vc{v}} = \vc{v}/v$, $\hat{\vc{k}} = \vc{k} / k$, переписываем интеграл в виде
\begin{equation*}
	\varepsilon_{\alpha \beta} = \delta_{\alpha \beta} + \frac{4 \pi e^2}{\omega^2} \int \d \Gamma\ s v^2 \frac{\partial f_0}{\partial \varepsilon} \int \frac{\d \Omega}{4 \pi} \frac{\hat{v}_\alpha \hat{v}_\beta}{s - \hat{\vc{k}} \cdot \hat{\vc{v}} + i \delta}	
\end{equation*}
где $s = \omega / k v$. Итого усредняя находим\footnote{
	см. Бурмистров, считается $A(s)$, $B(s)$.
} 
\begin{equation*}
	\int \frac{\d \Omega}{4 \pi} \frac{\hat{v}_\alpha \hat{v}_\beta}{s - \hat{\vc{k}} \cdot \hat{\vc{v}} + i \delta} = 
	A(s) \delta_{\alpha \beta} + B(s) \hat{k}_\alpha \hat{k}_\beta f,
\end{equation*}
Итого находим выражение в виде
\begin{equation*}
	\varepsilon_{\alpha \beta}(\omega, \vc{k}) = \varepsilon_l \frac{k_\alpha k_\beta}{k^2} + \varepsilon_t \left(
		\delta_{\alpha \beta} - \frac{k_\alpha k_\beta}{k^2}
	\right).
\end{equation*}
Считая $\vc{E} = \vc{E}_e + \vc{E}_t$, где $\vc{E}_l = (\vc{E} \cdot \vc{k}) \vc{k} / k^2$ и $\vc{E}_t = \vc{E}- \vc{E}_l$, найдём
\begin{equation*}
	D_\alpha = \varepsilon_{\alpha \beta} E_\beta = D_{l\alpha} + D_{t \alpha}, 
	\hspace{10 mm} 
	D_{l \alpha} = \varepsilon_l E_{l \alpha},
	\hspace{5 mm} 
	D_{t \alpha} = \varepsilon_t D_{t \alpha}.
\end{equation*}
Рассмотрим теперь только $\vc{E}_l \propto \vc{k}$, тогда
\begin{equation*}
	\rot \vc{E} = i \left[
		\vc{k} \times  \vc{E}_l
	\right] = 0 = - \frac{i \omega}{c} \vc{B},
	\hspace{0.5cm} \Rightarrow \hspace{0.5cm}
	\rot \vc{B} = 0 = \frac{i \omega}{c} \vc{D},
	\hspace{0.5cm} \Rightarrow \hspace{0.5cm}
	\vc{D}_l = 0.
\end{equation*}
Таким образом $\varepsilon_l (\omega, \vc{k}) = 0$  задаёт дисперсию продольных плазменных колебаний. Для поперечных плазменных колебаний уравнение примет вид
\begin{equation*}
	k^2 = \frac{\omega^2}{c^2} \varepsilon_t (\omega, \vc{k}).
\end{equation*}




\subsection*{2D}

Специфично для двухмерного случая
\begin{equation*}
	\left(
		\frac{\partial }{\partial t} + \vc{v} \frac{\partial }{\partial \vc{r}} + \vc{F} \frac{\partial }{\partial \vc{p}} 
	\right) f = 0,
\end{equation*}
поле и поправка
\begin{equation*}
	\vc{E} = \vc{E}_{k \omega} e^{i(\vc{k} \vc{r} - \omega t)},
	\hspace{5 mm} 
	\delta f_{\vc{k}, \omega} = \frac{i e (\vc{v} \cdot \vc{E}_{k \omega}(z=0))}{\omega - \vc{k} \cdot \vc{v} + i \delta} \frac{\partial f_0}{\partial \varepsilon}.
\end{equation*}
Рассмотрим выражение для $\rho$, которая имеет принципиально двухмерный характер
\begin{equation*}
	\rho_{\omega \vc{k}} = - i e^2 \int \frac{i e (\vc{v} \cdot \vc{E}_{k \omega}(z=0))}{\omega - \vc{k} \cdot \vc{v} + i \delta} \frac{\partial f_0}{\partial \varepsilon} \d \Gamma.
\end{equation*}
Отдельно найдём
\begin{equation*}
	I(\omega, \vc{k}) = \int \frac{\vc{v}}{\omega - \vc{k} \vc{v} + i \delta} \frac{\partial f_0}{\partial \varepsilon}  \d \Gamma = \frac{\vc{k}}{k^2} \int \frac{\partial f_0}{\partial \varepsilon} A(s) \d \Gamma,
	\hspace{10 mm} 
	A(s) = \int_{0}^{2\pi} \frac{\d \varphi}{2\pi} \frac{\cos \varphi-s+s}{s- \cos \varphi + i \delta}.
\end{equation*}
Сделаем замену
\begin{equation*}
	x = \tg \frac{\varphi}{2}, \ \ \d \varphi = \frac{2 \d x}{1+x^2},
	\hspace{0.5cm} \Rightarrow \hspace{0.5cm}
	A(s) = \left\{\begin{aligned}
	    &-1 + \tfrac{s}{\sqrt{s^2-1}}, &s>1 \\
	    &-1  - \tfrac{i s}{\sqrt{1-s^2}}, &s>1.
	\end{aligned}\right.
\end{equation*}
где мнимая часть связана с затуханием Ландау. Подставляя в плотность
\begin{equation*}
	\rho_{\omega k} = - \frac{i e^2}{k^2} \left(
		\vc{k} \cdot \vc{E}_{\omega k}
	\right) \int \frac{\partial f_0}{\partial \varepsilon} A(s) \d \Gamma.
\end{equation*}
Расписывая уравнения Максвелла	
\begin{equation*}
	\frac{\partial E_x}{\partial x}  + \frac{\partial E_y}{\partial y} + \frac{\partial E_z}{\partial z} = 4 \pi \rho (x, y) \delta(z),
	\hspace{10 mm} 
	\vc{E} = - \nabla \varphi,
	\hspace{5 mm} 
	\varphi = \varphi_{k \omega} (z) e^{i (\vc{k} \vc{r} -  \omega t)}.
\end{equation*}






% T3
\section{Кондактанс}
Lorem ipsum dolor sit amet, consectetur adipisicing elit, sed do eiusmod
tempor incididunt ut labore et dolore magna aliqua. Ut enim ad minim veniam,
quis nostrud exercitation ullamco laboris nisi ut aliquip ex ea commodo
consequat. Duis aute irure dolor in reprehenderit in voluptate velit esse
cillum dolore eu fugiat nulla pariatur. Excepteur sint occaecat cupidatat non
proident, sunt in culpa qui officia deserunt mollit anim id est laborum.

% T4
\section{Упругое рассеяние электронов на примесях}





Рассеяние электронов на примесях
\begin{equation*}
	\left(
		\frac{\partial }{\partial t} + \vc{v} \frac{\partial }{\partial \vc{r}} +  \dot{\vc{k}} \frac{\partial }{\partial \vc{k}} 
	\right) f = I_{\vc{k}} = - \frac{\delta f}{\tau}.
\end{equation*}
В данном случае для линеаризованного кинетического уравнения $\tau$-приближение является точным, где $\delta f = f_{k}-f_0$. 


Поработаем с самим интегралом столкновений
\begin{equation*}
	I_k = \frac{2\pi}{\hbar} \sum_{\vc{k}'} \left(
		|\bk{\vc{k}'}[\sub{U}{пол}]{\vc{k}}|^2 \delta (\varepsilon_{k}-\varepsilon_{k'}) \left[
			f_{k'}(1-f_{k}) - f_k (1-f_{k'})
		\right]
	\right),
\end{equation*}
где $f_{k'}(1-f_{k}) - f_k (1-f_{k'}) = f_{k'}-f_k$. Для матричного элемента
\begin{equation*}
	\sub{U}{пол} (\vc{r}) = \sum_{j=1}^{N} U(\vc{r}-\vc{R}_j),
	\hspace{10 mm} 
	\langle \vc{k}| = \frac{1}{\sqrt{V}} e^{i \vc{k} \vc{r}}.
\end{equation*}
Тогда для матричного эдлемента находим
\begin{equation*}
	|\bk{\vc{k}'}[\sub{U}{пол}]{\vc{k}}|^2 = \frac{1}{V^2} |\tilde{U}(\vc{k}-\vc{k}')|^2 \cdot \bigg|
		\sum_{j=1}^{N} e^{i(\vc{k}-\vc{k}')\vc{R}_j}
	\bigg|^2,
	\hspace{10 mm} 
	\tilde{U} (\vc{q}) = \int e^{i \vc{q} \vc{r}} U(\vc{r}) \d^3 \vc{r}.
\end{equation*}
Усредняя по случайному положению примесей
\begin{equation*}
	\left\langle \bigg| 
		\sum_{j=1}^{N} e^{i (\vc{k}-\vc{k}') R_j}
	\bigg| \right\rangle_{\text{прим}} = N + N (N-1) \delta_{\vc{k}, \vc{k}'}.
\end{equation*}
Для матричного элемента получили выражение
\begin{equation*}
	|\bk{\vc{k}'}[\sub{U}{пол}]{\vc{k}}|^2 = \frac{N}{V^2} |\tilde{U}(\vc{q})|^2 + \frac{N(N-1)}{V^2} |\tilde{U}(0)|^2 \delta_{\vc{k}, \vc{k}'},
	\hspace{10 mm} 
	\vc{q} = \vc{k}-\vc{k}'.
\end{equation*}
Итого для интеграла столкновений получаем выражение после замены $\sum_{\vc{k}} \to \int \frac{V \d^3 k}{(2 \pi)^3}$
\begin{equation*}
	I_k (f) = \frac{2\pi n}{\hbar} \int \frac{d^3 \vc{k}'}{(2\pi)^3} |\tilde{U} (\vc{q})|^2 \delta(\varepsilon_k - \varepsilon_{\vc{k}}) \cdot \left(
		\delta f_{k'} - \delta f_{k}
	\right),
\end{equation*}
где уже линеаризовали выражение. Здесь $n$ -- примесное.



Рассмотрим стационарный однородный случай, когда $\hbar \dot{\vc{k}} = - e \vc{E}$, где поле считаем малой поправкой, тогда
\begin{equation*}
	\dot{\vc{k}} \frac{\partial }{\partial \vc{k}}  = - e \vc{E} \cdot \vc{v} \frac{\partial }{\partial \varepsilon},
	\hspace{10 mm} 
	\delta f_k \overset{\mathrm{def}}{=}  \tau(\varepsilon) \left(e \vc{E} \cdot \vc{v}_{\vc{k}} \right) \frac{\partial f_0}{\partial \varepsilon},
\end{equation*}
то есть ищем решение в $\tau$-приближение. Получается уравнение
\begin{equation*}
	- (\vc{E} \cdot \vc{v}) \frac{\partial f_0}{\partial \varepsilon} = I_k = \frac{2\pi n}{\hbar} e \vc{E} \int \frac{d^3 \vc{k}'}{(2\pi)^3} |\tilde{U}(\vc{q})|^2 \delta(\varepsilon_k - \varepsilon_{k'}) \left(
		\tau (\varepsilon') \vc{v}' \frac{\partial f_0}{\partial \varepsilon} \bigg|_{\varepsilon'} 
		-
		\tau (\varepsilon) \vc{v} \frac{\partial f_0}{\partial \varepsilon} \bigg|_{\varepsilon}
	\right).
\end{equation*}
Сокращая $\partial_\varepsilon f_0$ и всё лишнее, находим
\begin{equation*}
	\vc{v} = \frac{\hbar \vc{k}}{m} = \frac{n \tau(\varepsilon[\vc{k}])}{4 \pi^2 \hbar} \int_{0}^{\infty} dk' \ (k')^2 \int d \Omega_{k'} |\tilde{U}(\vc{q})|^2 \cdot \frac{\delta(k-k')}{\hbar^2 k/m} \frac{\hbar}{m} (\vc{k}-\vc{k}').
\end{equation*}
Остаётся выражение
\begin{equation}
	\frac{1}{\tau(\varepsilon)} = \frac{m k n}{4 \pi^2 \hbar^3} \int d \Omega_k \ |\tilde{U}(\vc{q})|^2 (1 - \hat{\vc{k}} \cdot \hat{\vc{k}}'),
	\hspace{10 mm} 
	\hat{\vc{k}} = \frac{\vc{k}}{k}.
\end{equation}



\textbf{Дифференциальное сечение рассеяния}. Найдём выражение
\begin{equation*}
	\frac{\d \sigma}{\d \Omega} = \left(\frac{m}{2\pi \hbar^2}\right)^2 |\tilde{U}(\vc{q})|^2,
	\hspace{10 mm} 
	\vc{q} = \vc{k}-\vc{k}',
	\hspace{5 mm} 
	q^2 = 4 k^2 \sin^2 \left(\frac{\theta}{2}\right).
\end{equation*}
И интеграл столкновений перепишется в виде
\begin{equation}
	\frac{1}{\tau(\varepsilon)} = n v \int \frac{d \sigma}{d \Omega} (1-\cos \theta) \d \Omega = n v \sub{\sigma}{tr},
\end{equation}
где возникло новое $\sub{\sigma}{tr}$ с подавленным рассеянием на малых углах. 


Вспоминая формулу Друде, находим
\begin{equation*}
	\vc{j} = \sigma_D \vc{E},
	\hspace{10 mm} 
	\sigma_D = \frac{e^2 n_0 \sub{\tau}{tr}}{m},
\end{equation*}
где входит именно $\tau_{\text{tr}}$.





\textbf{Фурье-образ}. Для экранированного кулоновского потенциала 
\begin{equation*}
	U(r) = - e^{-r/\lambda} \frac{Z e^2}{r},
	\hspace{10 mm} 
	\tilde{U}(\vc{q}) = \int U(\vc{r}) e^{-i \vc{q} \vc{r}} \d V = \frac{4 \pi Z e^2}{q^2 + \lambda^{-2}}.
\end{equation*}
Для дифференциального сечения рассеяния находим
\begin{equation*}
	\frac{d \sigma}{d \Omega} = \frac{m^2}{4 \pi^2 \hbar^2} \left(
		\frac{4\pi Z e^2}{q^2+\lambda^{-2}}
	\right)^2 = \left(
		\frac{Z e^2}{4 E_F} \frac{1}{\sin^2 \frac{\theta}{2} + (2 k_F \lambda)^{-2}}
	\right)^2.
\end{equation*}
где $q = 2 k_F \sin \frac{\theta}{2}$.
Полное сечение рассеяния тогда получается
\begin{equation*}
	\sigma = \int \d \sigma = \int_{0}^{\pi} \left(
		\frac{Z e^2}{4 E_F} \frac{1}{\sin^2 \frac{\theta}{2} + (2 k_F \lambda)^{-2}}
	\right)^2 = \frac{2\pi Z^2 e^4}{4 E_F^2} \int_{0}^{2}  \frac{\d u}{(u + \frac{1}{2}(k_F \lambda)^{-2})^2},
\end{equation*}
где $u = 1 - \cos \theta$.  Итого, введя $\zeta \overset{\mathrm{def}}{=}  \frac{4}{\pi} (k_F \lambda)^2$, находим
\begin{equation*}
	\sigma = \frac{\pi Z^2 e^4}{2 E_F^2} \frac{(\pi \zeta)^2/2}{1 + \pi \zeta}.
\end{equation*}
Для транспортного $\sub{\sigma}{tr}$, находим
\begin{equation*}
	\sub{\sigma}{tr} = \frac{2\pi Z^2 e^4}{4 E_F^2} \int_{0}^{2}  \frac{u \d u}{(u + \frac{1}{2}(k_F \lambda)^{-2})^2} = \frac{\pi Z^2 e^4}{2 E_F^2} \left(
		\ln(1+\pi \zeta) - \frac{\pi \zeta}{1+\pi \zeta}
	\right).
\end{equation*}
Для проводимости $\rho$ можем найти
\begin{equation*}
	\rho = \frac{m}{n e^2 \sub{\tau}{tr}} = \frac{m n v_F}{n_0 e^2} \frac{\pi Z^2 e^4}{2 E_F^2} \zeta^3 F(\zeta),
	\hspace{10 mm} 
	F(\zeta) = \frac{1}{\zeta^3} \left(
		\ln(1+\pi \zeta) - \frac{\pi \zeta}{1+\pi \zeta}
	\right).
\end{equation*}
Итого, находим
\begin{equation}
	\rho = Z^2 R_q \sub{a}{B} \frac{n}{n_0} F(\zeta) \cdot \left[
		\frac{e^2}{2 \pi \hbar} \frac{m e^2}{\hbar^2} \frac{p_F}{e^2} \frac{\pi e^4}{p_F^2/2m^2} \frac{64 k_F^6 \lambda^6}{\pi^3}
	\right] = Z^2 R_q \sub{a}{B} \frac{n}{n_0} F(\zeta).
\end{equation}
где подставили $\lambda^2 = \frac{\pi \sub{a}{B}}{4 k_F}$.


% T5
\section{Рассеяние электронов на фононах}



\textbf{Эффект Иоффе-Регеля}. На высоких температурах $r^2 \sim T$ для ионов, тогда
\begin{equation*}
	\tau = \frac{1}{\sub{n}{ion} v \sigma} \sim \frac{1}{T},
	\hspace{10 mm} 
	\rho = \frac{m}{n e^2 \tau} \sim T.
\end{equation*}
Для $\tau v_F \sim \lambda_F$, можем записать с учётом $n \sim k_F^3$
\begin{equation*}
	\rho = \frac{m v_F}{n e^2 \tau v_F} = \frac{m v_F}{n e^2 \lambda_F} \sim \frac{\hbar}{k_F e^2},
\end{equation*}
что называется пределом Иоффе-Регеля, которые неплохо работает для легированных полупроводников. 


\textbf{Испускание фононов}. И снова запишем столкновительный интеграл в терминах приход-уход:
\begin{equation*}
	I_p = \sum_{\vc{p}'} w_{\vc{p} \vc{p}'} n_{\vc{p}'} (1-n_{\vc{p}}) - \sum_{\vc{p}'}  w_{\vc{p}' \vc{p}} n_{\vc{p}} (1-n_{\vc{p}'}).
\end{equation*}
Рассматриваем однородную ситуацию, тогда 
\begin{equation*}
	\dot{\vc{p}} \frac{\partial n}{\partial \vc{p}} = 
	- e (\vc{E} \cdot \vc{v}) \frac{\partial n_0}{\partial \varepsilon}
	= \sub{I}{ст}.
\end{equation*}
Учитвая что $w_q \sim q$, можем расписать
\begin{align*}
	\sub{I}{ст} = &\frac{2\pi}{\hbar V} \sum_{\vc{q}} \left(
		w_{\vc{q}} (1 + N_{\vc{q}}) n_{\vc{p} + \hbar \vc{q}} (1-n_{\vc{p}}) \delta (\varepsilon_p - \varepsilon_{\vc{p} + \hbar \vc{q}} + \hbar \omega_q) 
		+ w_q N_q n_{\vc{p}-\hbar \vc{q}} (1-n_{\vc{p}}) \delta(\varepsilon_p - \varepsilon_{\vc{p}-\hbar \vc{q}}- \hbar \omega_q) 
	\right) - \\
	- & \frac{2\pi}{\hbar V} \sum_{\vc{q}}
	\left(
		w_{\vc{q}} (1+N_{\vc{q}}) n_q (1-n_{\vc{p}-\hbar \vc{q}}) \delta (\varepsilon_p - \varepsilon_{\vc{p} - \hbar \vc{q}} - \hbar \omega_q) + w_q N_q (1-n_{\vc{p}+\hbar \vc{q}}) \delta(\varepsilon_p - \varepsilon_{\vc{p}+\hbar \vc{q}} + \hbar \omega_q)
	\right).
\end{align*}
% вставить формулу с фото
Будем считать, что фононы равновесные
\begin{equation*}
	N_q = N_{q}^0 = \frac{1}{e^{\hbar \omega_q/T}-1},
	\hspace{0.5cm} \Rightarrow \hspace{0.5cm}
	\frac{1+N_q}{N_q} = e^{\hbar \omega_q/T}.
\end{equation*}
Для электронов
\begin{equation*}
	n_p^0 = \frac{1}{e^{(\varepsilon_p-\mu)/T}+1},
	\hspace{0.5cm} \Rightarrow \hspace{0.5cm}
	\frac{1-n_p^0}{n_p^0} = e^{(\varepsilon_p-\mu)/T}.
\end{equation*}
Преобразуем выражение из квадратных скобок *
\begin{equation}
	(1+N_q)(1-n_p)(1-n_{p+\hbar q}) \left(
		\frac{n_{\vc{p}+\hbar \vc{q}}}{1-n_{\vc{p}+\hbar \vc{q}}}
		- \frac{N_q}{1+N_q} \frac{n_p}{1-n_p}
	\right),
	\label{eqsq}
\end{equation}
которое очевидно зануляется для равновесных функций. 


Решение будем искать в виде
\begin{equation*}
	n_p = n_{p}^0 + \delta n_p = n_p^0 - \frac{\partial n^0_p}{\partial \varepsilon_p} \Phi_p = n_p^0 + \frac{n^0(\varepsilon_p)(1-n^0(\varepsilon_p))}{T} \Phi_p.
\end{equation*}
Возвращаясь к \eqref{eqsq}, получаем линеаризуя
\begin{align*}
	&(1+N_q)(1-n_p^0) (1-n_{\vc{p} + \hbar \vc{q}}^0) \left[
		\frac{\delta n_{\vc{p} + \hbar \vc{q}}}{(1-n_{\vc{p} + \hbar \vc{q}}^0)^2} - \frac{N_q}{1+N_q} \frac{\delta n_p}{(1-n_p^0)^2}
	\right] = \\
	= & +\frac{1}{T} (1+N_q)(1-n_p^0) n_{\vc{p} + \hbar \vc{q}}^0 \left[
		\Phi_{\vc{p} + \hbar \vc{q}}-\Phi_p
	\right] = \\
	= & - \frac{1}{T} (1+N_q) N_q (n_{\vc{p} + \hbar \vc{q}}^0-n_p^0) \left[
		\Phi_{\vc{p} + \hbar \vc{q}}-\Phi_p
	\right]
	.
\end{align*}
Аналогично преобразуется второе слагаемое в *, откуда находим линеаризованный интеграл столкновений:
\begin{align*}
	\sub{I}{ст}(\Phi_p) &= - \frac{2\pi}{\hbar V} \sum_{\vc{q}} w_q \frac{(1+N_q)N_q (n_{\vc{p}+\hbar \vc{q}}^0-n_p^0)}{T} \left[
		\Phi_{\vc{p}+\hbar \vc{q}} - \Phi_{\vc{p}}
	\right] \times \left[
		\delta(\varepsilon_p - \varepsilon_{\vc{p}+\hbar \vc{q}} + \hbar \omega_q) - \delta(\varepsilon_p - \varepsilon_{\vc{p}+\hbar \vc{q}} - \hbar \omega_q)
	\right]
\end{align*}
Выделим физ. смысл в слагаемых
\begin{align*}
		\sub{I}{ст}(\Phi_p) = - \frac{2 \pi}{\hbar V} \sum_{\vc{q}} w_q \frac{(1+N_q) N_q}{T} \bigg(
		&\left[
			n^0(\varepsilon_p + \hbar \omega_q) - n^0(\varepsilon_p)
		\right] 
		\delta(\varepsilon_p - \varepsilon_{\vc{p}+\hbar \vc{q}} + \hbar \omega_q) 
		- \\ - &
		\left[
			n^0(\varepsilon_p - \hbar \omega_q) - n^0(\varepsilon_p)
		\right] \delta(\varepsilon_p - \varepsilon_{\vc{p}+\hbar \vc{q}} - \hbar \omega_q)
	\bigg) \left[
		\Phi_{\vc{p}+\hbar \vc{q}} - \Phi_{\vc{p}}
	\right]
	.
\end{align*}
Учтём, что мы живём вблизи поверхности Ферми, тогда $\hbar \omega_q$ мало по сравнению с $\varepsilon_p$, приходим к выражению
\begin{equation*}
	\sub{I}{ст}(\Phi_p) = - \frac{\partial n^0}{\partial \varepsilon} \frac{2\pi}{\hbar V} \sum_q w_q \frac{2 \hbar \omega_q (1+N_q) N_q}{T} \delta(\varepsilon_{\vc{p}+\hbar \vc{q}}- \varepsilon_{\vc{p}}) \left[
		\Phi_{\vc{p}+\hbar \vc{q}} - \Phi_{\vc{p}}
	\right].
\end{equation*}
Аргумент $\delta$-функции можем расписать в виде
\begin{equation*}
	\varepsilon_p - \varepsilon_{\vc{p}+\hbar \vc{q}} \pm \hbar \omega_q = 
	\frac{2 p \hbar q \cos \theta}{2 m} + \frac{\hbar^2 q^2}{2m} \pm \hbar c_L q = \frac{\hbar p q}{m}\left(
		\cos \theta + \frac{\hbar q}{2p} \pm \frac{m c_L}{p}
	\right),
\end{equation*}
где $c_L \ll v_F$, поэтому можем опустить последнее слагаемое. 



\textbf{Кинетическое уравнение}. Итого, будем решать кинетическое уравнение на $\Phi_{\vc{p}}$ вида
\begin{equation}
	- e (\vc{E} \cdot \vc{v}) \frac{\partial n^0}{\partial \varepsilon} = \sub{I}{ст}(\Phi_p) = - \frac{\partial n_0}{\partial \varepsilon} \frac{2\pi}{\hbar V} \sum_q w_q \frac{2 \hbar \omega_q (1+N_q) N_q}{T} \delta(\varepsilon_{\vc{p}+\hbar \vc{q}}- \varepsilon_{\vc{p}}) \left[
		\Phi_{\vc{p}+\hbar \vc{q}} - \Phi_{\vc{p}}
	\right].
\end{equation}
Решение аналогично будем искать в виде $\Phi_p = - e (\vc{E} \cdot \vc{v}) \sub{\tau}{tr} (\varepsilon_p)$, что соответствует $\tau$-приближению: $\sub{I}{ст} = - \delta n_p / \tau$. Таким образом остаётся найти $\sub{\tau}{tr}$, и найти остальные величины по формуле Друде.  Выражая из двух уравнений $(\vc{E} \cdot \vc{v})$, находим
% в данном случае метод моментов сводится к одному слагаемому
\begin{equation*}
	(\vc{E} \cdot \vc{v}) = - \frac{4 \pi}{\hbar V} \sum_{\vc{q}} w_q \frac{ \hbar \omega_q (1+N_q) N_q}{T} \delta(\varepsilon_{\vc{p}+\hbar \vc{q}}- \varepsilon_{\vc{p}}) 
	\frac{\hbar (\vc{q} \cdot \vc{E})}{m} \sub{\tau}{tr}(\varepsilon_p).
\end{equation*}
Переходя к интегрированию, нахоим
\begin{equation*}
	(\vc{E} \cdot \vc{v}) = - \frac{4 \pi}{\hbar} \int \frac{q^2 \d q \d \Omega_q}{(2\pi)^3} w_q \frac{\hbar \omega_q (1+N_q) N_q}{T} \delta(\varepsilon_{\vc{p} + \hbar \vc{q}} - \varepsilon_p) \frac{\hbar (\vc{q} \cdot \vc{E})}{m} \sub{\tau}{tr}(\varepsilon_p).
\end{equation*}
Проведём интегрирование, введя полярную ось и расписав
\begin{align*}
	\vc{q}  &= (q \sin \theta \cos \varphi,\, q \sin \theta \sin \varphi,\, q \cos \theta), \\
	\vc{E} &= (E \sin \theta_E \cos \varphi_E,\, E \sin \theta_E \sin \varphi_E,\, E \cos \theta_E).
\end{align*}
Тогда скалярное произведение перепишется в виде
\begin{equation*}
	(\vc{q} \cdot \vc{E}) = q E \left(
		\cos \theta \cos \theta_E + \sin \theta \sin \theta_E \cos(\varphi-\varphi_E)
	\right),
\end{equation*}
где после интегрирование второе слагаемое зануляется. Также подставляя $(\vc{E} \cdot \vc{v}) = E v \cos \theta_E$, тогда
\begin{equation*}
	\frac{p}{m \sub{\tau}{tr}(\varepsilon_p)} = - \frac{4 \pi}{T} \int_0^{q_D} \frac{q^2 \d q \sin \theta \d \theta}{(2\pi)^2} w_q \omega_q (1 + N_q) N_q \times \delta\left(
		\tfrac{\hbar q p}{m}\left(\cos \theta + \tfrac{\hbar q}{2p}\right)
	\right) \times \frac{\hbar q}{m} \cos \theta,
\end{equation*}
где $q_D$ -- максимальный дебаевский импульс. Таким образом
\begin{equation*}
	\frac{1}{\sub{\tau}{tr}(\varepsilon_p)} = \frac{4 \pi m}{T p^2} \int_{0}^{q_D} \frac{q^2 \d q}{(2\pi)^2} w_q \omega_q (1+N_q)N_q \int_{-1}^{1} dx\ x \times \delta\left(x + \tfrac{\hbar q}{2p}\right),
\end{equation*}
где ввели $x = \cos \theta$.

Вообще $q_D = \sqrt[3]{6 \pi^2 n}$, $p_F = \sqrt[3]{3 \pi^2 n}$, тогда $\frac{\hbar q_D}{2 p_F} < 1$. Учитывая, что $w_q \propto \omega_q \propto q$, находим
\begin{equation*}
	\frac{1}{\sub{\tau}{tr}(\varepsilon_p)} \propto \frac{1}{T} \int_{0}^{q_D} q^5 \d q \ \frac{e^{\hbar \omega_q/T}}{(e^{\hbar \omega_q/T}-1)^2}. 
\end{equation*}
Введём $z = \frac{\hbar \omega_q}{T} = \frac{T_D}{T} \frac{q}{q_D}$, где $T_D = \hbar c_L q_D$. Таким образом 
\begin{equation*}
	\frac{1}{\sub{\tau}{tr}(\varepsilon_p)} \propto \frac{1}{T} \left(
		\frac{T}{T_D}
	\right)^6 \int_{0}^{T_D/T} \frac{e^z z^5 \d z}{(e^z-1)^2},
\end{equation*}
где из-за разности скоростей возникла пятая степень вместо четвертой. Итого, искомое выражение 
\begin{equation}
	\frac{1}{\sub{\tau}{tr}(\varepsilon_p)} \propto \left(\frac{T}{T_D}\right)^5 \int_{0}^{T_D/T} \frac{z^5 \d z}{\sh^2 \frac{z}{2}}.
\end{equation}


\textbf{Формула Друде}. Вспоминая, что
\begin{equation*}
	\sigma = \sigma_D = \frac{e^2 n \sub{\tau}{tr}}{m},
\end{equation*}
находим 
\begin{equation*}
	\frac{\sub{\rho}{e-ph}(T)}{\sub{\rho}{e-ph}(T_D)} = \frac{\sub{\sigma}{e-ph}(T)}{\sub{\sigma}{e-ph}(T_D)} = \left(\frac{T}{T_D}\right)^5 \int_{0}^{T_D/T} \frac{z^5 \d z}{\sh^2 \frac{z}{2}} \bigg/ \int_{0}^{1} \frac{z^5 \d z}{\sh^2 \frac{z}{2}}.
\end{equation*}
Для $T \ll T_D$ получится
\begin{equation*}
	\frac{\sub{\rho}{e-ph}(T)}{\sub{\rho}{e-ph}(T_D)} = 526 \left(\frac{T}{T_D}\right)^5.
\end{equation*}
И в обратную сторону, для $T \gg T_D$, раскладываясь в ряд, находим
\begin{equation*}
	\frac{\sub{\rho}{e-ph}(T)}{\sub{\rho}{e-ph}(T_D)} = 1.06 \left(\frac{T}{T_D}\right).
\end{equation*}



% T6
\section{Электроны в магнитном поле}
Запишем энергию в виде
\begin{equation*}
	\varepsilon(\vc{p}) = m_{\alpha \beta}^{-1} \frac{p_\alpha p_\beta}{2},
	\hspace{10 mm} 
	m_{\alpha \beta} = m_{\beta \alpha}.
\end{equation*}
Рассмотрим анзац, вида
\begin{equation*}
	\delta f(\vc{p}, t) = \vc{p} \cdot \vc{A}(\varepsilon) e^{- i \omega t},
\end{equation*}
подставляя в уравнение Больцмана, найдём
\begin{equation*}
	(\tau^{-1} + i \omega) (p_\mu A_\mu) - \frac{e}{c} \varepsilon_{\alpha \beta \gamma} v_\alpha B_\beta \frac{\partial }{\partial p_\gamma}  (p_\mu A_\mu) = e (\vc{v} \cdot \vc{E}) \frac{\partial f_0}{\partial \varepsilon}.
\end{equation*}
Свёртка симметричного тензора с антисимметричным даст 0, тогда
\begin{equation*}
	(\tau^{-1} - i \omega) m_{\alpha \beta} v_\alpha A_\beta - \frac{e}{c} \varepsilon_{\alpha \beta \gamma} v_\alpha B_{\beta} A_\gamma = e v_\alpha E_\alpha \frac{\partial f_0}{\partial \varepsilon}.
\end{equation*}
Вынося $v_\alpha$, можем получить выражение
\begin{equation*}
	\left(
		(\tau^{-1} - i \omega) m_{\alpha \beta} + \frac{e}{c} \varepsilon_{\alpha \beta \gamma} B_\gamma
	\right) A_\beta - e E_\alpha \frac{\partial f_0}{\partial \varepsilon} = 0,
\end{equation*}
что составляет уравнение на величину $\vc{A}$. 

Введём тензор 
\begin{equation*}
	\Gamma_{\alpha \beta} = (\tau^{-1} - i \omega) m_{\alpha \beta} + \frac{e}{c} \varepsilon_{\alpha \beta \gamma} B_\gamma,
	\hspace{0.5cm} \Rightarrow \hspace{0.5cm}
	A_\beta = e \Gamma_{\beta \gamma}^{-1} E_\gamma \frac{\partial f_0}{\partial \varepsilon}.
\end{equation*}
Таким образом нашли поправку к функции распределения
\begin{equation*}
	\delta f (\vc{p}) = e v_\alpha m_{\alpha \beta} \Gamma^{-1}_{\beta \gamma} E_\gamma \frac{\partial f_0}{\partial \varepsilon},
\end{equation*}
и, соответственно, можем найти ток
\begin{equation*}
	j_\alpha = - e \int \frac{2 \d^3 p}{(2 \pi \hbar)^3} v_\alpha \delta f = e^2 E_\gamma \int \frac{2 (\d^3 p)}{(2 \pi \hbar)^3} v_\alpha v_\nu m_{\nu \beta} \Gamma^{-1}_{\beta \gamma} (- \frac{\partial f_0}{\partial \varepsilon} ),
\end{equation*}
откуда можем найти тензор проводимости $j_\alpha = \sigma_{\alpha \beta} E_\beta$:
\begin{equation*}
	\sigma_{\alpha \beta}(\omega, \vc{B}) = e^2 \int \frac{2 \d^3 p}{(2 \pi \hbar)^3} v_\alpha v_\nu m_{\nu \mu} \Gamma^{-1}_{\mu \beta} \left(
		- \frac{\partial f_0}{\partial \varepsilon} 
	\right) = e^2 \int \frac{2 \d^3 p}{(2 \pi \hbar)^3} m_{\alpha \gamma}^{-1} p_\gamma m_{\nu \delta}^{-1} P_\delta m_{\nu \mu} \Gamma_{\mu \beta}^{-1} 
	\left(- \frac{\partial f_0}{\partial \varepsilon} \right)
	.
\end{equation*}
Свернув тензоры, находим
\begin{equation*}
	\sigma_{\alpha \beta} (\omega, \vc{B}) = e^2 m_{\alpha \gamma}^{-1} 
	\int \frac{2 \d^3 p}{(2 \pi \hbar)^3} p_\gamma p_\mu \Gamma_{\mu \beta}^{-1} (\varepsilon) \left(-\frac{\partial f_0}{\partial \varepsilon} \right) = 
	\frac{2}{3} e^2 \int d \varepsilon\ g(\varepsilon) \varepsilon \cdot \left(- \frac{\partial f_0}{\partial \varepsilon} \right) \Gamma_{\alpha \beta}^{-1} (\varepsilon).
\end{equation*}
где $g(\varepsilon) \sim \sqrt{ \varepsilon}$ -- плотность  состояний.  Переходя к плотности электронов, находим
\begin{equation*}
	\sigma_{\alpha \beta} (\omega, \vc{B}) =  \frac{2}{3} e^2 n \int d \varepsilon g(\varepsilon) \varepsilon \Gamma_{\alpha \beta}^{-1} (\varepsilon) \left(- \frac{\partial f_0}{\partial \varepsilon} \right) = n e^2 \left\langle \Gamma_{\alpha \beta}^{-1} (\varepsilon)\right\rangle.
\end{equation*}
Для металла усреднение тревиально и с учётом $\delta$-образной производной $\partial_\varepsilon f_0$ при низких температурах просто берём $\tau(\varepsilon_F)$:
\begin{equation*}
	\rho_{\alpha \beta} = \frac{1}{n e^2}\left[
		(\tau^{-1} - i \omega) m_{\alpha \beta} + \frac{e}{c} \varepsilon_{\alpha \beta \gamma} B_\gamma
	\right].
\end{equation*}
Далее считая $m_{\alpha \beta} = m \delta_{\alpha \beta}$, получим
\begin{equation*}
	\rho_{\alpha \beta} = \frac{m}{n e^2 \tau} \begin{pmatrix}
	    1- i \omega \tau & \omega_c \tau & 0 \\
	    - i \omega_c \tau & 1-i \omega \tau & 0 \\
	    0 & 0 & 1- i \omega \tau \\
	\end{pmatrix},
\end{equation*}
и для обратной матрицы $\sigma_{\alpha \beta}$, находим
\begin{equation*}
	\sigma_{\alpha \beta} (\omega,\, \vc{B}) = \frac{\sigma_D}{(1-i \omega \tau)^2 + (\omega_c \tau)^2} \begin{pmatrix}
	    1- i \omega \tau & - \omega_c \tau & 0 \\
	    \omega_c \tau & 1-i \omega \tau & 0 \\
	    0 & 0 & \frac{(1-i \omega \tau)^2 + (\omega_c \tau)^2}{1-i \omega \tau}  \\
	\end{pmatrix},
	\hspace{10 mm} 
	\omega_c = \frac{e B}{m c}.
\end{equation*}
где $\sigma_D = \frac{n e^2 \tau}{m}$.


Для тока можем записать
\begin{equation*}
	j_\alpha (t) = \int_{-\infty}^{+\infty} j_\alpha(\omega) e^{- i \omega t} \frac{\d \omega}{2\pi} = \int_{-\infty}^{+\infty} \sigma_{\alpha \beta}(\omega, \vc{B}) E_\beta (\omega) e^{- i \omega t} \frac{\d \omega}{2\pi}.
\end{equation*}
Переходя к обратному Фурье-образу для поля, находим
\begin{equation*}
	j_\alpha (t) = \int_{-\infty}^{+\infty} 
	\sigma_{\alpha \beta} (t-t', \vc{B})
	 E_\beta (t') \d t',
	 \hspace{10 mm} 
	 \sigma_{\alpha \beta} (t-t', \vc{B}) = \int_{-\infty}^{+\infty} \sigma_{\alpha \beta} (\omega, \vc{B}) e^{- i \omega (t-t')} \frac{\d \omega}{2\pi}.
\end{equation*}
Теперь можем явно найти
\begin{equation*}
	\sigma_{z z} (t, \vc{B}) = \theta(t)  \sigma_D  \frac{e^{-t/\tau}}{\tau},
	\hspace{5 mm} 
	\sigma_{xx} = \sigma_{yy} = \sigma_D \theta(t) \frac{e^{-t/\tau}}{\tau} \cos(\omega_c t),
	\hspace{5 mm} 
	\sigma_{yx} = - \sigma_{xy} = \sigma_D \theta(t) \frac{e^{-t/\tau}}{\tau} \sin(\omega_c t),
\end{equation*}
где учли, что полюса подинтегрального выражения находятся в нижней полуплоскости:
\begin{equation*}
	\omega = - \frac{i}{\tau}, \hspace{10 mm} 
	\omega = - \frac{i}{\tau} \pm \omega_c.
\end{equation*}






% T7
\section{Модель диффузии Лоренца}


\textbf{Несохранение числа частиц}. 
В $\tau$-приближении:
\begin{equation*}
	\frac{\partial f}{\partial t}  + \vc{v} \cdot \frac{\partial f}{\partial \vc{r}} = - \frac{f-f_0}{\tau},
	\hspace{10 mm} 
	\delta n = \int \delta f \d^3 \vc{r},
	\hspace{5 mm} 
	F(\vc{v}, t) \overset{\mathrm{def}}{=} \int \d^3 \vc{r}\ f(\vc{r}, \vc{v}, t).
\end{equation*}
Проинтегрируем уравнение Больцмана по координатам:
\begin{equation*}
	\frac{\partial F}{\partial t} = - \frac{F-F_0}{\tau},
\end{equation*}
Введя $\delta F (\vc{v}, t) = F(\vc{v}, t) - F_0 (\vc{v})$, найдём
\begin{equation*}
	\delta F(\vc{v}, t) = \delta F(\vc{v}, 0) e^{- t/\tau},
\end{equation*}
таким образом $\tau$-приближение не сохраняет число частиц, релаксируя к равновесному. 


\textbf{Модификация}. Исправим эту проблему следующим образом
\begin{equation*}
	\frac{\partial f}{\partial t} + \vc{v} \frac{\partial f}{\partial \vc{r}}  = \frac{1}{\tau} \left[
		- f + \int \frac{\d \Omega_v}{4\pi} f
	\right] = \frac{1}{\tau} \left(Pf - f\right),
	\hspace{10 mm} 
	P f = \int \frac{\d \Omega_v}{4\pi} f(\vc{r}, \vc{v}, t).
\end{equation*}
что называется моделью Лоренца, случай легкой примеси в тяжелом газе, а именно слабо-ионизированный газ. Здесь $Pf$ -- члены прихода.  Электроны рассеиваются\footnote{
	См. ЛЛX.
}  на тяжелых частицах.
Забавный факт -- тут возникает диффузия, а ещё эта модель имеет точное решение. 


\textbf{Проверка}. Аналогично перейдём к функции $F$, тогда
\begin{equation*}
	\frac{\partial F}{\partial t} = \frac{1}{t}\left(
		P F(v, t) - F(\vc{v}, t)
	\right),
\end{equation*}
тогда, после применения проекции $P$, находим
\begin{equation*}
	\frac{\partial (PF)}{\partial t}  = \frac{1}{\tau}\left[P^2 F - P F\right] = 0,
	\hspace{0.5cm} \Rightarrow \hspace{0.5cm}
	PF(v, t) = \Phi(v).
\end{equation*}
Так находим, что
\begin{equation*}
	F(\vc{v}, t) = \Phi(v) + \left[
		F_0 (\vc{v}) - \Phi(v)
	\right] e^{- t/\tau}.
\end{equation*}



\textbf{Лаплас}. Рассмотрим уравнение
\begin{equation*}
	\frac{\partial f}{\partial t}  + \vc{\nabla} \cdot \frac{\partial f}{\partial \vc{r}} = - \frac{1}{\tau}\left(
		f - \langle f\rangle
	\right).
\end{equation*}
Сдлаем преобразование Фурье в пространстве и преобразование Лапласа по времени:
\begin{equation*}
	\hat{f} (\vc{k}, \vc{v}, s) = \int_{0}^{\infty} e^{-st}\d t \int d^3 r\ e^{- i \vc{k} \vc{r}} f(\vc{r}, \vc{v}, t).
\end{equation*}
\textbf{вставить из фото}.


Приходим к интегралу
\begin{equation*}
	\frac{1}{2} \int_{-1}^{1} dx\ \frac{(1+s \tau)  - i v k \tau x}{(i + s \tau)^2 + (v k \tau x)^2} = \frac{1}{v k \tau} \arctg \frac{v k \tau}{1+s \tau}.
\end{equation*}
Подставляем всё в $P \hat{f}$
\begin{equation*}
	P \hat{f}(\vc{k}, \vc{v}, s) = \left[
		1 - \frac{1}{vk\tau} \arctg \frac{v k \tau}{1 + s \tau}
	\right] \int \frac{\d \Omega_v}{4\pi} \frac{f(\vc{k}, \vc{v}, t=0)}{s + i \vc{k} \cdot \vc{v} + \tau^{-1}},
\end{equation*}
находим
\begin{equation*}
	\hat{f}(\vc{k}, \vc{v}, s) = \frac{\tau^{-1}}{s + i \vc{v} \cdot \vc{k} + \tau^{-1}} \left[
		1 - \frac{1}{vk\tau} \arctg \frac{v k \tau}{1 + s \tau}
	\right] \int \frac{\d \Omega_v}{4\pi} \frac{f(\vc{k}, \vc{v}, t=0)}{s + i \vc{k} \cdot \vc{v} + \tau^{-1}} +  \frac{f(\vc{k}, \vc{v}, t=0)}{s + i \vc{k} \cdot \vc{v} + \tau^{-1}}.
\end{equation*}


Конкретизируем начальные условия:
\begin{equation*}
	f(\vc{r}, \vc{v}, t=0) = \delta(\vc{r}) \delta(\vc{v}-\vc{v}_0),
	\hspace{0.5cm} \Rightarrow \hspace{0.5cm}
	f(\vc{k}, \vc{}, t=0) = \delta(\vc{v}-\vc{v}_0).
\end{equation*}
Подставляя в интеграл по телесному углу, находим
\begin{equation*}
	\int \frac{\d \Omega_v}{4\pi}  \frac{f(\vc{k}, \vc{v}, t=0)}{s + i \vc{k} \cdot \vc{v} + \tau^{-1}} 
	= \int \frac{\d \Omega_v}{4\pi} \frac{\delta(\vc{v} - \vc{v}_0)}{s + i \vc{k} \cdot \vc{v} + \tau^{-1}}
	= \frac{1}{s + i \vc{k} \cdot \vc{v} + \tau^{-1}} \frac{\delta(v-v_0)}{4 \pi v_0^2}.
\end{equation*}




\textbf{Диффузия}. Рассматриваем время $t \gg \tau$, тогда малые $s \tau \ll 1$, и можем разложиться
\begin{equation*}
	1 - \frac{1}{vk\tau} \arctg \frac{v k \tau}{1 + s \tau} = 1 - \frac{1}{1 + s \tau} + \frac{1}{3} \frac{(v k \tau)^2}{(1+s \tau)^3} \approx s \tau + \frac{1}{3} v^2 k^2 \tau^2 + \ldots 
\end{equation*}
Подставляя в выражение для $\hat{f}$, находим
\begin{equation*}
	\hat{f}(\vc{k}, \vc{v}, s) =  \left(
		\frac{\tau^{-1}}{s + i \vc{v} \cdot \vc{k} + \tau^{-1}} 
	\right)^2 \frac{1}{s + \frac{1}{3} v^2 k^2 \tau^2}  \frac{\delta(v-v_0)}{4 \pi v_0^2} + \frac{\delta(\vc{v}-\vc{v}_0)}{s + i (\vc{k} \cdot \vc{v}_0) + \tau^{-1}}.
\end{equation*}
Смотрим большие времена и большие расстояния, тогда самое большое это $\tau^{-1}$, и можем переписать функцию распределения $\hat{f}$ в виде
\begin{equation*}
	\hat{f}(\vc{k}, \vc{v}, s) \approx \frac{1}{s + D k^2} \frac{\delta(v-v_0)}{4 \pi v_0^2},
	\hspace{10 mm} 
	D = \frac{1}{3} v_0^2 \tau.
\end{equation*}
Возвращаясь к обратному Фурье-образу, находим
\begin{equation*}
	f(\vc{r}, \vc{v}, t) = \int_{s^* - i \infty}^{s^* + i \infty} \frac{e^{st \d s}}{2 \pi i} \int \frac{d^3 k}{(2\pi)^3} e^{i \vc{k} \vc{r}} \hat{f}(\vc{k}, \vc{v}, s).
\end{equation*}
Считая по вычетам, находим
\begin{equation*}
	f(\vc{r}, \vc{v}, t) = \frac{\delta(v-v_0)}{4 \pi v_0^2} \left[
		\int_{-\infty}^{+\infty} \frac{\d k_x}{2\pi}  \exp\left(
			- Dt( k_x - \frac{i x}{2 D t})^2 -\frac{x^2}{4 D t}
		\right)
	\right] \left[ \int_{-\infty}^{+\infty}  \frac{\d k_y}{2\pi} \ldots \right] \left[ \int_{-\infty}^{+\infty}  \frac{\d k_z}{2\pi} \ldots \right],
\end{equation*}
так приходим к явной диффузии
\begin{equation}
	f(\vc{r}, \vc{v}, t) =  \frac{1}{(4 \pi D t)^{3/2}}  \frac{\delta(v-v_0)}{4 \pi v_0^2} e^{-r^2/4Dt},
	\hspace{10 mm} 
	D = \frac{1}{3} v_0^2 \tau.
\end{equation}





% T8
\section{Электронный газ}
И снова смотрим на уравнение Больцмана, ищём решение в виде $f = f_0 + \delta f$, смотрим на $\tau$-приближение, равновесным будет распределение Ферми:
\begin{equation*}
	f_0 = \frac{1}{e^{\frac{\varepsilon-\mu}{T}} + 1},
	\hspace{10 mm} 
	\mu = \mu (t, \vc{r}),
	\hspace{5 mm} 
	T = T(t, \vc{r}).
\end{equation*}
Будем решать уравнение рассматривая стационарный случай
\begin{equation*}
	\vc{v} \cdot \frac{\partial f_0}{\partial \vc{r}} - e \vc{E} \cdot \vc{v} \frac{\partial f_0}{\partial \varepsilon}  = - \frac{\delta f}{\tau}.
\end{equation*}
Можем переписать 
\begin{equation*}
	\frac{\partial f_0}{\partial \vc{r}} = \frac{\partial f_0}{\partial T}  \vc{\nabla} T + \frac{\partial f_0}{\partial \mu} \vc{\nabla} \mu = - \frac{\varepsilon-\mu}{T} \frac{\partial f}{\partial \varepsilon} \vc{\nabla} T - \frac{\partial f_0}{\partial \varepsilon} \nabla \mu.
\end{equation*}
Тогда, после подстановки, левая часть уравнения может быть найдена в виде
\begin{equation*}
	\delta f = \tau \left(
		\frac{\varepsilon-\mu}{T} (\vc{v} \cdot \vc{\nabla} T) + \vc{v} \cdot (\vc{\nabla} \mu + e \vc{E})
	\right) \frac{\partial f_0}{\partial \varepsilon} .
\end{equation*}


\textbf{Металл}. Достаточно рассмотреть $- \frac{\partial f_0}{\partial \varepsilon} \approx \delta(\varepsilon - \varepsilon_F)$. Для тока $\vc{j}$ находим
\begin{equation*}
	\vc{j} = - e \int \vc{v} (f_0 + \delta f) \frac{2 \d^3 p}{(2\pi \hbar)^3} = \frac{e}{3} (\vc{\nabla} \mu + e \vc{E}) \int \tau v^2 \left(
		- \frac{\partial f_0}{\partial \varepsilon} 
	\right) \frac{2 \d^3 p}{(2\pi \hbar)^3} + \frac{e}{3} \frac{\vc{\nabla} T}{T} \int \tau v^2 (\varepsilon-\mu) \left(-\frac{\partial f_0}{\partial \varepsilon} \right) \frac{2 \d^3 p}{(2\pi \hbar)^3}.
\end{equation*}
Для плотности потока энергии $\vc{q}$ 
\begin{equation*}
	\vc{q} = \int \vc{v} (\varepsilon- e \varphi) (f_0 + \delta f) \frac{2 \d^3 p}{(2\pi \hbar)^3} = - \frac{\vc{j}}{e} (\mu - e \varphi) - \ldots.
\end{equation*}
% вставить из фото!
Введём диссипативную часть $\vc{q}'$
\begin{equation*}
	\vc{q}' = \vc{q} + \frac{\vc{j}}{e} (\mu- e \varphi).
\end{equation*}
Также определим усреднение в виде
\begin{equation*}
	\langle F(\varepsilon)\rangle = \frac{m}{3n} \int \frac{2 \d^3 p}{(2 \pi \hbar)^3} v^2  \left(- \frac{\partial f_0}{\partial \varepsilon} \right)  F(\varepsilon) = \frac{2}{3n} \int_{0}^{\infty} \varepsilon \left(- \frac{\partial f_0}{\partial \varepsilon} \right) F(\varepsilon) g(\varepsilon) \d \varepsilon,
	\hspace{10 mm} 
	n = \int_{0}^{\infty} \varepsilon \left(- \frac{\partial f_0}{\partial \varepsilon} \right) g(\varepsilon) \d \varepsilon.
\end{equation*}
Тогда уравнение перепишется в виде
\begin{equation*}
	\vc{E} + \frac{\vc{\nabla} \mu}{e} = \frac{m \vc{j}}{n e^2 \langle \tau\rangle} - \frac{\vc{\nabla} T}{ e T} \frac{\langle (\varepsilon -\mu) \tau\rangle}{\langle \tau\rangle} = \frac{\vc{j}}{\sigma} + \alpha \vc{\nabla} T.
\end{equation*}
Тогда для потока энергии
\begin{equation*}
	\vc{q}' = - \frac{\langle (\varepsilon-\mu) \tau\rangle}{e \langle  \tau\rangle} \vc{j} + \frac{\vc{\nabla} T}{m T} \frac{n \langle (\varepsilon-\mu) \tau\rangle^2}{\langle \tau\rangle} - 
	\frac{\vc{\nabla} T}{m T} n \langle (\varepsilon-\mu)^2 \tau\rangle 
	= 
	\alpha T \vc{j} - \varkappa \vc{\nabla} T
	.
\end{equation*}
Где коэффициенты соответственно равны
\begin{equation}
	\alpha = - \frac{\langle (\varepsilon-\mu) \tau \rangle}{e T \langle \tau\rangle},
	\hspace{10 mm} 
	\varkappa = \frac{n \langle \tau\rangle}{m T}\left[
		\frac{\langle (\varepsilon-\mu)^2 \tau\rangle}{\langle \tau\rangle} - \frac{\langle (\varepsilon-\mu) \tau\rangle^2}{\langle \tau\rangle^2}
	\right],
	\hspace{10 mm} 
	\sigma = \frac{n e^2 \langle \tau\rangle}{m}.
\end{equation}
где $\varkappa$ -- коэффициент теплопроводности, $\alpha$ -- термоэлектрический коэффициентр, $\sigma$ -- проводимость. 


% Соотношения Онзагера можем получить, записав с фото
% \begin{equation*}
% 	\vc{j} = 
% \end{equation*}




\textbf{Полупроводник}. Здесь можем написать, что $\frac{\partial f_0}{\partial \varepsilon} = - \frac{\partial f_0}{\partial T}$, так как $f_0 \approx e^{(\mu-\varepsilon)/T}$. Тогда усредение можем переписать в виде
\begin{equation*}
	\langle F(\varepsilon)\rangle  = \frac{m}{3 n T} \frac{2 \d^3 p}{(2 \pi \hbar)^3} f_0 v^2 F(\varepsilon).
\end{equation*}
Считая, что $\tau(\varepsilon) \propto v^k \propto \varepsilon^{k/2}$ и что $f_0 \propto e^{-\frac{m v^2}{2T}}$, находим
\begin{equation*}
	\langle v^k\rangle \propto \left(\frac{2 T}{m}\right)^{k/2} \Gamma\left(\frac{3+k}{2}\right).
\end{equation*}
Так, например, для $\alpha$ получится
\begin{equation*}
	\alpha = \frac{1}{e} \left(
		\frac{\mu}{T} - \frac{\langle \tau v^2 \varepsilon\rangle}{\langle  \tau v^2\rangle}
	\right) = \frac{1}{e} \left(
		\frac{\mu}{T} - \frac{\Gamma\left(\frac{1+k}{2}\right)}{\Gamma\left(\frac{3+k}{2}\right)}
	\right) = \frac{1}{e} \left(
		\frac{\mu}{T} - \frac{5+k}{2}
	\right).
\end{equation*}


% Для полупроводника получается коэффициент плетье гораздо больше.
% ДЗ: найти коэффициент плетье в металле, должно получиться \mu/E_F 


% число Лоренца -- винеман -франц закон вывести
% 

% T9
\section{Уравнения Навье-Стокса}
Lorem ipsum dolor sit amet, consectetur adipisicing elit, sed do eiusmod
tempor incididunt ut labore et dolore magna aliqua. Ut enim ad minim veniam,
quis nostrud exercitation ullamco laboris nisi ut aliquip ex ea commodo
consequat. Duis aute irure dolor in reprehenderit in voluptate velit esse
cillum dolore eu fugiat nulla pariatur. Excepteur sint occaecat cupidatat non
proident, sunt in culpa qui officia deserunt mollit anim id est laborum.




%%%%%%%%%% ВТОРОЕ ЗАДАНИЕ %%%%%%%%%%%%%%%%%%%%%%%

% Т12
\setcounter{section}{9}
\section{Холловская проводимость}
Рассмотрим систему
\begin{align*}
	\dot{x}_\alpha + \varepsilon_{\alpha \beta \gamma} \dot{k}_\beta \Omega_n^\gamma = v_n^\alpha \\ 
	\dot{k}_\alpha + \frac{e}{\hbar c} \varepsilon_{\alpha \beta \gamma} \dot{x}_\beta B_\gamma = - \frac{e}{\hbar} E_\alpha.
\end{align*}
Решение можем найти, переписа в виде
\begin{equation*}
	\begin{pmatrix}
	    \delta_{\alpha \beta} & \varepsilon_{\alpha \beta \gamma} \Omega^\gamma_n  \\
	    \frac{e}{\hbar c}\varepsilon_{\alpha \beta \gamma} B^\gamma & \delta_{\alpha \beta}  \\
	\end{pmatrix} \begin{pmatrix}
		\dot{x}_\alpha \\ \dot{k}^\beta
	\end{pmatrix} = \begin{pmatrix}
		v_n^{\alpha} \\  -\frac{e}{\hbar} E^{\alpha}
	\end{pmatrix},
\end{equation*}
для координаты и импульса
\begin{align*}
	\dot{x}_\alpha &= (1 + \frac{e}{\hbar c} \vc{B} \cdot \vc{\Omega}_n)^{-1}\left(
		v_n^\alpha + \frac{e}{\hbar c} (\vc{v}_n \cdot \vc{\Omega}_n) B^\alpha + \frac{e}{\hbar} \varepsilon_{\alpha \beta \gamma} E^\beta \Omega_n^\gamma
	\right) \\
	\dot{k}_\alpha &= -\frac{e}{\hbar} \left(1 + \frac{e}{\hbar c} \vc{B} \cdot \vc{\Omega}_n\right)^{-1} \left(
		E^\alpha + \frac{e}{\hbar c} \left(\vc{E} \cdot \vc{B}\right) \Omega_n^\alpha + \frac{1}{c} \varepsilon_{\alpha \beta \gamma} v_n^\beta B^\gamma
	\right).
\end{align*}

\textbf{Несохранение фазового объема}. Заметим, что
\begin{equation*}
	\frac{\partial \dot{x}_\alpha}{\partial x_\alpha}  + \frac{\partial \dot{k}_\alpha}{\partial k_\alpha} = - \frac{d \ln D_n}{d t},
	\hspace{10 mm} 
	D_n(\vc{r}, \vc{k}, t) = 1 + \frac{e}{\hbar c} \vc{B}(\vc{r}, t) \cdot \vc{\Omega}_n (\vc{k}).
\end{equation*}
Таким образом фазовый объем увеличивается в соответствии с 
\begin{equation*}
	\frac{d \ln \Delta V}{d t}  = \vc{\nabla}_{\vc{r}} \cdot \dot{\vc{r}} + \vc{n}_{\vc{k}} \cdot \dot{\vc{k}} = - \frac{\d}{dt} \ln D_n (\vc{r}, \vc{k}, t),
\end{equation*}
где $\Delta V = \Delta \vc{r} \ \Delta \vc{k}$, и тогда $\Delta V(t) = \Delta V(0) / D_n (\vc{r}, \vc{k}, t)$.
Это можно исправить заменой
\begin{equation*}
	\d \mu = \frac{\d^3 r \d^3 k}{(2\pi)^3} 
	\hspace{5 mm} 
	\to
	\hspace{5 mm} 
	\d \tilde{\mu} \overset{\mathrm{def}}{=} D_n(\vc{r}, \vc{k}, t) \frac{\d^3 r \d^3 k}{(2\pi)^3} .
\end{equation*}
и в дальнейшем интегрировать уже в новой метрике.

\textbf{Проводимость}. Среднее для любой локальной наблюдаемой может быть получено в виде
\begin{align*}
	\langle \mathcal O\rangle(\vc{r}, t) = \sum_n \int \d \tilde{\mu} f_n (\vc{r}, \vc{k}, t) \langle u_{n \vc{k}}| \mathcal O | u_{n \vc{k}}\rangle \delta(\vc{r}-\vc{r}')
	.
\end{align*}
В равновесном случае $f_n(\vc{k})$ -- функция Ферми $f(E_n(\vc{k}) - \mu)$. Для тока тогда
\begin{equation*}
	j^\alpha_n (\vc{r}, t) = -e \int  \frac{\d^3 k}{(2\pi)^3} \left(
		v_n^\alpha + \frac{e}{\hbar c} (\vc{v}_n \cdot \vc{\Omega}_n) B^\alpha + \frac{e}{\hbar} \varepsilon_{\alpha \beta \gamma} E^\beta \Omega_n^\gamma
	\right) f_n (\vc{k}).
\end{equation*}
Для случая $\vc{B} = 0$ явно можем найти
\begin{equation*}
	\vc{j}_n = -\frac{e^2}{\hbar} \vc{E} \times \int \frac{\d^3 k}{(2\pi)^3} \vc{\Omega}_n (\vc{k}).
\end{equation*}



% Т12
\setcounter{section}{11}
\section{Тяжелая частица в лёгком газе}
\textbf{Уравнениве Фоккера-Планка}.
Заметим, что
\begin{equation*}
	\frac{\partial f(t, \vc{p})}{\partial t} =  \int d^3 \vc{q}\
	\left(
		w(\vc{p}+\vc{\vc{q},q}) f(t, \vc{p}+\vc{q}) - w(\vc{p}, \vc{q}) f(t, \vc{p})
	\right),
\end{equation*}
раскладываясь до второго порядка малости по $\vc{q}$, находим
\begin{equation*}
	\frac{\partial f(t, \vc{p})}{\partial t}  = 
	\frac{\partial }{\partial p_\alpha} \left(
		\tilde{A}_\alpha f + \frac{\partial }{\partial p_\beta} (B_{\alpha \beta} f)
	\right),
\end{equation*}
где ввели
\begin{equation*}
	\tilde{A}_\alpha = \int q_\alpha w(\vc{p}, \vc{q}) \d^3 \vc{q} = \frac{\sum_{\delta t} q_\alpha}{\delta t},
	\hspace{10 mm} 
	B_{\alpha \beta} = \frac{1}{2} \int q_\alpha q_\beta q(\vc{p}, \vc{q}) \d^3 \vc{q} = \frac{\sum_{\delta t} q_\alpha q_\beta}{2 \delta t}.
\end{equation*}
Перепишем уравнение в виде
\begin{equation*}
	\frac{\partial f}{\partial t}  = - \frac{\partial s_\alpha}{\partial p_\alpha},
	\hspace{5 mm} \Leftrightarrow \hspace{5 mm} 
	\frac{\partial f}{\partial t}  + \div_{\vc{p}} \vc{s} = 0,
\end{equation*}
где величина $\vc{s}$ -- плотность потока в импульсном пространстве
\begin{equation*}
	s_\alpha = - \tilde{A}_\alpha f - \frac{\partial }{\partial p_\beta} (B_{\alpha \beta} f) = - A_\alpha f - B_{\alpha \beta} - B_{\alpha \beta} \frac{\partial f}{\partial p_\beta},
	\hspace{10 mm} 
	A_\alpha = \tilde{A}_\alpha + \frac{\partial B_{\alpha \beta}}{\partial p_\beta}.
\end{equation*}
Итого в общем виде уравнение Фоккера-Планка можем иметь вид
\begin{equation}
	\frac{\partial f}{\partial t} + \vc{v} \frac{\partial f}{\partial \vc{r}} + \vc{F} \frac{\partial f}{\partial \vc{p}} + \div_{\vc{p}} \vc{s} = 0.
\end{equation}
Считатать обычно проще $B_{\alpha \beta}$, а потом они друг через друга выражаются с учётом того, что в равновесии поток $\vc{s}^{(0)}$ зануляется.



\textbf{Тяжелые частицы}. Будем считать, что
\begin{equation*}
	f^{(0)} \sim \exp\left(
		- \frac{p^2}{2MT}
	\right).
\end{equation*}
Начнём с вычисления коэффициентов $B_{\alpha \beta}$:
\begin{equation*}
	B_{\alpha \beta} = B \delta_{\alpha \beta},
	\hspace{10 mm} 
	B = \frac{\sum_{\delta t} q^2}{6 \delta t},
\end{equation*}
и кинетическое уравнение перепишется в виде
\begin{equation*}
	\frac{\partial f(t, \vc{p})}{\partial t} = B \div_{\vc{p}} \left(
		\frac{\vc{p} f}{MT} + \nabla_{\vc{p}} f
	\right).
\end{equation*}
% уравнение Полуховского
% D = b T -- соотношение эйнштейна, из зануления потока в равновесии
Таким образом $B$ иммет смысл коэффициента диффузии в импульсном пространстве. 

Для определения величины $B$ выразим
\begin{equation*}
	\vc{q} = \Delta \vc{p}_b = \vc{p}_b - \bar{p}_b' = \vc{p}_a' - \vc{p}_a.
\end{equation*}
Считая, что $p_a = p_a'$, находим
\begin{equation*}
	q^2 = 2 p_a^2 - 2 p_a^2 \cos \theta = 2 p_a^2 (1-\cos \theta).
\end{equation*}
И тогда можем посчитать интеграл вида
\begin{equation*}
	B = \frac{1}{6} \frac{\sum_{\delta t} q^2}{\delta t} = \frac{1}{6} \int 2 p_a^2 (1-\cos \theta) f_a^{(0)} (\vc{p}_a) v_a \d \sigma \d^3 \vc{p}_\alpha.
\end{equation*}
Вводя $n = \int f^{(0)}(\vc{p}_\alpha) \d^3 \vc{p}_a$, находим
\begin{equation*}
	B = \frac{n_a}{3m} \langle p_a^3 \sigma_t(v_a)\rangle = \frac{m^2 n_a}{3} \langle v_a^3 \sigma_t \rangle,
\end{equation*}
где $m$ -- масса легкой частица, $\sigma_t$ -- транспортное сечение рассеяния легких частиц на тяжелых. 
% Тяжелые частицы диффузируют в легких


\textbf{Диффузия}. Добавив силу $\vc{F}$ можем найти
\begin{equation*}
	\frac{\partial f}{\partial t} + \frac{\partial }{\partial \vc{p}} \left(
		\left(\vc{F} - \frac{B \vc{p}}{MT}\right) f - B \frac{\partial f}{\partial \vc{p}} 
	\right) = 0.
\end{equation*}
На больших $\vc{p}$ функция распределения обращается в ноль. Для стационарного случая
\begin{equation*}
	\left(\vc{F} - \frac{B \vc{p}}{MT}\right) f - B \frac{\partial f}{\partial \vc{p}}  = \const = 0.
\end{equation*}
Функция распределения при внешней силе модифицируется к виду
\begin{equation*}
	f \sim \exp\left(
		- \frac{(\vc{p} - \vc{F} M T/B)^2}{2MT}
	\right) = \exp\left(
		- \frac{(\vc{p} - M \vc{u})^2}{2MT}
	\right),
	\hspace{10 mm} 
	\vc{u} = \frac{T}{B} \vc{F},
\end{equation*}
где $\vc{u}$ -- средняя потоковая скорость. Вообще подвижность $b$ определяется из $\vc{u} = b \vc{F}$, откуда находим $b = T/B$ и $D = b T = T^2/B$.




% Т12
% 
% со звёздочкой, не обязательно

% Т14 
% \newpage
\setcounter{section}{13}
\section{x Броуновское движение}
% #12 по гайду

Добавим случайную силу к уравнению движения
\begin{equation*}
	m \frac{d \vc{v}}{d t} = \Frand (t) + \vc{F}(t),
	\hspace{0.5cm} \Rightarrow \hspace{0.5cm}	
	m \frac{d \langle \vc{v}\rangle}{d t} = \langle \Frand (t) \rangle + \vc{F}(t).
\end{equation*}
Вообще $\langle \vc{v}\rangle = b \vc{F}$, при этом $\langle \Frand(t)\rangle + \vc{F} = 0$, тогда
\begin{equation*}
	\Frand (t) = - \frac{\vc{v}(t)}{b} + \frand(t),
	\hspace{10 mm} 
	\langle \frand(t) \rangle = 0.
\end{equation*}
Тогда уравнение движения перепишется в виде
\begin{equation*}
	\frac{d \vc{v}}{d t} = - \gamma \vc{v} + \frac{1}{M}\left(
		\frand(t) + \vc{F}(t)
	\right),
	\hspace{10 mm} 
	\gamma = \frac{1}{bM}.
\end{equation*}
Уже можем сказать, что
\begin{equation*}
	\langle \frand_i(t) \frand_k(t')\rangle = \kappa \delta_{ik} \delta(t-t').
\end{equation*}
Введём также $\ffull = \frand + \vc{F}(t)$. Теперь перейдём к Фурье-образу
\begin{equation*}
	\vc{v}(t) = \int_{-\infty}^{+\infty} v_\omega e^{- i \omega t} \frac{\d \omega}{2\pi},
	\hspace{10 mm} 
	\ffull(t) = \int_{-\infty}^{+\infty} \ffull_\omega e^{- i \omega t} \frac{\d \omega}{2\pi}.
\end{equation*}
Тогда уравнение перепишется в виде
\begin{equation*}
	- i \omega \vc{v}_\omega = - \gamma \vc{v}_\omega + \frac{1}{M} \ffull_\omega,
	\hspace{0.5cm} \Rightarrow \hspace{0.5cm}	
	\vc{v}_\omega = \frac{\ffull_\omega}{M(\gamma-i \omega)}.
\end{equation*}
Полезно ввести отклик системы $\vc{r}_\omega$
\begin{equation*}
	\vc{r}_\omega = \frac{\vc{v}_\omega}{-i \omega} = \chi(\omega) \ffull_\omega,
	\hspace{10 mm} 
	\chi(\omega) = \frac{i \gamma/\omega - 1}{M(\gamma^2 + \omega^2)},
	\hspace{5 mm} 
	|\chi|^2 = \frac{1}{M^2 \omega^2 (\gamma^2 + \omega^2)} = \frac{\Im \chi}{M \omega \gamma}.
\end{equation*}

\textbf{Диссипативная теорема}. Рассмотрим
\begin{equation*}
	x(t) = \int_{-\infty}^{+\infty} x_\omega e^{- i \omega t} \frac{\d \omega}{2\pi},
\end{equation*}
тогда коррелятор
\begin{equation*}
	\langle x(t) x(t')\rangle = \int_{-\infty}^{+\infty} \int_{-\infty}^{+\infty} \langle x_{\omega} x_{\omega'}\rangle e^{- i \omega t - i \omega' t} \frac{\d \omega \d \omega'}{(2\pi)^2}.
\end{equation*}
Учитывая, что $\langle x_\omega x_{\omega'}\rangle = 2 \pi (x^2)_\omega\delta(\omega+\omega')$, где $(x^2)_\omega$ -- спектральная плотность, находим
\begin{equation*}
	\langle x(t) x(t')\rangle = \int_{-\infty}^{+\infty} (x^2)_\omega e^{- i \omega(t-t')} \frac{\d \omega}{2\pi}.
\end{equation*}
Для $t=t'$ просто
\begin{equation*}
	\langle x^2(t)\rangle = \int_{-\infty}^{+\infty} (x^2)_\omega \frac{\d \omega}{2\pi}.
\end{equation*}
Но мы знаем, что $\vc{v} = d \vc{r} / dt$, и тогда
\begin{equation*}
	(v_i v_k)_\omega = - i \omega \chi(\omega) \cdot i \omega \chi(-\omega) \cdot \left(
		\frand_i \frand_k
	\right)_\omega = \omega^2 |\chi(\omega)|^2 (\frand_i \frand_k)_\omega.
\end{equation*}

Для нахождения константы удобно посмотреть на одновременной коррелятор
\begin{equation*}
	\langle v_i (t) v_k (t)\rangle = \delta_{ik} \frac{T}{M} = \int_{-\infty}^{+\infty} (v_i v_k)_\omega \frac{\d \omega}{2\pi} = \int_{-\infty}^{+\infty} \omega^2 |\chi(\omega)|^2 (\frand_i \frand_k)_\omega \frac{\d \omega}{2\pi}.
\end{equation*}
Итак, получаем уравнение 
\begin{equation*}
	\delta_{ik} \frac{T}{M} = \delta_{ik} \kappa  \int_{-\infty}^{+\infty} \omega^2 \frac{1+\gamma^2/\omega^2}{M^2(\gamma^2+\omega^2)^2} \frac{\d \omega}{2\pi}.
\end{equation*}
Интеграл равен $\pi/\gamma$ и тогда
\begin{equation}
	\kappa = 2 \gamma M T.
\end{equation}
Искомый коррелятор тогда равен
\begin{equation*}
	\langle \frand_i(t) \frand_k(t')\rangle =  2 \gamma M T \delta_{ik} \delta(t-t').
\end{equation*}
Аналогично можем найти
\begin{equation*}
	\langle v_i v_k\rangle_\omega = \omega^2 |\chi(\omega)|^2 \cdot 2 \gamma MT \delta_{ik},
	\hspace{10 mm} 
	\langle x_i x_k\rangle_\omega = |\chi(\omega)|^2 \cdot 2 \gamma MT \delta_{ik}.
\end{equation*}
Подставляя через мнимую часть отклика, находим
\begin{equation*}
	\langle v_i v_k\rangle_\omega = 2 \delta_{ik} T \omega \Im \chi,
	\hspace{10 mm} 
	\langle x_i x_k\rangle_\omega = 2 \delta_{ik} \frac{T}{\omega} \Im \chi.
\end{equation*}
Более явно можем найти 
\begin{equation*}
	\langle v_i(t) v_k(t')\rangle = \int_{-\infty}^{+\infty} (v_i v_k)_\omega e^{- i \omega(t-t')} \frac{d \omega}{2\pi} = \delta_{ik} \int_{-\infty}^{+\infty} \frac{2T \gamma e^{- i \omega(t-t')}}{M(\gamma^2+\omega^2)} \frac{\d \omega}{2\pi} = \delta_{ik} \frac{T}{M} e^{-\gamma|t-t'|}.
\end{equation*}
Интегрируя полученное выражение по $t$, получим
\begin{equation*}
	\int_{t'}^{\infty} \langle  v_i(t) v_k(t')\rangle \d t = \delta_{ik} \frac{T}{\gamma M} = \delta_{ik} \cdot \frac{T}{M} \cdot b M = \delta_{ik} b T = D \delta_{ik}.
\end{equation*} 


% 15, 16, 18, 19 -- решены у лектора, часть из смотреть в бурмистрове


\subsection{Среднеквадратичное отклонение}

Частица двигается случайным образом и хотим найти $\Delta \vc{r} (t) = \vc{r}(t+t_0) - \vc{r}(t_0)$. Умеем выражать коррелятор через спектральную плотность:
\begin{equation*}
	\left\langle (\Delta \vc{r}(t))^2\right\rangle = 
	2 \int_{-\infty}^{+\infty} (1-e^{-i \omega t}) (\vc{r}^2)_\omega \frac{\d \omega}{2\pi}  = 2 \int_{-\infty}^{+\infty} 
	(1-e^{- i \omega t}) \frac{6T}{\omega} \Im \chi
	\frac{\d \omega}{2\pi}.
\end{equation*}
Подставляя $\Im \chi$, находим
\begin{equation*}
	\left\langle (\Delta \vc{r}(t))^2\right\rangle = \frac{6 T t}{M \gamma} \left(1 - \frac{1-e^{-\gamma t}}{\gamma t}\right).
\end{equation*}
Таким образом есть два предела: при $\gamma t \gg 1$:
\begin{equation*}
	\left\langle (\Delta \vc{r}(t))^2\right\rangle = 6 D t,
\end{equation*}
где $\gamma = 1/ bM$, $T b = D$.  И для $\gamma t \ll 1$ получается
\begin{equation*}
	\left\langle (\Delta \vc{r}(t))^2\right\rangle = \frac{3 T}{M} t^2 = \langle \vc{v}^2\rangle t^2,
\end{equation*}
то есть просто свободное движение со средней тепловой скоростью. 


% \subsection{Уравнение Фоккера–Планка для броуновского движения.}

% Вспоминая уравнение
% \begin{equation*}
% 	\frac{\partial n}{\partial t} = \frac{\partial }{\partial x_\alpha} \left(
% 		\tilde{A}_\alpha n + \frac{\partial (B_{\alpha \beta} n)}{\partial x_\beta} 
% 	\right),
% 	\hspace{10 mm} 
% 	B_{\alpha \beta} = \delta_{\alpha \beta} D.
% \end{equation*}


% ландау лифшиц, 
 % #12 по гайду

\setcounter{section}{14}
\section{модель Калдейры-Леггетта}
Запишем функцию Лагранжа
\begin{equation*}
	L(\dot{q}, q, \{\dot{x}_\alpha, x_\alpha\}) = \frac{M \dot{q}^2}{2} - U_0 (q) + \sum_\alpha \left(
		\frac{m \dot{x}_\alpha^2}{2} - \frac{m \omega_\alpha^2 x_\alpha^2}{2}
	\right) - q \sum_\alpha C_\alpha x_\alpha.
\end{equation*}
Теперь уравнения дваижения -- уравнения Лагранжа
\begin{equation*}
	M \ddot{q} + U'_0 (q) = - \sum_\alpha C_\alpha x_\alpha,
	\hspace{10 mm} 
	m \ddot{x}_\alpha + m \omega^2_\alpha x_\alpha = - q C_\alpha.
\end{equation*}
Решение для фононов можем записать в виде
\begin{equation*}
	x_\alpha(t) = - C_\alpha \int_{-\infty}^{t} \frac{\sin (\omega_\alpha (t-s))}{m \omega_\alpha} q(s) \d s + 
	x_\alpha(0) \cos \omega_\alpha t + \frac{\dot{x}_\alpha(0)}{\omega_\alpha} \sin (\omega_\alpha t).
\end{equation*}
Введем также функцию запаздывающего отклика
\begin{equation*}
	K_\alpha(t) = - \theta(t) \frac{C_\alpha \sin (\omega_\alpha t)}{m \omega_\alpha},
	\hspace{10 mm} 
	K_\alpha(\omega) = - \frac{C_\alpha}{m(\omega_\alpha^2 - (\omega+i \delta)^2)},
\end{equation*}
с полюсами $\omega = \pm \omega_\alpha - i \delta$, $\delta \to +0$. 

Можем ввести силу со стороны термостата на частицу
\begin{equation*}
	F(t) = \int_{-\infty}^{+\infty} K(t-s) q(s) \d s + f(t),
	\hspace{10 mm} 
	f(t) = - \sum_\alpha C_\alpha \left(
		x_\alpha(0) \cos \omega_\alpha t + \frac{\dot{x}_\alpha(0)}{\omega_\alpha} \sin \omega_\alpha t
	\right),
\end{equation*}
где $K(t)= - \sum_\alpha C_\alpha K_\alpha(t) = \theta(t) \sum_\alpha C_\alpha^2 \frac{\sin (\omega_\alpha t)}{m \omega_\alpha}$. Для Фурье-образа:
\begin{equation*}
	K(\omega) = \frac{2}{\pi} \int_{0}^{\infty}  \frac{\Omega^2}{\Omega^2- (\omega+i\delta)^2} \frac{J(\Omega)}{\Omega} \d \Omega,
	\hspace{10 mm} 
	J(\Omega) = \frac{\pi}{2} \sum_\alpha \frac{C_\alpha^2}{m \omega_\alpha} \delta(\Omega-\omega_\alpha).
\end{equation*}
Итого, для движения частицы
\begin{equation*}
	M \ddot{q} + U'_0 (q) = \int_{-\infty}^{+\infty} K(t-s) q(s) \d s + f(t).
\end{equation*}
Можем представить $K(\omega) = K_0 + i \eta \omega$ или
\begin{equation*}
	K(t) = K_0 \delta(t) - \eta \delta'(t), 
	\hspace{10 mm} 
	J(\Omega) = \eta \Omega, 
	\hspace{5 mm} 
	K_0 = \sum_\alpha \frac{C_\alpha^2}{m \omega_\alpha^2},
\end{equation*}
что приведёт для уравнений движения
\begin{equation}
\boxed{
	M \ddot{q} + U'_0 (q) = K_0 q(t) - \eta \dot{q}(t) + f(t)
}
\end{equation}
что формально соответствует трению $\eta$, перенормировке потенциала $U(q) = U_0 (q) - K_0 q^2/2$ и добавлению случайной силы $f(t)$. 

\textbf{Коррелятор}. Найдём корреляционную функцию случайной силы
\begin{equation*}
	\langle f(t) f(t')\rangle = \sum_{\alpha, \beta} C_\alpha C_\beta \langle \left(
	x_\alpha(0) \cos \omega_\alpha t + \frac{\dot{x}_\alpha(0)}{\omega_\alpha} \sin \omega_\alpha t
	\right)\left(
	x_\beta(0) \cos \omega_\beta t + \frac{\dot{x}_\beta(0)}{\omega_\beta} \sin \omega_\beta t
	\right)
	\rangle.
\end{equation*}
Так как фононы независимы друг от друга, то можем переписать в виде
\begin{equation*}
	\langle f(t) f(t')\rangle = \sum_{\alpha, \beta} C_\alpha^2 \left(
		\langle x_\alpha(0)^2\rangle \cos(\omega_\alpha t) \cos(\omega_\alpha t') + \frac{\langle \dot{x}_\alpha(0)^2\rangle}{\omega_\alpha^2} \sin(\omega_\alpha t) \sin(\omega_\alpha t')
	\right).
\end{equation*}
Здесь учли, что среднее от полной производной по времени равно нулю. Для скоростей
\begin{equation*}
	\langle \dot{x}_\alpha^2(0)\rangle = \omega_\alpha^2 \langle x_\alpha^2(0)\rangle = \frac{2}{m} \frac{1}{2} \left(\bar{n}_\alpha + \frac{1}{2}\right) \hbar \omega_\alpha = \frac{\hbar \omega_\alpha}{2m} \cth \frac{\hbar \omega_\alpha}{2T}.
\end{equation*}
И для временного коррелятора
\begin{align*}
	\langle f(t) f(t')\rangle 
	&= \sum_\alpha C_\alpha^2 \frac{\hbar}{2 m \omega_\alpha} \left(
		\cth\left(\frac{\hbar \omega_\alpha}{2 T}\right) \cos(\omega_\alpha(t-t')) - i \sin(\omega_\alpha(t-t'))
	\right) 
	\\ &= 
	\eta \int_{-\infty}^{+\infty} \hbar \Omega \cth\left(\frac{\hbar \Omega}{2T}\right) e^{-i\Omega(t-t')} \frac{\d \Omega}{2\pi}
	\\ &= 
	\int_{-\infty}^{+\infty} \langle f(t) f(t')\rangle _\Omega e^{-i\Omega(t-t')} \frac{\d \Omega}{2\pi}.
\end{align*}
Для Фурье-компоненты
\begin{equation*}
	\langle f(t) f(t')\rangle_\Omega = \hbar \hbar \Omega \cth\left(\frac{\hbar \Omega}{2T}\right) = \left\{\begin{aligned}
	    &2 \eta T, &\hbar=0\\
	    &\eta \hbar |\Omega|, &T=0.
	\end{aligned}\right.
\end{equation*}
Во временном представлении
\begin{equation*}
	\langle f(t) f(t')\rangle = \left\{\begin{aligned}
	    2 \eta T \delta(t-t'), &\hbar=0,
	    -\eta \hbar  (t-t')^{-2} /\pi, &T=0.
	\end{aligned}\right.
\end{equation*}
Это согласуется с ФДТ в виде
\begin{equation*}
	\langle f(t) f(t')\rangle_\Omega = - \Im\left(\frac{1}{\alpha(\omega)}\right) \hbar \cth\left(\frac{\hbar \omega}{2T}\right) = \frac{\hbar \alpha''(\omega)}{|\alpha(\omega)|^2} \cth \frac{\hbar \omega}{2T},
\end{equation*}
где $\alpha(\omega) f(\omega) = q(\omega)$, и тогда
\begin{equation*}
	\alpha(\omega) = \left(-M \omega^2 + M \omega_0^2 - i \eta \omega\right)^{-1},
\end{equation*}
где $U'(q) = \frac{m \omega_0^2}{2} q^2$, и значит
\begin{equation*}
	\Im(\alpha(\omega))^{-1} = - \eta \omega,
	\hspace{10 mm} 
	\langle f(t) f(t')\rangle_\Omega = \hbar \Omega \hbar \cth\left(\frac{\hbar \Omega}{2T}\right).
\end{equation*}


\setcounter{section}{15}
\section{Уравнение Линдблада}
Полный гамильтониан системы
\begin{equation*}
	\hat{H} = \hat{H}_s + \hat{H}_r + \hat{H}_{sr}.
\end{equation*}
Знаем уравнение на полную матрицу плотности
\begin{equation*}
	\frac{\partial \hat{\rho}}{\partial t}  = \frac{i}{\hbar}\left[\hat{\rho}, \hat{H}\right],
\end{equation*}
далее будем искать уравнение на $\hat{\rho}_s = \tr_r \hat{\rho}$. Будем работать в впредставление взаимодействия
\begin{equation*}
	\hat{\tilde{\rho}} = \exp\left(\frac{it}{\hbar}\left(\hat{H}_s + \hat{H}_r\right)\right) \hat{\rho} \exp\left(
		\frac{-it}{\hbar}(\hat{H}_s + \hat{H}_r)
	\right).
\end{equation*}
Можем переписать тогда в виде
\begin{equation*}
	\frac{\partial \hat{\tilde{\rho}}}{\partial t} = \frac{i}{\hbar}\left[
		\hat{\tilde{\rho}}(t_0),\, \hat{\tilde{H}}_{sr}(t)
	\right] + \left(\frac{i}{\hbar}\right)^2 \int_{t_0}^{t} \left[
		\left[
			\hat{\tilde{\rho}}(t'),\, \hat{\tilde{H}}_{sr}(t')
		\right],\, 
		\hat{\tilde{H}}_{sr}(t)
	\right] \d t'.
\end{equation*}
Будем считать, что взаимодействие адиабатически включалось, тогда можно забыть про первое слагаемое. Также считаем, что 
\begin{equation*}
	\hat{\tilde{\rho}}(t) \approx \hat{\tilde{\rho}}_s(t) \otimes \hat{\tilde{\rho}}_r(t),
	\hspace{10 mm} 
	\hat{\tilde{\rho}}(t) = \hat{\tilde{\rho}}_s(t) \otimes \hat{\tilde{\rho}}_r(-\infty),
\end{equation*}
то есть резервуар в термодинамическом равновесии и слабо взаимодействует с системой. В итоге
\begin{equation*}
	\frac{\partial \hat{\tilde{\rho}}_s(t)}{\partial t} \approx \left(\frac{i}{\hbar}\right)^2 \int_{-\infty}^{t} \d t' \tr \left(
		\left[
			\left[
				\hat{\tilde{\rho}}_s(t') \otimes \hat{\tilde{\rho}}_r(-\infty),\, \hat{\tilde{H}}_{sr}(t')
			\right],\, 
		\hat{\tilde{H}}_{sr}(t)
	\right] 
	\right).
\end{equation*}
Явно учитывая вид $\hat{H}$, можем найти коммутаторы. Введя
\begin{equation*}
	\hat{F}(t) = \sum_\alpha \gamma_\alpha \hat{b}_\alpha e^{i(\omega_s-\omega_\alpha)t},
	\hat{\tilde{H}}_{sr}(t) = \hbar\left(\hat{F}\con(t) \hat{a} + \hat{F}(t) \hat{a}\con\right),
\end{equation*}
посчитав корреляторы вида$\langle \hat{F}\con(t) \hat{F}(t')\rangle_r$, интегралы от них, и вернувшись к представлению Шрёдингера, получаем выражение
\begin{align*}
	\frac{\partial \hat{\rho}_s(t)}{\partial t} &= \frac{i}{\hbar}\left[
		\hat{\rho}_s, \hat{H}_s + \hbar \Delta_1 \hat{a} \hat{a}\con - \hbar (\Delta_1 + \Delta_2) \hat{a}\con \hat{a}
	\right] 
	\\&+ 
	\gamma_s\left(
		n(\omega_s) + 1
	\right)\left(
		2 \hat{a} \hat{\rho}_s(t) \hat{a}\con - \{\hat{\rho}_s(t),\, \hat{a}\con \hat{a}\}
	\right) + \gamma_s n(\omega_s) \left(
		2 \hat{a}\con \hat{\rho}_s (t) \hat{a} - \{\hat{\rho}_s(t),\, \hat{a} \hat{a}\con\}
	\right).
\end{align*}
где введены обозначения
\begin{equation*}
	\gamma_s = \pi D(\omega_s) |\gamma(\omega_s)|^2,
	\hspace{5 mm} 
	D(\omega) = \sum_\alpha \delta(\omega-\omega_\alpha),
\end{equation*}
а также
\begin{equation*}
	\Delta_1 = P \int_{0}^{\infty} D(\omega) |\gamma(\omega)|^2 n(\omega) \frac{1}{\omega-\omega_s} \d \omega,
	\hspace{10 mm} 
	\Delta_2 = P \int_{0}^{\infty} D(\omega) |\gamma(\omega)|^2 \frac{1}{\omega-\omega_s} \d \omega.
\end{equation*}
Таким образом добавление резервуара привело к перенормировки гамильтониана
\begin{equation*}
	\hat{H}_S = \hbar \Delta_1 + \hbar (\omega_s - \Delta_2) \hat{a}\con \hat{a},
\end{equation*}
и возникновения диссипативного члена:
\begin{equation*}
	\frac{\partial \hat{\rho}_s(t)}{\partial t} = \frac{1}{i\hbar}\left[\hat{H}_S, \hat{\rho}_s\right] + \gamma_s\left(
		n(\omega_s) + 1
	\right)\left(
		2 \hat{a} \hat{\rho}_s(t) \hat{a}\con - \{\hat{\rho}_s(t),\, \hat{a}\con \hat{a}\}
	\right) + \gamma_s n(\omega_s) \left(
		2 \hat{a}\con \hat{\rho}_s (t) \hat{a} - \{\hat{\rho}_s(t),\, \hat{a} \hat{a}\con\}
	\right).
\end{equation*}
Общий вид уравнения Линдблада:
\begin{equation*}
	\frac{\partial \hat{\rho}(t)}{\partial t} = \frac{1}{i\hbar} \left[\hat{H},\, \hat{\rho}\right] + \frac{1}{2} \sum_k \left(
		\left[\hat{L}_k \hat{\rho},\, \hat{L}\con_k\right] + 
		\left[\hat{L}_k,\, \hat{\rho} \hat{L}_k\con\right]
	\right),
\end{equation*}
сводится к полученному для $\hat{L}_1 \approx \hat{a}\con$ и $\hat{L}_2 \approx \hat{a}$.


% Т17
\setcounter{section}{16}
\section{Уравнение Смолуховского}
\subsection{Сведение к осциллятору}

Работаем примерно с уравнением 
\begin{equation*}
	\frac{\partial n}{\partial t} = D \Delta n + \div(b n \nabla U),
\end{equation*}
точнее с уравнением вида
\begin{equation*}
	\frac{\partial P}{\partial t} = \frac{D}{T} k P + \frac{D}{T} k x \frac{\partial P}{\partial x}  + D \frac{\partial^2 P}{\partial x^2} ,
\end{equation*}
где подставили потенциал $U = kx^2/2$.

Введём $g=2 \gamma T$ и $\tau = b t$, тогда можем сделать подстановку
\begin{equation*}
	P(x, \tau) = e^{-\gamma k x^2/2g} \psi(x, \tau),
\end{equation*}
получаем уравнение вида
\begin{equation*}
	\frac{1}{k} \frac{\partial \psi}{\partial \tau} = \left(
		\frac{1}{2} - \frac{x^2}{4A}
	\right) \psi + A \frac{\partial^2 \psi}{\partial x^2},
	\hspace{10 mm} 
	A = \frac{g}{2k \gamma} = \frac{T}{k}. 
\end{equation*}
Таким образом пришли к гамильтониану гармонического осциллятора, с собственными функциями в виде полиномов эрмита
\begin{equation*}
	\varphi_n (x) = \frac{1}{\sqrt{2^n n! \sqrt{2\pi A}}} H_n\left(\frac{x}{\sqrt{2A}}\right) e^{-x^2/4A}.
\end{equation*}
Итого, искомая вероятность
\begin{equation*}
	P(x, \tau) = e^{-\gamma k x^2/2g} \psi(x, \tau) = e^{-x^2/4A} \psi(x, \tau) = \sum_{n=0}^{\infty} a_n e^{- n k \tau} \varphi_0(x) \varphi_n(x).
\end{equation*}


\subsection{Забываются начальные условия}

Забавный факт:
\begin{equation*}
	\sum_{n=0}^{\infty} \frac{H_n(x) H_n(y)}{n!} \left(\frac{U}{2}\right)^n = \frac{1}{\sqrt{1-U^2}} \exp\left(
		\frac{2 u xy}{1+u} - \frac{u^2 (x-y)^2}{1-u^2}
	\right),
\end{equation*}
и воспользуемся соотношением ортогональности, что найти эволюцию от $P(x, 0) = \delta(x-x_0)$:
\begin{equation*}
	a_n = \frac{1}{\sqrt{2^n n!}} H_n\left(\frac{x_0}{\sqrt{2A}}\right) = \frac{\varphi_n(x_0)}{\varphi_0(x_0)}.
\end{equation*}
Итого, эволюция запишется в виде
\begin{align*}
	P(x, \tau) &= \sum_{n=0}^{\infty} e^{-n k \tau} \frac{\varphi_0(x)}{\varphi_0(x_0)} \varphi_n(x) \varphi_n(x_0)
	= \frac{1}{\sqrt{2\pi A}} e^{-x^2/2A} \sum_{n=0}^{\infty} \frac{1}{2^n n!} e^{-n k \tau} H_n\left(\frac{x_0}{\sqrt{2A}}\right) H_n \left(\frac{x}{\sqrt{2A}}\right) \\ 
	&= \frac{1}{\sqrt{2\pi A(1-e^{-2k \tau})}} \exp\left(
		- \frac{(x-x_0 e^{- k \tau})}{2A(1-e^{-2 k \tau})}
	\right),
\end{align*}
где подставили ту сумму с $u = e^{- k \tau}$. Таким образом начальные условия забываются!


% Теперь работаем с непрерывным пространство, поэтому ПОВМ будем искать в виде
% \begin{align*}
% 	\Pi_0 &= \int p(0, x) \kb{0}{x} \d x, \\
% 	\Pi_1 &= \int p(1, x) \kb{1}{x} \d x.
% \end{align*}
% Для удобства введём функцию $I_\Delta (x)$:
% \begin{equation*}
% 	I_\Delta (x) = \left\{\begin{aligned}
% 	    &0, &|x|>\Delta, \\
% 	    &1, &|x|<\Delta.
% 	\end{aligned}\right.
% \end{equation*}
% Тогда искомые вероятности запишутся в виде
% \begin{align*}
% 	p(1, x) &= p \cdot I_\Delta(x), \\
% 	p(0, x) &= (1-p)I_\Delta(x) + (1-I_\Delta(x)).
% \end{align*}
% \begin{align*}
% 	\hat{\Omega}_{0} &= \sqrt{p} \kb{0}{0} + \sqrt{1-p} \kb{1}{1} \\
% 	\hat{\Omega}_{1} &= \sqrt{1-p} \kb{0}{0} + \sqrt{p} \kb{1}{1} \\
% \end{align*}

% \begin{equation*}
% 	P_0 = P_1 = \frac{1}{2}\left(\frac{1}{2} - \sqrt{(1-p)(p)}\right)
% \end{equation*}

Состояние с заданной $x$-компонентной спина -- собственное для $\hat{\sigma}_x$: $\psi \sim (\pm 1, 1)$, а значит 
\begin{equation*}
	\psi(t) \sim \begin{pmatrix}
		\pm e^{\mp i \Omega t/2} \\ e^{\mp i \Omega t/2}
	\end{pmatrix},
	|\psi(t)|^2 = \const,
\end{equation*}
то есть ситема будет равновероятно наблюдаться в состояние $\ket{0}$ или $\ket{1}$. 

Для постоянных измерений будем работать с системой, вида
\begin{equation*}
	\dot{\rho}_x = - 2 \gamma \rho_x,
	\hspace{5 mm} 
	\dot{\rho}_y = -\Omega \rho_z - 2 \gamma \rho_y,
	\hspace{5 mm} 
	\dot{\rho}_z = \Omega \rho_y,
\end{equation*}
которая очевидно для $\rho_x = 1$ будет иметь решение, вида
\begin{equation*}
	\rho_x (t) = e^{-2\gamma t}.
\end{equation*}


% Т18
\setcounter{section}{17}
\section{Метастабильное состояние}


Начнём с уравнение Лиувилля, считая заданными $\vc{r}^N = (\vc{r}_1,\,  \ldots,\, \vc{r}_N)$ и $\vc{p}^N = (\vc{p}_1,\,  \ldots,\, \vc{p}_N)$
\begin{equation*}
	\dot{\vc{r}}_i = \frac{\partial H}{\partial \vc{p}_i},
	\hspace{10 mm} 
	\dot{\vc{p}}_i = -\frac{\partial H}{\partial \vc{r}_i},
\end{equation*}
где Гамильтониан запишется в виде
\begin{equation*}
	H = K(\vc{p}^N) + V(\vc{r}^N) + \Phi (\vc{r}^N),
	\hspace{10 mm} 
	K(\vc{p}^N) = \sum_{i=1}^{N} \frac{\vc{p}_i^2}{2m},
	\hspace{5 mm} 
	\Phi (\vc{r}^N) = \sum_{i=1}^{N} \varphi (\vc{r}_i).
\end{equation*}
Введём также функцию распределения $f^{[N]}(\vc{r}^N, \vc{p}^N, t)$ так чтобы $f^{[N]}(\vc{r}^N, \vc{p}^N, t) \d \vc{r}^N \d \vc{p}^N$ -- вероятность находиться в данной точке фазового пространства. Нормировка единичная. 

\textbf{Закон сохранения}.
Закон сохранения в дифференциальном виде запишется в виде
\begin{equation*}
	\frac{\partial \rho}{\partial t} + \div \vc{j} = 0,
\end{equation*}
где в нашем случае $\rho$ -- $f^{[N]}$, и $\vc{j} = \{f^{[N]} \dot{\vc{r}}_i, f^{[N]} \dot{\vc{p}}_i\}$, тогда
\begin{equation*}
	\frac{\partial f^{[N]}}{\partial t} + \sum_{i=1}^{N}\left( \frac{\partial }{\partial \vc{r}_i} \left[f^{[N]} \dot{\vc{r}}_i\right] + \frac{\partial }{\partial \vc{p}_i} \left[
				f^{[N]} \dot{\vc{p}}_i
			\right]\right) = \frac{d f^{[N]}}{d t} = 0, 
\end{equation*}
при подстановке уранений Гамильтона. 



\textbf{Редуцированная функция}. Редуцированная функция $f^{(n)}$ определяется как
\begin{equation*}
	f^{(n)} (\vc{r}^n,\, \vc{p}^n,\, t) = \frac{N!}{(N-n)!} \int f^{[N]} (\vc{r}^N,\, \vc{p}^N,\, t) \d \vc{r}^{(N-n)} \d \vc{p}^{(N-n)},
\end{equation*}
где $d \vc{r}^{(N-n)} = d \vc{r}_{n+1} \ldots \d \vc{r}_N$ и $d \vc{p}^{(N-n)} = d \vc{p}_{n+1} \ldots \d \vc{p}_N$. 

Работаем  приближение потенциального внешнего поля
\begin{equation*}
	\dot{\vc{p}}_i = \vc{X}_i + \sum_{j=1}^{N} \vc{F}_{ij} (\vc{r}_i,\, \vc{r}_j),
	\hspace{10 mm} 
	\vc{F}_{ii} = 0.
\end{equation*}
Тогда сохранение перепишется в виде
\begin{equation*}
	\frac{\partial f^{[N]}}{\partial t} + \sum_{i=1}^{N} \frac{\vc{p}_i}{m} \frac{\partial f^{[N]}}{\partial \vc{r}_i} + \sum_{i=1}^{N} \vc{X}_i \frac{\partial f^{[N]}}{\partial \vc{p}_i} = -
	\sum_{i=1}^{N} \sum_{j=1}^{N} \vc{F}_{ij} \frac{\partial f^{[N]}}{\partial \vc{p}_i}.
\end{equation*}
При редуцирование в силу ограниченности в фазовом пространстве, остаётся
\begin{equation*}
	\frac{\partial f^{(n)}}{\partial t} + \sum_{i=1}^{n} \frac{\vc{p}_i}{m} \frac{\partial f^{(n)}}{\partial \vc{r}_i} + \sum_{i=1}^{n} \vc{X}_i \frac{\partial f^{(n)}}{\partial \vc{p}_i} = - \sum_{i=1}^{n} \sum_{j=1}^{n} \vc{F}_{ij} \frac{\partial f^{(n)}}{\partial \dot{\vc{p}}_i} - \frac{N!}{(N-n)!} \sum_{i=1}^{n} \sum_{j=n+1}^{N} \int \vc{F}_{ij} \frac{\partial f^{[N]}}{\partial \vc{p}_i} \d \vc{r}^{(N-n)} \d \vc{p}^{(N-n)}.
\end{equation*}
С учетом симметричности функции распределения, последнее слагаемое можем переписать в виде
\begin{equation*}
	- \frac{N! (N-n)}{(N-n)!} \sum_{i=1}^{n} \int F_{i, n+1} \frac{\partial f^{[N]}}{\partial \vc{p}_i} \d \vc{r}^{(N-n-1)} \d \vc{p}^{(N-n-1)} \d \vc{r}_{n+1} \d \vc{p}_{n+1},
\end{equation*}
Так приходим к выражению, вида
\begin{equation*}
	\left(
		\frac{\partial }{\partial t} + \sum_{i=1}^{n} \frac{\vc{p}_i}{m} \frac{\partial }{\partial \vc{r}_i} + \sum_{i=1}^{n} \left[
			\vc{X}_i + \sum_{j=1}^{n} F_{ij}
		\right] \frac{\partial }{\partial \vc{p}_i} 
	\right) f^{(n)} = - \sum_{i=1}^{n} \int F_{i, n+1} \frac{\partial f^{(n+1)}}{\partial \vc{p}_i} \d \vc{r}_{n+1} \d \vc{p}_{n+1}.
\end{equation*}
Эта система уравнений называется цепочкой уравнений Боголюбова-Борна-Грина 
Обычно интерес представляют $n = 1, 2$, кстати $\int f^{(n)} \d \vc{r}^{n} \d \vc{p}^n = \frac{N!}{(N-n)!}$.

\subsection*{Одночастичный случай}

Для $n=1$ уравнение сведётся к
\begin{equation*}
	\left(
		\frac{\partial }{\partial t} + \frac{\vc{p}_1}{m} \frac{\partial }{\partial \vc{r}_1} + \vc{X}_1 \frac{\partial }{\partial \vc{p}_1} 
	\right) f^{(1)} (\vc{r}_1,\, \vc{p}_1,\, t) = - \int \vc{F}_{12} 
	\frac{\partial }{\partial \vc{p}_1} f^{(2)}(\vc{r}_1, \vc{p}_1, \vc{r}_2, \vc{p}_2, t) \d \vc{r}_2 \d \vc{p}_2.
\end{equation*}
В силу отсутствия корелляций между столкновениями попробуем сделать приближение
\begin{equation*}
 	f^{(2)}(\xi_1, \xi_2, t) = f^{(1)} (\xi_1^t)f^{(1)}(\xi_2^t).
\end{equation*}
Определяя
\begin{equation*}
	\tilde{\vc{F}} (\vc{r}, t) = \int \vc{F}_{12} (\vc{r}_1,\, \vc{r}_2) f^{(1)} (\vc{r}_2,\, \vc{p}_2,\, t) \d \vc{r}_2 \d \vc{p}_2,
\end{equation*}
приходим к бесстолкновительному уравнению Власова 
\begin{equation}
	\left(
		\frac{\partial }{\partial t} + \frac{\vc{p}_1}{m} \frac{\partial }{\partial \vc{r}_1}  + \left[
			\vc{X}_1 + \tilde{\vc{F}}
		\right] \frac{\partial }{\partial \vc{p}_1} 
	\right) f^{(1)} = 0.
\end{equation}
которое валидно при $n d^3 \gg 1$. 




\subsection*{Двухчастичный случай}

Для $n=2$:
\begin{equation*}
	\left(
		\frac{\partial }{\partial t} + \frac{\vc{p}_1}{m} \frac{\partial }{\partial \vc{r}_1} + \frac{\vc{p}_2}{m} \frac{\partial }{\partial \vc{r}_2} + \left[
			\vc{X}_1 + \vc{F}_{12}
		\right] \frac{\partial }{\partial \vc{p}_1} + 
		\left[
			\vc{X}_2 + \vc{F}_{21}
		\right] \frac{\partial }{\partial \vc{p}_2} 
	\right) f^{(2)} (\xi_1, \xi_2, t) =  - \int \left(
		\vc{F}_{13} \frac{\partial }{\partial \vc{p}_1} + \vc{F}_{23} \frac{\partial }{\partial \vc{p}_2} 
	\right) f^{(3)} \d \vc{r}_3 \d \vc{p}_3
\end{equation*}
Считая $n d^3 \ll 1$, можем игнорировать\footnote{
	Также будем считать, что $\vc{X}_i$ меняются слабо. 
}  трёхчастичные столкновения, тогда
\begin{equation*}
	\left(
		\frac{\vc{p}_1}{m} \frac{\partial }{\partial \vc{r}_1} + \frac{\vc{p}_2}{m} \frac{\partial }{\partial \vc{r}_2} + F_{12} \left[\frac{\partial }{\partial \vc{p}_1} - \frac{\partial }{\partial \vc{p}_2} \right]
	\right) f^{(2)} = 0.
\end{equation*}
Переходя к координатам, находим
\begin{equation*}
	\vc{F}_{12} \left(\frac{\partial }{\partial \vc{p}_1} - \frac{\partial }{\partial \vc{p}_2}  \right) f^{(2)} = - \left(
		\frac{\vc{p}_1}{m} \frac{\partial }{\partial \vc{r}_1} + \frac{\vc{p}_2}{m} \frac{\partial }{\partial \vc{r}_2} 
	\right) f^{(2)}.
\end{equation*}
Введём $\vc{r} = \vc{r}_1 - \vc{r}_2$, $\vc{R} = \frac{1}{2}(\vc{r}_1 + \vc{r}_2)$, тогда
\begin{equation*}
	\frac{\partial f^{(2)}}{\partial \vc{R}} \ll \frac{\partial f^{(2)}}{\partial \vc{r}}.
\end{equation*}

Возвращаемся к одночастичной функции, интегрируя находим
\begin{equation*}
	\left(
		\frac{\partial }{\partial t} + \frac{\vc{p}_1}{m} \frac{\partial }{\partial \vc{r}_1} + \vc{X}_1 \frac{\partial }{\partial \vc{p}_1} 
	\right) f^{(1)} (\vc{r}_1,\, \vc{p}_1,\, t) =  - \int \vc{F}_{12} \left(
		\frac{\partial }{\partial \vc{p}_1} - \frac{\partial }{\partial \vc{p}_2} 
	\right) f^{(2)} \d \xi_2 = \int 
		\left[
			\frac{\vc{p}_2}{m}-\frac{\vc{p}_2}{m}
		\right] \frac{\partial f^{(2)}}{\partial \vc{r}} \d \vc{r} \d \vc{p}_2,
\end{equation*}
продолжая с правой частью, вводя $\sub{\vc{v}}{отн} = \frac{\vc{p}_2}{m} - \frac{\vc{p}_1}{m}$ находим
\begin{equation*}
	\int \d p_2 \d^2 \sigma \d z  \sub{\vc{v}}{отн} \left(
		f^{(2)}(t_+) - f^{(2)}(t_-)
	\right).
\end{equation*}
После столкновения меняются импульсы частиц, тогда правую часть можем переписать в виде
\begin{equation}
	\int \d \vc{p}_2 \d^2 \sigma \sub{\vc{v}}{отн} \left(
		f^{(1)}(\vc{p}_2', \vc{r}, t) f^{(1)} (\vc{p}_1', \vc{r}, t) - f^{(1)} (\vc{p}_2, \vc{r}, t) f^{(1)}(\vc{p}_1, \vc{r}, t)
	\right), \text{\ \ -- интеграл столкновений}.
\end{equation}
Формально есть частицы прилетевшие и улетевшие.  К слову, $\d \vc{p}_1 \d \vc{p}_2 = \d \vc{p}_1' \d \vc{p}_2'$.





% зависимость коэффициентов теплопроводности по Т8 для полупроводников и металлов. 
% первые две пары, ещё можно после 15:20.

% после 15:20 здесь или на кафедре, 526


% У
\setcounter{section}{19}
\section{Упражнения}
\subsection{У1}

Матрица перехода
\begin{equation*}
	T = \begin{pmatrix}
	    a & b  \\
	    1-a & 1-b  \\
	\end{pmatrix} = \begin{pmatrix}
	    0.85 & 0.5  \\
	    0.15 & 0.5  \\
	\end{pmatrix}.
\end{equation*}
Собственный вектор c $\lambda=1$:
\begin{equation*}
	\vc{q} = \left(
		\frac{b}{1-a+b},\ 
		\frac{a-1}{a-1-b}
	\right) \approx \begin{pmatrix}
		0.77 \\ 0.23
	\end{pmatrix}.
\end{equation*}
Для него выполняется детальный баланс $T_{j \neq i} q_i = T_{i \neq j} q_j$. Таким образом данный процесс обратимый.



\subsection{У2}

Рассмотрим одномерное случайное блуждение с для разных $p$. Матрица перехода имеет вид
\begin{equation*}
	T = \begin{pmatrix}
		q & q & 0 & 0 & \ldots \\
		p & 0 & q & 0 & \ldots \\
		0 & p & 0 & q & \ldots \\
		0 & 0 & p & 0 & \ldots \\
		\ldots &&&&
	\end{pmatrix}
\end{equation*}
И снова ищем $T \vc{x} = \vc{x}$, что значит 
\begin{equation*}
	x_0 p = x_1 q, 
	\hspace{5 mm} 
	x_1 p = x_2 q, 
	\hspace{5 mm} 
	\ldots
\end{equation*}
откуда находим
\begin{equation*}
	\frac{1}{x_0} =
		1 + \sum_{i=0}^{N-1} \left(\frac{p}{1-p}\right)^{i}
	= 1 + \frac{1-(p/q)^{N}}{1-p/q} = \left\{\begin{aligned}
	    &\infty, &p \geq 0.5 \\
	    & \tfrac{2-3p}{1-2p}, &p<0.5
	\end{aligned}\right.
\end{equation*}
Видно, что при $p \geq 0.5$ сумма расходится и $x_0 \to 0$, а при $x_0(p < 0.5)$ конечна, гарантировано возвращаемся. 
