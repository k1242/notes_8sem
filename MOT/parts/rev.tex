% \red{Написать про первые работы по ЗЗ. Написать про ЗЗ на Tm. Написать про 2D-МОЛ.}


% Обзор существующих решений
% Здесь надо рассмотреть все существующие решения поставленной задачи, но не просто пересказать, в чем там дело, а оценить степень их соответствия тем ограничениям, которые были сформулированы в постановке задачи.

% монте-карло для зеемана
% 2d-мол
% 

Изначально ЗЗ использовался для замедления атомов I группы (например 
Li \cite{stack_ultra-cold_2010}, 
K \cite{Lee2007}, %zhao2014optimizing -- spin-flip оптимизация% down
Na \cite{zhao2014optimizing}) в силу про -- стой электронной структуры. Так, впервые встречные лазерный луч использовался для охлаждения атомарного пучка в \cite{__1981} для охлаждения Na до 1.5\,К в продольном направление в 1981 году. В работе Филлипса \cite{PhysRevLett.48.596} 1982 года уже встречается современный вид ЗЗ, также используемый для замедления атомов Na. 

Для использования ЗЗ необходим широкий циклический переход в оптическом диапазоне. Сложная электронная структура Tm не позволяет выделить циклический переход, однако некоторые переходы можно считать приближенно циклическими \cite{Kolachevsky2007}. Так, первое использование ЗЗ на длине волны $410.6$ нм для замедления атомов Tm продемонстрировано в работе \cite{Chebakov_2009}. 


В 1998 году предложен альтернативный способ формирования коллимированного атомарного пучка: 2D-МОЛ \cite{PhysRevA.58.3891}. Более компактная в отличие от ЗЗ конфигурация при соизмеримых потоках атомов делает 2D-МОЛ перспективной альтернативой. На 2D-МОЛ уже продемонстрировано получение потока замедленных атомов на элементах первой группы, например Li \cite{tiecke_high-flux_2009}, Rb \cite{ravenhall_high-flux_2021} и Na \cite{Lamporesi_2013}. Отсноительно недавно продменстрировано получение атомарного потока с помощью 2D-МОЛ на Tm \cite{golovizin_compact_2021}, однако в \cite{golovizin_compact_2021} используется встречный пучку зеемановский луч. 



