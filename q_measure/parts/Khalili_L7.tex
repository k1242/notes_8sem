\section{Лекция №7}

% Работаем с квантовым шумом
% \begin{equation*}
% 	\tilde{x} = x_s + \hat{s}_{fl} + \chi \hat{F}_{fl},
% 	\hspace{0.5cm} \Rightarrow \hspace{0.5cm}
% 	S^x_{\text{sum}} = S_{xx} + |\chi|^2 S_{FF} \geq \sub{S}{SQL} = \hbar |\chi|.
% \end{equation*}


% попробовать почистить лекцию от отражений ламп, сделать проекцию 

% \subsection*{Пример}

% Рассмотрим гравитационный детектор, зеркало подвешенное на нити. 

\subsection{Одномерный случай}



\textbf{Общий вид}. Рассмотрим линейные системы. Живём на малых отклонениях, поэтому линейность естественна. 
Пусть есть черный ящик, на который можем воздействовать через $F(t)$ и наблюдать отклик $q$
\begin{equation}
	\hat{q}(t) = \hat{q}_0 (t) + \int_{-\infty}^{+\infty} \chi(t, t') F(t') \d t'
	\label{7_I}
\end{equation}
где $\chi$ как раз отвечает за динамику системы.

\textbf{Гамильтонов подход}. При этом помним, что есть квантовая механика -- пишем гамильтониан вида
\begin{equation*}
	\hat{H} = \hat{H}_0 - F(t) \hat{Q},
	\hspace{1cm}
	i \hbar \frac{d \hat{Q}(t)}{d t} = \left[\hat{Q}(t),\, \hat{H}\right].
\end{equation*}
Тогда эволюция примет вид
\begin{equation*}
	\hat{Q}(t) = \hat{Q}_0 (t) - \frac{1}{i\hbar} \int_{-\infty}^{t}d t'\, \left[
		\hat{Q}_0(t),\, \hat{Q}_0(t')
	\right] F(t')  + \left(\frac{1}{i\hbar}\right)^2 \int_{-\infty}^{t} dt'\int_{-\infty}^{t'} dt'' \left[
		\left[
			\hat{Q}_0(t),\, \hat{Q}_0(t')
		\right], 
		\hat{Q}_0 (t'')
	\right] F(t') F(t''),
\end{equation*}
но до тех пор, пока система линейна верно, что
\begin{equation*}
	\left[
		\left[
			\hat{Q}_0(t),\, \hat{Q}_0(t')
		\right], 
		\hat{Q}_0 (t'')
	\right] = 0.
\end{equation*}
Остаётся выражение, вида
\begin{equation}
	\hat{Q}(t) = \hat{Q}_0 (t) - \frac{1}{i\hbar} \int_{-\infty}^{t}d t'\, \left[
		\hat{Q}_0(t),\, \hat{Q}_0(t')
	\right] F(t').
	\label{7_II}
\end{equation}
Таким образом динамика принципиально завязаны эволюция и шумы. 

Сравнивая \eqref{7_I} и \eqref{7_II}, видим что $Q \equiv q$ и
\begin{equation}
	\chi(t, t') = \theta(t-t') \frac{i}{\hbar} \left[
		\hat{q}_0 (t),\,  \hat{q}_0(t')
	\right],
	\hspace{5 mm} 
	\Leftrightarrow
	\hspace{5 mm} 
	\left[q_0 (t),\, q_0 (t')\right] = i \hbar \left(
		\chi(t', t) - \chi(t, t')
	\right),
\end{equation}
что гордо именуется \textit{формулой Кубо} для линейного отклика. 

\textbf{Корреляционная функция}. Введём корреляционную функцию, вида
\begin{equation*}
	B(t, t') = \frac{1}{2}\left\langle 
		\hat{q}_0(t) \hat{q}_0(t') + \hat{q}_0 (t') \hat{q}_0 (t)
	\right\rangle.
\end{equation*}
Также определим
\begin{equation*}
	\hat{\Q} = \int_{-\infty}^{+\infty} Q(t) \hat{q}_0 (t)  \d t,
	\hspace{10 mm} 
	\hat{\Q}\con = \int_{-\infty}^{+\infty} Q^*(t) \hat{q}_0 (t) \d t.
\end{equation*}
Для неё верно, что
\begin{equation*}
	0 \leq \bk{\psi}[\hat{\Q}\con \hat{\Q}]{\psi},
	\hspace{10 mm} 
	\hat{\Q} \ket{\psi} = \ket{\varphi},
	\hspace{5 mm} 
	\bra{\psi} \Q\con = \bra{\varphi}.
\end{equation*}
Для произвольных операторов $\hat{A},\, \hat{B}$ верно, что $\hat{A} \bar{B} = \hat{A} \circ \hat{B} + \frac{1}{2} \left[\hat{A}, \hat{B}\right]$, где $\hat{A} \circ \hat{B} = 
\frac{1}{2}\{\hat{A}, \hat{B}\} = \frac{1}{2}\left(
		\hat{A} \hat{B} + \hat{B} \hat{A}
\right)$. Подставляя и собирая всё вместе,
 находим
 \begin{equation*}
 	\bk{\psi}[\hat{\Q}\con \hat{\Q}]{\psi} = \int_{-\infty}^{+\infty} \langle \hat{q}_0 (t) \hat{q}_0 (t') \rangle =  \int_{-\infty}^{+\infty} Q^* (t') Q(t') \left(
 		B(t, t') + \frac{i\hbar}{2} \left(\chi(t', t) - \chi(t, t')\right)
 	\right) \geq 0,
 \end{equation*}
 что верно $\forall Q$ и таким образом составляет соотношение неопределенности, связывающее $\chi$ и $B$.

\textbf{Стационарность}. 
 На этом заканчивается общей рассмотрение, поэтому ограничимся частным случаем стационарности. \texttt{Это высокая стена, но увидели в стороночке дверь и сразу же туда ныряем}. Под стационарностью имеем ввиду, что
\begin{equation*}
 	\chi(t, t') = \int_{-\infty}^{+\infty} \chi(\omega) e^{- i \omega(t-t')} \frac{\d \omega}{2\pi},
 	\hspace{10 mm} 
 	B(t, t')= \int_{-\infty}^{+\infty} S(\omega) e^{- i \omega (t-t')} \frac{\d \omega }{2\pi}.
\end{equation*}
Здесь $S(\omega)$ -- спектральная плотность.
Знаем из фурье-анализа, что
\begin{equation*}
	\int_{-\infty}^{+\infty} X^*(t) Y(t-t') Z(t') \d t \d t' = \int_{-\infty}^{+\infty} X^*(\omega) Y(\omega) Z(\omega) \frac{\d \omega}{2\pi}.
\end{equation*}
Подставляя фурье-представления в соотношение неопределенности, находим
\begin{equation*}
	\int_{-\infty}^{+\infty} |Q(\omega)|^2  \frac{\d \omega}{2\pi} \times  S(\omega) \geq \frac{i \hbar}{2} \int_{-\infty}^{+\infty} |Q(\omega)|^2 \frac{\d \omega}{2\pi} \times 2 i \Im \chi(\omega),
\end{equation*}
что делает наш интеграл одномерным и даёт возможность свести всё к неотрицательности аргументов
\begin{equation*}
	S(\omega) \geq - \hbar \Im \chi(\omega).
\end{equation*}
Зная, что для спектральной плотности $S^*(\omega) = S(-\omega) = S(\omega)$ и $\xi(-\omega) = \chi^*(\omega)$, можем переписать неравенство в виде
\begin{equation*}
	S(\omega) \geq - \hbar \Im \chi(-\omega) = \hbar \Im \chi(\omega).
\end{equation*}
Таким образом приходим к выражению 
\begin{equation}
	S(\omega) \geq \hbar |\Im \chi(\omega)|,
\end{equation}
что уже достаточно занятно. 

% сюжет про флуктуационную теорему, шум Найквиста, ...




\subsection{Многомерный случай}


Обобщим предыдущую секцию на многомерный случай, в частности двухмерный, но вообще не обязательно этим ограничиваться -- систему с $N$ интерфейсами, которые как-то друг с другом связаны
\begin{equation*}
	\hat{q}_j (t) = \hat{q}_{j0} (t) + \sum_{k=1}^{N} \int_{-\infty}^{\infty} \chi_{jk} (t, t') F_k(t') \d t'.
\end{equation*}
Переписывая это в рамках гамильтонова подхода, с учётом линейности, можем найти, что
\begin{equation*}
	\chi(t, t') = \theta(t-t') \frac{i}{\hbar}\left[
		\hat{q}_{j0}(t), \hat{q}_{k0} (t')
	\right],
	\hspace{10 mm} 
	\left[\hat{q}_{j0}(t), \hat{q}_{k0} (t')\right] = i \hbar \left(
		\chi_{kj}(t', t) - \chi_{jk}(t, t')
	\right).
\end{equation*}
Если раньше это было не ноль только для ненулевой мнимой части -- наличия трения в системе, то теперь ситуация немного другая. 





