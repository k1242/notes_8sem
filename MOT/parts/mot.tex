
\startp
\upar{Динамика количества атомов в МОЛ}
Количество атомов в ловушке $N$ во время загрузки может быть оценено уравнением \cite{vlad}
\begin{equation}
	\frac{d N}{d t} = \sub{\Phi}{load} - \gamma N - \beta \int_V n(\vc{r}, t)^2 \d^3 \vc{r},
\end{equation}
где $\gamma$ -- коэффициент линейных потерь, обусловленных столкновениями с буферным газом, $\beta$ -- скорость неэластичных бинарных столкновений, $n(\vc{r}, t)$ -- концентрация атомов, $V$ -- объем атомного облака, $\sub{\Phi}{load} = \eta \sub{\Phi}{sol}$ -- поток атомов после замедлителя со скоростью $v < \sub{v}{cap}$. Зависимость $n(\vc{r})$ в каждый момент времени может быть аппроксимирована гауссовой функцией с дисперсиями $(w_x, x_y, w_z)$, что позволяет явно посчитать интеграл:
\begin{equation}
	\frac{d N}{d t}  = \sub{\Phi}{load}  - \gamma N - \tilde{\beta} N^2,
	\hspace{10 mm} 
	\tilde{\beta} = \frac{\beta}{(2\pi)^{3/2}} \frac{1}{w_x w_y w_z}.
	\label{eq:mot1}
\end{equation}
Физический смысл $w_{x, y,z}$ -- радиус атомного облака по уровню $e^{-1}$.

Решая уравнение \eqref{eq:mot1}, можем найти зависимость $N(t)$:
\begin{equation}
	N = \frac{\sub{\Phi}{load}}{\gamma} \left(
		\frac{1}{2} + \frac{\mu}{\th \mu \gamma t }
	\right)^{-1}
	,
	\hspace{10 mm} 
	\mu \overset{\mathrm{def}}{=}  \frac{1}{2} \sqrt{1 + 4 \frac{\beta \sub{\Phi}{load}}{\gamma^2}}.
\end{equation}
Для достаточно большого времени загрузки $\gamma\, \mu\,  \sub{t}{load} \gg 1$ можем рассматривать стационарное значение и выразить связь $N$ с $\eta$:
\begin{equation}
	N = \frac{\gamma}{2\beta}
	\left(\sqrt{1 + 4 \frac{\beta \eta \Phi}{\gamma^2}}-1\right),
	\hspace{10mm} 
	\eta = \frac{\gamma}{\sub{\Phi}{sol}} N + \frac{\beta^2}{\sub{\Phi}{sol}} N^2.
	\label{eq:etaN}
\end{equation}
Таким образом задача оптимизации количества атомов в магнитооптической ловушке может быть сведено к оптимизации безразмерного параметра $\eta$.  
