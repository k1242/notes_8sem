\subsection{У1}

Матрица перехода
\begin{equation*}
	T = \begin{pmatrix}
	    a & b  \\
	    1-a & 1-b  \\
	\end{pmatrix} = \begin{pmatrix}
	    0.85 & 0.5  \\
	    0.15 & 0.5  \\
	\end{pmatrix}.
\end{equation*}
Собственный вектор c $\lambda=1$:
\begin{equation*}
	\vc{q} = \left(
		\frac{b}{1-a+b},\ 
		\frac{a-1}{a-1-b}
	\right) \approx \begin{pmatrix}
		0.77 \\ 0.23
	\end{pmatrix}.
\end{equation*}
Для него выполняется детальный баланс $T_{j \neq i} q_i = T_{i \neq j} q_j$. Таким образом данный процесс обратимый.



\subsection{У2}

Рассмотрим одномерное случайное блуждение с для разных $p$. Матрица перехода имеет вид
\begin{equation*}
	T = \begin{pmatrix}
		q & q & 0 & 0 & \ldots \\
		p & 0 & q & 0 & \ldots \\
		0 & p & 0 & q & \ldots \\
		0 & 0 & p & 0 & \ldots \\
		\ldots &&&&
	\end{pmatrix}
\end{equation*}
И снова ищем $T \vc{x} = \vc{x}$, что значит 
\begin{equation*}
	x_0 p = x_1 q, 
	\hspace{5 mm} 
	x_1 p = x_2 q, 
	\hspace{5 mm} 
	\ldots
\end{equation*}
откуда находим
\begin{equation*}
	\frac{1}{x_0} =
		1 + \sum_{i=0}^{N-1} \left(\frac{p}{1-p}\right)^{i}
	= 1 + \frac{1-(p/q)^{N}}{1-p/q} = \left\{\begin{aligned}
	    &\infty, &p \geq 0.5 \\
	    & \tfrac{2-3p}{1-2p}, &p<0.5
	\end{aligned}\right.
\end{equation*}
Видно, что при $p \geq 0.5$ сумма расходится и $x_0 \to 0$, а при $x_0(p < 0.5)$ конечна, гарантировано возвращаемся. 