\section*{Семинар №2}


Рассмотрим частицы в силе Лоренца
\begin{equation*}
	\hat{F} = q \left(
		\vc{E} + \frac{1}{c} \left[ \vc{v} \times  \vc{B} \right]
	\right).
\end{equation*}
Запишем
\begin{equation*}
	\left(
		\frac{\partial }{\partial t} + \vc{v} \frac{\partial }{\partial \vc{r}} - e \left[
			\vc{E} + \frac{1}{c} \left[\vc{v} \times  \vc{B}\right]
		\right] \frac{\partial }{\partial \vc{p}} 
	\right) f = 0.
\end{equation*}
Учтём, что $M \gg m$, тогда $f_i = f_{io}$  и $f = f_0 + \delta f$. В линейном отклике
\begin{equation*}
	\rho = - e \int \delta f \d \Gamma,
	\hspace{10 mm} 
	\vc{j} = - e \int \vc{v} \, \delta f  \d \Gamma,
\end{equation*}
где уже учли отсутствие вклада равновесных слагаемых. Равновесная функция распределения $f_0 = f_0 (\varepsilon(\vc{p}))$, подставляя, находим
\begin{equation*}
	\frac{\partial \delta f}{\partial t} + \vc{v} \frac{\partial \delta f}{\partial \vc{r}} - e\left(
		\vc{E} + \frac{1}{c}  \left[\vc{v} \times  \vc{B}\right]
	\right) \frac{\partial f_0}{\partial \vc{p}} = 0,
	\hspace{10 mm} 
	\frac{\partial f_0}{\partial \vc{p}} = \vc{v} \frac{\partial f_0}{\partial \varepsilon}.
\end{equation*}
Итого остаётся 
\begin{equation*}
	\frac{\partial \delta f}{\partial t}  + \vc{v} \frac{\partial \delta f}{\partial \vc{r}} = e (\vc{v} \cdot \vc{E}) \frac{\partial f_0}{\partial \varepsilon} - \frac{\delta f}{\tau},
\end{equation*}
где добавление $- \frac{\delta f}{\tau}$ приводит к причинности дальнейшего выражения $+ i \delta = + i / \tau$. 

Рассмотрим $\vc{E} = E_{k, \omega}e^{i\left(\vc{k} \vc{r} - \omega t\right)}$, и тогда $\delta f = \delta f_{k, \omega} e^{i\left(\vc{k} \vc{r} - \omega t\right)}$, подставляя находим 
\begin{equation*}
	(\omega - \vc{k} \vc{v}) \delta f_{k \omega} = i e (\vc{v} \cdot \vc{E}_{k \omega}) \frac{\partial f_0}{\partial \varepsilon},
\end{equation*}
и выражение на Фурье образ первой поправки
\begin{equation*}
	\delta f_{k \omega} (\vc{p}) = \frac{i e (\vc{v} \cdot \vc{E}_{k, \omega})}{\omega - \vc{k} \cdot \vc{v} + i \delta} \frac{\partial f_0}{\partial \varepsilon}.
\end{equation*}
Вспомним, что $\vc{D} = \vc{E} + 4 \pi \vc{P}$, тогда
\begin{equation*}
	- e \int \frac{i e \vc{v} (\vc{v} \cdot \vc{E}_{k \omega})}{\omega - \vc{k} \vc{v} + i \delta} \frac{\partial f_0}{\partial \varepsilon} \d \Gamma = j_{k \omega} = - i \omega P_{k \omega},
\end{equation*}
откуда находим поляризацию
\begin{equation*}
	P_{\alpha, k \omega} = \frac{e^2}{\omega} E_\beta \int \frac{v_\alpha v_\beta}{\omega - \vc{k} \cdot \vc{v}} \frac{\partial f_0}{\partial \varepsilon} \d \Gamma = \chi_{\alpha \beta} E_\beta.
\end{equation*}
Для трёхмерного случая итого находим
\begin{equation*}
	D_\alpha = E_\alpha + 4 \pi P_\alpha = \varepsilon_{\alpha \beta} E_\beta,
	\hspace{0.5cm} \Rightarrow \hspace{0.5cm}
	\varepsilon_{\alpha \beta} (\omega, \vc{k}) = \delta_{\alpha \beta} + \frac{4 \pi e^2}{\omega} \int \frac{v_\alpha v_\beta}{\omega - \vc{k} \cdot \vc{v}}	\frac{\partial f_0}{\partial \varepsilon} \d \Gamma.
\end{equation*}
Перейдём к переменным $\hat{\vc{v}} = \vc{v}/v$, $\hat{\vc{k}} = \vc{k} / k$, переписываем интеграл в виде
\begin{equation*}
	\varepsilon_{\alpha \beta} = \delta_{\alpha \beta} + \frac{4 \pi e^2}{\omega^2} \int \d \Gamma\ s v^2 \frac{\partial f_0}{\partial \varepsilon} \int \frac{\d \Omega}{4 \pi} \frac{\hat{v}_\alpha \hat{v}_\beta}{s - \hat{\vc{k}} \cdot \hat{\vc{v}} + i \delta}	
\end{equation*}
где $s = \omega / k v$. Итого усредняя находим\footnote{
	см. Бурмистров, считается $A(s)$, $B(s)$.
} 
\begin{equation*}
	\int \frac{\d \Omega}{4 \pi} \frac{\hat{v}_\alpha \hat{v}_\beta}{s - \hat{\vc{k}} \cdot \hat{\vc{v}} + i \delta} = 
	A(s) \delta_{\alpha \beta} + B(s) \hat{k}_\alpha \hat{k}_\beta f,
\end{equation*}
Итого находим выражение в виде
\begin{equation*}
	\varepsilon_{\alpha \beta}(\omega, \vc{k}) = \varepsilon_l \frac{k_\alpha k_\beta}{k^2} + \varepsilon_t \left(
		\delta_{\alpha \beta} - \frac{k_\alpha k_\beta}{k^2}
	\right).
\end{equation*}
Считая $\vc{E} = \vc{E}_e + \vc{E}_t$, где $\vc{E}_l = (\vc{E} \cdot \vc{k}) \vc{k} / k^2$ и $\vc{E}_t = \vc{E}- \vc{E}_l$, найдём
\begin{equation*}
	D_\alpha = \varepsilon_{\alpha \beta} E_\beta = D_{l\alpha} + D_{t \alpha}, 
	\hspace{10 mm} 
	D_{l \alpha} = \varepsilon_l E_{l \alpha},
	\hspace{5 mm} 
	D_{t \alpha} = \varepsilon_t D_{t \alpha}.
\end{equation*}
Рассмотрим теперь только $\vc{E}_l \propto \vc{k}$, тогда
\begin{equation*}
	\rot \vc{E} = i \left[
		\vc{k} \times  \vc{E}_l
	\right] = 0 = - \frac{i \omega}{c} \vc{B},
	\hspace{0.5cm} \Rightarrow \hspace{0.5cm}
	\rot \vc{B} = 0 = \frac{i \omega}{c} \vc{D},
	\hspace{0.5cm} \Rightarrow \hspace{0.5cm}
	\vc{D}_l = 0.
\end{equation*}
Таким образом $\varepsilon_l (\omega, \vc{k}) = 0$  задаёт дисперсию продольных плазменных колебаний. Для поперечных плазменных колебаний уравнение примет вид
\begin{equation*}
	k^2 = \frac{\omega^2}{c^2} \varepsilon_t (\omega, \vc{k}).
\end{equation*}




\subsection*{2D}

Специфично для двухмерного случая
\begin{equation*}
	\left(
		\frac{\partial }{\partial t} + \vc{v} \frac{\partial }{\partial \vc{r}} + \vc{F} \frac{\partial }{\partial \vc{p}} 
	\right) f = 0,
\end{equation*}
поле и поправка
\begin{equation*}
	\vc{E} = \vc{E}_{k \omega} e^{i(\vc{k} \vc{r} - \omega t)},
	\hspace{5 mm} 
	\delta f_{\vc{k}, \omega} = \frac{i e (\vc{v} \cdot \vc{E}_{k \omega}(z=0))}{\omega - \vc{k} \cdot \vc{v} + i \delta} \frac{\partial f_0}{\partial \varepsilon}.
\end{equation*}
Рассмотрим выражение для $\rho$, которая имеет принципиально двухмерный характер
\begin{equation*}
	\rho_{\omega \vc{k}} = - i e^2 \int \frac{i e (\vc{v} \cdot \vc{E}_{k \omega}(z=0))}{\omega - \vc{k} \cdot \vc{v} + i \delta} \frac{\partial f_0}{\partial \varepsilon} \d \Gamma.
\end{equation*}
Отдельно найдём
\begin{equation*}
	I(\omega, \vc{k}) = \int \frac{\vc{v}}{\omega - \vc{k} \vc{v} + i \delta} \frac{\partial f_0}{\partial \varepsilon}  \d \Gamma = \frac{\vc{k}}{k^2} \int \frac{\partial f_0}{\partial \varepsilon} A(s) \d \Gamma,
	\hspace{10 mm} 
	A(s) = \int_{0}^{2\pi} \frac{\d \varphi}{2\pi} \frac{\cos \varphi-s+s}{s- \cos \varphi + i \delta}.
\end{equation*}
Сделаем замену
\begin{equation*}
	x = \tg \frac{\varphi}{2}, \ \ \d \varphi = \frac{2 \d x}{1+x^2},
	\hspace{0.5cm} \Rightarrow \hspace{0.5cm}
	A(s) = \left\{\begin{aligned}
	    &-1 + \tfrac{s}{\sqrt{s^2-1}}, &s>1 \\
	    &-1  - \tfrac{i s}{\sqrt{1-s^2}}, &s>1.
	\end{aligned}\right.
\end{equation*}
где мнимая часть связана с затуханием Ландау. Подставляя в плотность
\begin{equation*}
	\rho_{\omega k} = - \frac{i e^2}{k^2} \left(
		\vc{k} \cdot \vc{E}_{\omega k}
	\right) \int \frac{\partial f_0}{\partial \varepsilon} A(s) \d \Gamma.
\end{equation*}
Расписывая уравнения Максвелла	
\begin{equation*}
	\frac{\partial E_x}{\partial x}  + \frac{\partial E_y}{\partial y} + \frac{\partial E_z}{\partial z} = 4 \pi \rho (x, y) \delta(z),
	\hspace{10 mm} 
	\vc{E} = - \nabla \varphi,
	\hspace{5 mm} 
	\varphi = \varphi_{k \omega} (z) e^{i (\vc{k} \vc{r} -  \omega t)}.
\end{equation*}




