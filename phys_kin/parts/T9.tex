% со вторым заданием всё разбирать не будем

Пишем уравнение Больцмана для двухчастичных столкновений
\begin{equation*}
	\frac{\partial f}{\partial t}  + \vc{v} \frac{\partial f}{\partial \vc{r}}  + \vc{F} \cdot \frac{\partial f}{\partial \vc{p}} = \int d^3 p_1\,   \sub{v}{отн} \d \sigma_{p p_1} (f' f_1' - f f_1).
\end{equation*} 
Столкновения упругие: $\vc{p} + \vc{p}_1 = \vc{p}' + \vc{p}_1'$.
Умножим уравнение на некоторую $\varphi(\vc{p})$ и проинтегрируем по импульсам:
\begin{equation*}
	\int d^3 \vc{p}\ \varphi(\vc{p}) \ldots = \frac{1}{4} \iint d^3\vc{p}\, d^3\vc{p}_1\, \sub{v}{отн} \d \sigma_{p p_1} (f' f_1' -f f_1) \times \left(
		\varphi(\vc{p}) + \varphi(\vc{p}_1) - \varphi(\vc{p}') - \varphi(\vc{p}_1')
	\right).
\end{equation*}

\textbf{Частицы}. 
Можем вспомнить законы сохранения и подставить $\varphi(\vc{p}) = \left[1,\, p_\alpha,\, \frac{p^2}{2m}\right]$, получим следующее выражение для $\varphi(p)=1$
\begin{equation}
	\frac{1}{n} \int d^3p\ \vc{v} f = \langle \vc{v}\rangle = \vc{u}(t, \vc{r}),
	\hspace{5 mm} 
	\int d^3 \vc{p}\ f = n(t, \vc{r}),
	\hspace{0.5cm} \Rightarrow \hspace{0.5cm}
	\frac{\partial n}{\partial t} + \div(n \vc{u}) = 0.
\end{equation}

\textbf{Импульс}. 
Теперь рассмотрим $\varphi(\vc{p}) = m v_\alpha$:
\begin{equation*}
	\frac{\partial }{\partial t} (m n u_\alpha) + \frac{\partial \Pi_{\alpha \beta}}{\partial x_\beta} = n F_\alpha,
\end{equation*}
где ввели тензор потока импульса
\begin{align*}
	\Pi_{\alpha \beta} &\overset{\mathrm{def}}{=} \int d^3\vc{p}\ m v_\alpha v_\beta f = n m \langle v_\alpha v_\beta \rangle 
	= m n u_\alpha u_\beta + mn \langle (v_\alpha - u_\alpha)(v_\beta-u_\beta)\rangle 
	= \\ &=
	m n u_\alpha u_\beta + mn \left\langle 
		(v_\alpha - u_\alpha)(v_\beta-u_\beta) - \tfrac{1}{3} \delta_{\alpha \beta} (\vc{v}-\vc{u})^2
	\right\rangle + \tfrac{1}{3} \delta_{\alpha \beta} mn  \langle (\vc{v}-\vc{u})^2\rangle
	= \\ &= 
	m n u_\alpha u_\beta  + P \delta_{\alpha \beta} - \sigma_{\alpha \beta}',
\end{align*}
где ввели для удобства величины тензора вязких напряжений и давления
\begin{equation*}
	\sigma_{\alpha \beta}' =  - mn \left\langle 
		(v_\alpha - u_\alpha) (v_\beta-u_\beta) - \tfrac{1}{3} \delta_{\alpha \beta} (\vc{v} - \vc{u})^2
	\right\rangle,
	\hspace{10 mm} 
	P  = \tfrac{1}{3} mn \left\langle (\vc{v}-\vc{u})^2\right\rangle.
\end{equation*}
Тогда уравнение перепишется в виде
\begin{equation*}
	\frac{\partial (mn u_\alpha)}{\partial t} + \frac{\partial P}{\partial x_\alpha} + \frac{\partial (mn u_\alpha u_\beta)}{\partial x_\beta}  = \frac{\partial \sigma_{\alpha\beta}'}{\partial x_\beta} + n F_\alpha.
\end{equation*}
В силу уравнения непрерывности часть слагаемых сократится, тогда 
\begin{equation*}
	m n \left(
		\frac{\partial u_\alpha}{\partial t}  + u_\beta \frac{\partial u_\alpha}{\partial x_\beta} 
	\right) + \frac{\partial p}{\partial x_\alpha}  = \frac{\partial \sigma_{\alpha \beta}'}{\partial x_\beta} + n F_\alpha.
\end{equation*}
Введем плотность $\rho \overset{\mathrm{def}}{=}  m n$, тогда
\begin{equation}
	\frac{d }{d t}  = \frac{\partial }{\partial t}  + (\vc{u} \cdot \vc{\nabla}),
	\hspace{10 mm} 
	\rho \frac{d u_\alpha}{d t} = \rho \left(
		\frac{\partial \vc{u}}{\partial t} + (\vc{u} \cdot \vc{\nabla}) \vc{u}
	\right)_\alpha = - \frac{\partial P}{\partial x_\alpha} + \frac{\partial \sigma_{\alpha \beta}'}{\partial x_\beta}  + n F_\alpha.
\end{equation}

\textbf{Энергия}. Подставим $\varphi(\vc{p}) = \frac{p^2}{2m}$, тогда
\begin{equation}
	\frac{\partial \varepsilon}{\partial t} + \div \vc{q} = n (\vc{F} \cdot \vc{u}),
	\hspace{10 mm} 
	\varepsilon = n \left\langle \tfrac{m \vc{v}^2}{2}\right\rangle,
	\hspace{5 mm} 
	\vc{q} = n \left\langle \vc{v} \tfrac{m \vc{v}^2}{2}\right\rangle.
\end{equation}
Вообще малостью будем считать $\vc{v} - \vc{u}$, тогда
\begin{align}
	\varepsilon &= \frac{mn}{2} \left(
		\left\langle \left(\vc{v}-\vc{u}\right)^2\right\rangle + \vc{u}^2
	\right) = \frac{3}{2} P + \frac{1}{2} mn u^2, \\
	q_\alpha  &=  u_\alpha \left( \frac{5}{2} P + \frac{1}{2} n m u^2\right) + q_\alpha' - \sigma_{\alpha \beta}' u_\beta,
\end{align}
где диссипативная часть плотности потока энергии $\vc{q}'$ имеет вид
\begin{equation*}
	q_\alpha' = \frac{mn}{2} \langle (v-u)_\alpha (\vc{v}-\vc{u})^2\rangle.
\end{equation*}
Подставляя и сокращая, находим
\begin{equation*}
	\frac{3}{2} \frac{\partial P}{\partial t}  + \frac{5}{2} P \div \vc{u} + \frac{3}{2} u_\alpha \frac{\partial P}{\partial x_\alpha} = \sigma_{\alpha \beta}' \frac{\partial u_\beta}{\partial x_\alpha}  - \div \vc{q}'.
\end{equation*}
Объединяя в $\frac{d }{d t} $, можем переписать в виде
\begin{equation}
	\frac{3}{2} \frac{\partial P}{\partial t}  + \frac{5}{2} P \div \vc{u} = \sigma_{\alpha \beta}' \frac{\partial u_\beta}{\partial x_\alpha} - \div \vc{q}'.
\end{equation}

\textbf{Температура}. Введём для одноатомного газа
\begin{equation*}
	\frac{3}{2} T = \left\langle  \frac{m (\vc{v}-\vc{u})^2}{2}\right\rangle,
	\hspace{0.5cm} \Rightarrow \hspace{0.5cm}
	P(t, \vc{r}) = n(t, \vc{r}) T(t, \vc{r}).
\end{equation*}
Снова учитывая уравнение непрерывности можем перписать выражение в виде
\begin{equation}
	\frac{3n}{2} \frac{d T}{d t} + n T \div \vc{u} = \sigma_{\alpha \beta}' \frac{\partial u_\alpha}{\partial x_\alpha}  - \div \vc{q}',
\end{equation}
что является уравнением на температуру. Три основные уравнения -- (9.1), (9.3), (9.7), которые мы получили из уравнения Больцмана. Осталось замкнуть эти уравнения. 

\textbf{$\tau$-приближение}. Будем решать систему уравнение в $\tau$-приближение, когда $\sub{I}{ст} = - \frac{f-f_0}{\tau}$. Выбираем функцию $f_0$ в виде локально равновестного распределения
\begin{equation*}
	f_0 = \frac{n(t, \vc{r})}{\left(2 \pi m T(t, \vc{r})\right)^{3/2}} \exp\left(
		- \frac{(\vc{p}-m \vc{u}(t, \vc{r}))^2}{2 m T(t, \vc{r})}
	\right),
	\hspace{10 mm} 
	\int f_0 \d^3 \vc{p} = n (t, \vc{r}).
\end{equation*}
Мы учли, что длина пробега $l \ll L$, характерных размеров системы. Число $\text{Kn} = \frac{l}{L}$ -- число Кнудсена, которое и характеризует то что достаточно часто происходят столкновения. 

Подставляя в левую часть $f_0$, находим поправку
\begin{equation*}
	\frac{\partial f_0}{\partial t}  + \vc{v} \frac{\partial f_0}{\partial \vc{r}} + \vc{F} \frac{\partial f_0}{\partial \vc{p}} = - \frac{\delta f}{\tau}.
\end{equation*}
Верно, что $\sigma_{\alpha\beta}'$ будет зануляться для локально равновесного распределения. После выражения временных производных из бездиссипативных уравнений, получается
\begin{equation*}
	\delta f = - \tau \frac{f_0}{T} \left(
		(\vc{v} \cdot \vc{\nabla} T) \left(
			\frac{m (v')^2}{2T} - \frac{5}{2}
		\right) + \frac{m}{2} v_\alpha' v_\beta' U_{\alpha \beta}
	\right),
\end{equation*}
где ввели переменные
\begin{equation*}
	\vc{v}' = \vc{v} - \vc{u},
	\hspace{10 mm} 
	U_{\alpha\beta} = \frac{\partial u_\alpha}{\partial x_\beta} + \frac{\partial u_\beta}{\partial x_\alpha} - \frac{2}{3} \delta_{\alpha \beta} \div \vc{u}.
\end{equation*}

\subsection*{Кинетические коэффициенты}

\textbf{Теплопроводность}. Главное здесь найти
\begin{equation*}
	q_\alpha' = \int d^3 \vc{p}\, 
	\delta f \, v_\alpha' \frac{m (v')^2}{2}.
\end{equation*}
Подставляя поправку $\delta f$, находим
\begin{equation*}
	q_\alpha' = - \nabla_\alpha T \int d^3 \vc{p}\ \frac{f_0 \tau m}{6 T}\left(
		\frac{m (v')^6}{2T} - \frac{5 (v')^4}{2}
	\right) = - \kappa n_\alpha T,
\end{equation*}
где \textit{коэффициент теплопроводности} $\kappa$ равен
\begin{equation*}
	\kappa = \frac{nm}{6T} \left\langle 
		\tau(v') \left(
			\frac{m (v')^6}{2T} - \frac{5 (v')^4}{2} 
		\right)
	\right\rangle.
\end{equation*}
Нам понадобятся интегралы, вида
\begin{equation*}
	\langle (v')^{2n} \rangle = \frac{(T/m)^n}{(2\pi)^{3/2}} \int d^3 \vc{x}\  x^{2n} e^{-x^2/2}
	=  (2n+1)!! \left(\frac{T}{m}\right)^n.
\end{equation*}
Собирая всё вместе, находим
\begin{equation}
	\kappa = \frac{5 n T}{2m}\tau,
\end{equation}
где мы считали $\tau = \const$.


\textbf{Тензор вязких напряжений}. Для тензора вязких напряжений
\begin{equation*}
	\sigma_{\alpha \beta}' =  - m \int d^3 \vc{p}\ \delta f \left(
		v_\alpha' v_\beta' - \frac{\delta_{\alpha\beta}}{3} (v')^2
	\right) = \frac{n m^2}{2T} U_{\mu \nu} \left\langle 
		\tau(v') v_\mu' v_\nu' \left(
			v_\alpha' v_\beta' - \frac{\delta_{\alpha \beta}}{3} (v')^2
		\right)
	\right\rangle.
\end{equation*}
Вспоминаем теорию поля
\begin{equation*}
	\langle v_\alpha' v_\beta' v_\mu' v_\nu' \rangle = \frac{1}{15} \left(
		\delta_{\alpha \beta} \delta_{\mu \nu} + \delta_{\alpha \mu} \delta_{\alpha \beta} + \delta_{\alpha \nu} \delta_{\beta \mu}
	\right).
\end{equation*}
Получается тензор, вида
\begin{equation*}
	\sigma_{\alpha \beta}' = \frac{n m^2}{2T} \cdot \frac{\langle \tau (v')^4\rangle_0}{15} 2 U_{\alpha \beta} = \eta \left(
		\frac{\partial u_\alpha}{\partial x_\beta} + \frac{\partial u_\beta}{\partial x_\alpha} - \frac{2}{3} \delta_{\alpha \beta} \div \vc{u}
	\right),
\end{equation*}
где \textit{коэффицент вязкости} $\eta$ равен
\begin{equation}
	\eta = \frac{n m^2}{T} \cdot \frac{\langle \tau (v')^4\rangle_0}{15}= n \tau T.
\end{equation}
Заметим, что
\begin{equation*}
	\eta = \frac{m \kappa}{c_p}, 
	\hspace{10 mm} 
	c_p = \frac{5}{2}.
\end{equation*}


\textbf{Уравнения Навье-Стокса}. Для $\sigma_{\alpha \beta}'$ обычно можем переписать её в виде
\begin{equation*}
	\sigma_{\alpha \beta}' = \eta \left(
		\frac{\partial u_\alpha}{\partial x_\beta} + \frac{\partial u_\beta}{\partial x_\alpha} - \frac{2}{3} \delta_{\alpha \beta} \div \vc{u}
	\right) + \zeta \delta_{\alpha \beta} \div \vc{u},
\end{equation*}
где в нашем случае для малой плотности $\zeta = 0$, в отличие от сдвиговой вязкости $\eta$. Считая $\zeta, \eta = \const$
\begin{equation*}
	\frac{\partial \sigma_{\alpha \beta}'}{\partial x_\beta}  = \eta\left(
		\nabla^2 u_\alpha + \nabla_\alpha \div \vc{u} - \frac{2}{3} \nabla_\alpha \div \vc{u}
	\right) + \zeta \nabla_\alpha \div \vc{u},
\end{equation*}
и подставляя в исходное уравнение, нахходим уравнение Навье-Стокса
\begin{equation}
	\rho \frac{d \vc{u}}{d t}  = - \vc{\nabla} P + \eta \nabla^2 \vc{u} + \left(
		\zeta + \frac{\eta}{3}
	\right) \grad \div \vc{u} + n \vc{F}.
\end{equation}


\textbf{Беспорядок}. Для наличия вмороженного беспорядка получим дополнительное слагаемое
\begin{equation*}
	\rho \frac{d \vc{u}}{d t} =\ldots - \rho \frac{\vc{u}}{\tau}.
\end{equation*}
Для получения формулы Друде, можем увидеть что почти всё в стационарном случае занулится, и получится
\begin{equation*}
	\vc{u} = \frac{n \tau}{\rho } \vc{F},
	\hspace{10 mm}
	\vc{j} = - n e \vc{u} = - n e \frac{n \tau}{\rho} (-e \vc{E}) = \frac{e^2 n \tau}{m} \vc{E}.
\end{equation*}


