
\textbf{Локальное равновесие}.
Мы в дальнейшем будем верить в локальное равновесие
\begin{equation*}
	f(\vc{r}, \vc{p}, t) \approx \left(
	 \exp \left(\frac{\varepsilon(\vc{p})-\mu(\vc{r}, t)}{T(\vc{r}, t)}\right) \pm 1
	\right)^{-1},
	\hspace{10 mm} 
	|l \, T_r'| \ll T,
\end{equation*}
где $l$ -- длина свободного пробега.


\textbf{Законы сохранения}.
Можем получить, что
\begin{equation*}
	\frac{\partial S(\vc{r}, t)}{\partial t} + \div \vc{J}_s = - \vc{J}_Q \cdot \frac{\vc{\nabla} T}{T^2} + \vc{J}_e \cdot \frac{\vc{\mathcal E}}{T},
\end{equation*}
где $\vc{\mathcal E}$ -- электрическое поле, $\vc{J}_Q$ -- поток тепла. При этом выполняются закон сохранения числа частиц:
\begin{equation*}
	\frac{\partial n(\vc{r}, t)}{\partial t} + \div \vc{J} = 0,
	\hspace{10 mm} 
	\vc{J} = \int \frac{\d^3 \vc{p}}{(2\pi \hbar)^3} f(\vc{r}, \vc{p}, t) \vc{v}(\vc{p}),
\end{equation*}
верным для любого локального в координатном пространстве интеграле столкновений. Аналогично для энергии
\begin{equation*}
	\frac{\partial E(\vc{r}, t)}{\partial t} + \div \vc{J}_E = 0,
	\hspace{10 mm} 
	\vc{J} = \int \frac{\d^3 \vc{p}}{(2\pi \hbar)^3} \varepsilon(\vc{p}, \vc{r}) f(\vc{r}, \vc{p}, t) \vc{v}(\vc{p}),
\end{equation*}
для упругих столкновений.


\textbf{Тепло}. Из термодинамики
\begin{equation*}
	T \frac{\partial S}{\partial t} = \frac{\partial Q}{\partial t} = \frac{\partial E}{\partial t} - (\mu + e \varphi) \frac{\partial n}{\partial t} = - \frac{\partial \vc{J}_E}{\partial \vc{r}} + (\mu + e \varphi) \frac{\partial \vc{J}_e}{\partial e \vc{r}}.
\end{equation*}
Здесь удобно ввести
\begin{equation*}
	\vc{\mathcal E} = \sub{\vc{\mathcal E}}{ext} - \nabla \mu / e = - \nabla (\mu + e \varphi) / e,
	\hspace{5 mm} 
	\vc{J}_Q \overset{\mathrm{def}}{=} \vc{J}_E - (\mu + e \varphi) \vc{J}_e / e,
\end{equation*}
и получить 
\begin{equation*}
	\frac{\partial Q}{\partial t} + \frac{\partial \vc{j}_Q}{\partial \vc{r}} = \vc{\mathcal E} \cdot \vc{J}_e. 
\end{equation*}
Задачу можно посмотреть на с. 103 \texttt{презентации}. 

\phantom{42}

\red{Теорема Онзагера} -- $L_{ij} = L_{ji}$.

\red{Эффект Зеебека} -- это термоэлектрический эффект, при котором появляется разность потенциалов на контакте двух проводников разной температуры.

\red{Эффект Пельтье} -- термоэлектрическое явление переноса энергии при прохождении электрического тока в месте контакта (спая) двух разнородных проводников, от одного проводника к другому.

