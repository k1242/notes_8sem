% базовая подстройка
\renewcommand{\d}{\, d}
\renewcommand{\leq}{\leqslant}
\renewcommand{\geq}{\geqslant}



% специфично под документ
\newcommand{\dC}{\,{}^\circ\textnormal{С}}
\newcommand{\un}[1]{\,\text{#1}}
\newcommand{\meas}[3]{(#1 \pm #2)\,\text{#3}}
\newcommand{\subt}[2]{#1_{\scalebox{0.5}{\textnormal{#2}}}}
\newcommand{\NMOT}{\subt{N}{MOT}}

\newcommand{\kB}{k_{\textnormal{B}}}
\newcommand{\vcap}{\sub{v}{cap}} %скорость захвата 2D-МОЛ



% \newcommand{\unewpage}{\newpage}

% \newcommand{\startp}{\vspace{13pt}}
% \makeatletter
% \renewcommand{\paragraph}{\vspace{-13pt}\@startsection{paragraph}{4}{\z@}
%                                     {3.25ex \@plus1ex \@minus.2ex}
%                                     {0em} % отступ перед текстом
%                                     {\indent \normalfont\normalsize\bfseries}
%                             }
% \makeatother

% \newcommand{\upar}[1]{\paragraph{#1} \label{#1}.}


% авторские команды
\newcommand{\vc}[1]{\boldsymbol{#1}}
\newcommand{\1}{\mathbbm{1}}
\newcommand{\T}{^{\textnormal{T}}}
\newcommand{\con}{^{\dag}}
\newcommand{\sub}[2]{#1_{\textnormal{#2}}}
\newcommand{\vp}{\vphantom{\dfrac{1}{2}}}

% операторы (просто прямой текст)
\renewcommand{\Im}{\mathop{\mathrm{Im}}\nolimits}
\renewcommand{\Re}{\mathop{\mathrm{Re}}\nolimits}
% \renewcommand{\P}{\mathop{\mathrm{P}}\nolimits}
% \newcommand{\E}{\mathop{\mathrm{E}}\nolimits}
% \newcommand{\D}{\mathop{\mathrm{D}}\nolimits}
% \newcommand{\cov}{\mathop{\mathrm{cov}}\nolimits}
\newcommand{\diag}{\mathop{\mathrm{diag}}\nolimits}
\newcommand{\card}{\mathop{\mathrm{card}}\nolimits}
\newcommand{\grad}{\mathop{\mathrm{grad}}\nolimits}
% \renewcommand{\div}{\mathop{\mathrm{div}}\nolimits}
\newcommand{\rot}{\mathop{\mathrm{rot}}\nolimits}
\newcommand{\Ker}{\mathop{\mathrm{ker}}\nolimits}
\newcommand{\spec}{\mathop{\mathrm{spec}}\nolimits}
\newcommand{\sign}{\mathop{\mathrm{sign}}\nolimits}
\newcommand{\tr}{\mathop{\mathrm{tr}}\nolimits}
\newcommand{\rg}{\mathop{\mathrm{rg}}\nolimits}
\newcommand{\const}{\textnormal{const}}


% цветной текст
\newcommand{\red}[1]{\textcolor{red}{#1}}
\newcommand{\green}[1]{\textcolor{urlcolor}{#1}}
\newcommand{\blue}[1]{\textcolor{ublue}{#1}}


% символы
\newcommand{\cmark}{\text{\ding{51}}}
\newcommand{\xmark}{\text{\ding{55}}}


% подгрузка pdf_tex картинок
% \newcommand{\incfig}[1]{%
%     \def\svgwidth{\columnwidth}
%     \import{./figures/}{#1.pdf_tex}
% }


% специфично к квантам
\newcommand{\ket}[1]{\left| #1 \right\rangle}
\newcommand{\bra}[1]{\left\langle #1 \right|}

% \newcommand{\dppp}{\frac{d^3 p}{(2 \pi \hbar)^3}}

\DeclareDocumentCommand{\bk}{m o m}{
    \IfNoValueTF{#2}{\langle #1 | #3 \rangle}{\langle #1 | #2 | #3 \rangle}
}



% \renewcommand{\thesubfigure}{\thefigure.\arabic{subfigure}} % 1.1, 1.2 %\renewcommand{\thesubfigure}{\thefigure.\asbuk{subfigure}} %1.a, 1.б 
\renewcommand{\thesubfigure}{(\asbuk{subfigure})}


\definecolor{mygray}{gray}{0.6}
\definecolor{mygreen}{RGB}{8,99,44}
\definecolor{myred}{RGB}{164,0,0}



\newcommand{\iitem}[1]{\item[\textcolor{seahorse}{\scalebox{0.7}{$\blacksquare$}}]{#1}}
