При реализации квантового симулятора на атомах Tm в нашей лаборатории возникла следующая проблема: из-за большого потока атомов из печи, на зеркало, которое находится напротив печи, напыляются атомы. Этот процесс приводит к ухудшению отражательных свойств зеркала, уменьшению стабильности установки и существенному ограничению времени жизни установки. В данной работе описано решение этой проблемы, путём уменьшения потока атомов из печи, при сохранение потока загрузки МОЛ. Оптимизация количества атомов Tm в МОЛ позволила сохранить поток атомов, загружающихся в МОЛ. Также в работе рассматривается альтернативное решение проблемы напыления атомов тулия: использование источника охлажденного атомарного пучка Tm, не требующего встречного лазерного пучка. Основой альтернативного источника является 2D-МОЛ.