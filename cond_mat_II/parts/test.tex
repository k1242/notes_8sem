\section{Сильные корреляции}

% \subsection*{Краткая история}
% С. П. Шубин (полярная модель, первичное квантование) -> Джон Хаббард (1931-1980) 
% погиб на каторге в 1938  							| -> Мотт			
% -> Филип Андерсон (1923-2020): спиновые стёкла -> Л. А. Максимов

% в зоне 2N электронов, N -- количество атомов

Рассмотрим электроны в решётке
\begin{equation*}
	\hat{H} = \sum_{k \sigma} \varepsilon_k c_{k \sigma}\con c_{k \sigma},
	\hspace{0.5cm} \Rightarrow \hspace{0.5cm}
	\hat{H} = \sum_{ij\sigma} t_{ij} c_{ij}\con c_{j \sigma}
\end{equation*}
Разделим спины $\uparrow, \downarrow$ на $\hat{a}$ и $\hat{b}$ и учитываем только ближайших соседей
\begin{equation}
	\sub{\hat{H}}{Hubbard} \overset{\mathrm{def}}{=} \HH =  t \sum_{\langle ij\rangle} \left(
		a_i\con a_j + b_i\con b_j
	\right) + 
	U \sum_i a_i\con a_i b_i\con b_i,
\end{equation}
где добавили взаимодействие электронов находящихся в одном узле. 

Собственно, при $U=0$ -- металл, при $t=0$ -- диэлектрик на $n=1$, где-то посередине происходит переход. Введём $X$-операторы
\begin{equation*}
	X_i^{pq} = |p \rangle \langle q|,
	\hspace{10 mm} 
	\ket{p} = \ket{0},\, \ket{\sigma},\, \ket{\bar{\sigma}},\, \ket{2}.
\end{equation*}
Условие, что кто-то должен в узле быть
\begin{equation*}
	X_i^{00} + \sum_\sigma X_i^{\sigma \sigma} + X_i^{22} = \1.
\end{equation*}
Есть выражение $X$-операторов через $c$ и $c\con$. В пределе $t \ll U$ и $n=1$
\begin{equation*}
	\HH \sim \sub{H}{Heis} = J \sum_{\langle ij\rangle} \hat{\vc{S}}_i \hat{\vc{S}}_j,
	\hspace{5 mm} 
	J \sim - t^2/U.
\end{equation*}



% В красноярске есть очень мощное гнездо хаббардистов
% z -- число соседей

% Большие знания рождают печаль






