\section{Квантовые вычисления на гейтах}

\begin{to_thr}[Gottesman-Knill]
    Поворот и CNOT полиномиально моделируются на классическом компьютере.
\end{to_thr}


Нужен гейт не-клиффордовской операции. 

\begin{to_thr}[Соловея-Китаева]
    Покрыли поверхность сферы точками, хотим с ошибкой не более $\varepsilon$ приближаться к ней. По идея имея поворот на $\alpha$, можно было бы прийти за $O(\varepsilon^{-1})$, но на самом деле можно за $O(\log^c \frac{1}{l})$, где $c \approx 2$. 
\end{to_thr}


Таким образом хотим несимулируемости за полиномиальное время и достаточно быстрого приближения ко всем точкам. 

\textbf{Процесс}. 
Начинаем с инициализации состояния, унитарно $U$ его преобразуем, а потом проводим селективные по кубитам измерения.  Экспериментально удобно в декомпозицию $U$ в 1q, 2q -вентили. 

Сейчас умеет 10-100 кубитов, 20-30 операций. Например для алгоритма Шора на $2048$ битов нужно 20млн кубитов и 8 часов на модельном сверхпроводниковом процессоре, где операция выполняется $\mu$с.



\subsection*{One-way quantum computing}

Ресурс -- кластерное состояние, уже запутанное. Далее делаем только однокубитные операции и измерения. Особенно важно, что можем делать условные операции, когда в зависимости от результата измерения меняются одночастичные операции (\textit{feedforward processing}). Предложено Raussendorf в 2001. 

Сейчас умеют делать 10-20 кубитов. Но выглядит не очень оптимистично, хотя и лучшее что есть на оптике. 



\subsection*{Adiabatic quantum computing}


Рассмотрим систему с гамильтонианом
\begin{equation*}
	\hat{H}^t = \sub{\hat{H}}{kin}^t + \sub{\hat{H}}{int}^t,
\end{equation*}
для гейтовой модели. В некоторых задачах однако мы можем посчитать гамильтониан, если выполняется адиабатическая теорема квантовой механики, то есть меняя медленно параметры переходим из основного состояния, можем сделать переход
\begin{equation*}
	H = \lambda H_0 + (1-\lambda) \sub{H}{aim},
\end{equation*}
где $\lambda$ медленно меняем от 1 до 0.

Любую задачу квантовых вычислений можно свести к поиску основного состояния для некоторого гамильтонинана, который ищется эффективно. Реальных прототипов нет, но есть устройства которые mimic -- квантовый отжиг. Они умеют готовить гамильтонианы не любые, а узкий класс задач. DWave скорее делают что получается, но не адиабатические квантовые вычисления. 



\subsection*{Вариационные квантовые алгоритмы}

Недавно показали, что вариационная модель квантовых вычислений претендует на универсальность. Собственно пусть $U(\theta)$, где $\theta$-- некоторый параметр, тогда
\begin{equation*}
	\text{cost} = \bk{\psi_\theta}[\hat{H}]{\psi_\theta} \to \min,
\end{equation*}
где $U(\theta)$ -- про 5-10 операций для чего-то содержательного. 
% VQE
% QAOA
% Jacob Biamonte
Хотим найти как раз значение минимума этого функционала над $\ket{\psi_\theta}$. 




% \section{Квантовые симуляторы}

% \subsection*{QUBO: MaxCut problem}



