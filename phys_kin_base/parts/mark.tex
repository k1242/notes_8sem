\textbf{Детальный баланс}. Марковский процесс c матрицей $T$ называем \textit{обратимым}, если для стационарного распределения $\vc{x}$ выполняется
\begin{equation*}
	x_i T_{ij} = x_j T_{ji}.
\end{equation*}
Для симметричных матриц перехода всегда будет детальный баланс. Есть тесная связь с $H$-теоремой, а именно условие детального баланса является \textit{достаточным} для строгого возрастания энтропии в изолированных системах. Ещё \textit{соотношения Онзагера} следуют из принципа детального баланса в линейном приближении вблизи равновесия. 
% https://en.wikipedia.org/wiki/Detailed_balance

\textbf{Уравнение Чепмена-Колмогорова}. Далее говорим о \textit{пропагаторе}
\begin{equation*}
	T(x, t| x', t') = \delta(x - \Phi_{t-t'}(x')),
\end{equation*}
иначе вероятности перехода из $x$ в $x'$ за время $t-t'$, где $\Phi_{t-t'}(x')$ -- эволюция системы. 

В дискретном случае \textit{уравнение Чепмена-Колмогорова} --
\begin{equation*}
	T_{t+s} = T_t T_s,
\end{equation*}
матрицы переходов просто перемножаются. 


Нам интереснее будет работать с диференциальным уравнением Чепмена-Колмогорова
\begin{equation}
	\partial_t P(n, t) = \sum_{m=-\infty}^{+\infty} \left(
		W(n|m, t) P(m,t) - W(m|t,t) P(n, t)
	\right),
\end{equation}
где первое слагаемое -- приток из системы, второе -- отток в систему.