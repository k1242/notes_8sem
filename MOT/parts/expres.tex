Откалибровано абсолютное значение температуры используемой печи. Описана схема детектирования атомов в МОЛ, описан алгоритм обработки фотографии и определения температуры облака атомов $T$ и полного количества атомов $N$. Снята зависимость $N$ от отстройки лучей МОЛ, токов ЗЗ и отстройки лучей ОП. Определены оптимальные значения, соответсвующие $\subt{\delta}{ОП} = -1.7\Gamma$, $\subt{\delta}{МОЛ} = -13.5\Gamma$ и $\sub{I}{small} = 18\,$А, $\sub{I}{big} = 35\,$А. Найденные значения соответсвуют температуре печи $T_2 = 680 \dC$ и $\sub{\Phi}{sol}(T_2) \sim 1 \times 10^{11}\,\text{c}^{-1}$. Раннее наша установка работала при температуре $T_1 = 730 \dC$, что соответсвует потоку атомов $\sub{\Phi}{sol}(T_1) \sim 5 \times 10^{11}\,\text{c}^{-1}$. Таким образом время работы установки было увеличено в 5 раз до 5 месяцев непрерывной работы, при этом поток загрузки МОЛ $\sub{\Phi}{load} \sim 10^8$ остался на том же уровне.