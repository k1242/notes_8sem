\textbf{Уравнение Ланжевена}. Стохастическое уравнение Ланжевена (математика и физика)
\begin{equation}
	\frac{d x}{d t}  = g(x(t)) + \xi(t),
	\hspace{10 mm} 
	\boxed{
	\frac{d p}{d t} = -\gamma p - \nabla U + \xi_p(t),
	\hspace{5 mm} 
	\langle \xi_p (t) \xi_p (t')\rangle = D \delta(t-t')
	}.
\end{equation}
где $\xi(t)$ -- случайный процесс. Заметим, что
\begin{equation*}
	T(x, t|x',t') = \langle \delta(x-\Phi_{t-t'}(x'))\rangle_\xi,
\end{equation*}
и можем разложиться в ряд
\begin{equation*}
	E(x, t+ \delta t| x' , t) = \langle \delta(x-\delta x(t)-x')\rangle 
	= \left(
		1 + \langle \delta x(t)\rangle \frac{d }{d x'} + \frac{1}{2} \langle |\delta x(t)|^2\rangle \frac{d^2}{d {x'}^2}  + \ldots
	\right) \delta(x-x')
\end{equation*}
Далее заменим средние первого порядка на $F_1$ и второго на $F_2$, а остальные малы.

\textbf{Уравнение Фокера-Планка}. Получаем \textit{уравнение Фокера}
\begin{equation*}
	\frac{\partial }{\partial t} T(x, t|x_0, t_0) = - \frac{d }{d x} \left(
		F_1(x) T(x, t|x_0, t_0)
	\right) + \frac{1}{2} \frac{d^2}{d x^2} \left(
		F_2(x) T(x, t|x_0, t_0)
	\right).
\end{equation*}
Вообще $F$ мы знаем из уравнения Ланжевена
\begin{equation*}
	F_1(x) \delta t = \langle \delta x(t)\rangle = g(x),
	\hspace{10 mm} 
	F_2(x) = \frac{1}{\delta t} \int_{t}^{t+\delta t} \langle \xi(t_1) \xi(t_2)\rangle \d t_1 \d t_2 = D.
\end{equation*}
Подробнее в \texttt{билетах} на странице 45.


Подставим для Фоккера-Планка
\begin{equation*}
	g(x) = (v(p),\, F(r)) = 
	\frac{d }{d t} \begin{pmatrix}
		r \\ p
	\end{pmatrix} = \begin{pmatrix}
		v(p) \\ - \gamma p - \nabla U + \xi_p(t)
	\end{pmatrix},
	\hspace{10 mm} 
	T(x, t|x_0,t_0) = \langle f(r,p, t)\rangle_\xi.
\end{equation*}
И в более конкретизированном виде для $P = \langle f(r, p, t)\rangle$ уравнение Фоккера-Планка имеет вид
\begin{equation}
\boxed{
	\frac{\partial }{\partial t}P + \frac{\partial }{\partial r} vP+ \frac{\partial}{\partial p}  (-\gamma p - \nabla U) P - \frac{D}{2} \frac{\partial }{\partial p^2} P = 0
}
\end{equation}
где $D=2m T \gamma$ -- соотношение Эйнштейна. 
% хороший вывод -- билет 21



\textbf{Уравнение Смолуховского}. Для уравнения Ланжевена в пределе очень вязкой жидкости
\begin{equation*}
	\gamma p = - \nabla U +\xi_p (t).
\end{equation*}
Для концентрации частиц $n = \int f \d p$, можем в этом приближении в одномерии записать \textit{уравнение Смолуховского}
\begin{equation*}
	\frac{\partial }{\partial t} \langle n\rangle_\xi = \frac{1}{\gamma m} \frac{\partial }{\partial x} \left(
		\frac{\partial U}{\partial x}  + \frac{D}{2\gamma m} \frac{\partial }{\partial x} 
	\right) \langle n\rangle_\xi.
\end{equation*}


\textbf{Соотношения Эйнштейна}. В отсутствие внешних сил верно, что
\begin{equation*}
	\frac{\partial n}{\partial t} + D \nabla^2 n= 0,
	\hspace{10 mm} 
	n_0 (\vc{r}, t) = \frac{1}{(4 \pi D t)^{3/2}} \exp\left(- \frac{r^2}{4 D t}\right),
\end{equation*}
где $n_0(t = 0) = \delta(\vc{r})$. Можем сразу найти среднеквадратичное отклонение
\begin{equation*}
	\langle r^2\rangle = \int n_0 r^2 \d^3 r = 6 D t.
\end{equation*}
В равновесных условиях диффузионный поток отсутствует $\vc{j} = - D \nabla n + b \vc{F} n = 0$, функция распределения частиц во внешнем поле $U(\vc{r})$ совпадает с больцмановской $n(\vc{r}) \sim \exp\left( - U / T\right), \vc{F} = - \nabla U$, откуда получаем соотношение Эйнштейна
\begin{equation}
 	\boxed{D = b T}.
 \end{equation} 